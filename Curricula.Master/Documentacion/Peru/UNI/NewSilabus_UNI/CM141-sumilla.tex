\begin{sumilla}

\curso{CM141. C\'alculo Vectorial I}{Obligatorio}{CM141}

\begin{fundamentacion}
Comprender los fundamentos del C\'alculo Vectorial en el plano y del espacio, y adquirir habilidades que le permitan usar los conceptos estudiados, en el desarrollo de otras asignaturas, as\'i como tambi\'en en la soluci\'on de problemas vinculados a su especialidad.
\end{fundamentacion}

\begin{objetivosdelcurso}
\item Comprender los principios C\'alculo Vectorial.
\item Utilizar los conceptos de la Geometr\'ia vectorial espacial.
\item Entender y aplicar el desarrollo de c\'onicas.
\end{objetivosdelcurso}

\begin{outcomes}
\ExpandOutcome{a}
\ExpandOutcome{i}
\ExpandOutcome{j}
\end{outcomes}

\begin{unit}{Vectores en el Plano}{Hasser97}{4}
   \begin{topicos}
      \item Sistemas de coordenadas cartesianas; producto cartesiano $RxR$. Sus elementos. Espacio Vectorial Bidimensional. Definici\'on
      \item Representaci\'on geom\'etrica de vectores. Paralelismo de vectores. Longitud de un  vector. Paralelismo de Vectores. Producto interno en $R^2$ Propiedades. Ortogonalidad de vectores
      \item Producto Escalar. Propiedades. \'Angulo entre vectores Proyecci\'on ortogonal. Componentes
      \item Combinaci\'on Lineal de Vectores. Independencia lineal de vectores. Bases
   \end{topicos}

   \begin{objetivos}
      \item Describir matem\'aticamente los vectores en el plano
      \item Conocer las propiedades de los vectores en el plano y aplicarlos en la soluci\'on de problemas
   \end{objetivos}
\end{unit}

\begin{unit}{Geometr\'ia Vectorial en el Plano}{Hasser97}{8}
\begin{topicos}
	\item El Plano euclideano. Definici\'on. Punto. Recta. Distancia entre dos puntos
	\item La recta. Sus ecuaciones. Posiciones relativas de las rectas. Paralelismo de rectas. Ortogonalidad de rectas. Distancia de un punto a una recta
      \item Intersecci\'on de rectas. Ecuaciones Lineales simult\'aneas. Pendiente de una recta. \'Angulo entre rectas. \'Area del tri\'angulo. \'Area del pol\'igono
\end{topicos}
   \begin{objetivos}
      \item Describir matem\'aticamente la Geometr\'ia Vectorial en el Plano
      \item Conocer y aplicar conceptos de rectas para resolver problemas
   \end{objetivos}
\end{unit}

\begin{unit}{Vectores en el Espacio}{Burgos94}{12}
\begin{topicos}
      \item Espacio Vectorial Tridimensional. Definici\'on: Igualdad de vectores. Adici\'on de Vectores. Multiplicaci\'on de un vector por un n\'umero real. Representaci\'on Geom\'etrica de los Vectores. Paralelismo de vectores
      \item Longitud de un vector. Propiedades. Vectores unitarios. Producto Escalar. Propiedades. Ortogonalidad de vectores. \'Angulo entre vectores. Proyecci\'on ortogonal
      \item Combinaci\'on lineal de vectores. Independencia lineal de vectores. Bases
      \item Producto Vectorial. Definici\'on. Significado geom\'etrico. Triple producto escalar. Significado geom\'etrico. Caracterizaci\'on de la independencia lineal de tres vectores con el triple producto escalar
\end{topicos}

   \begin{objetivos}
      \item Describir matem\'aticamente los vectores en el espacio
      \item Conocer las propiedades de los vectores en el espacio y aplicarlos en la soluci\'on de problemas
   \end{objetivos}
\end{unit}

\begin{unit}{Geometr\'ia Vectorial Espacial}{Venero94, Edwards96}{12}
\begin{topicos}
      \item Espacio Euclideano tridimensional. Definici\'on. Punto, recta, plano. Distancia entre dos puntos
      \item La Recta. Sus Ecuaciones. Posici\'on relativa de rectas; paralelismo de rectas y \'Angulo entre rectas. Distancia de un punto a una recta. Distancia entre rectas. Casos
      \item El Plano. Sus ecuaciones. Posiciones relativas de planos. Paralelismo y \'Angulo entre planos. Intersecci\'on de rectas y planos. Distancia de un punto a un plano. Caracterizaci\'on de la independencia lineal de  vectores con la intersecci\'on de planos. \'Area del paralelogramo. Volumen del paralelep\'ipedo y del tetraedro, etc.
	\end{topicos}

   \begin{objetivos}
      \item Describir matem\'aticamente la Geometr\'ia Vectorial Espacial
      \item Conocer y aplicar conceptos de planos y rectas para resolver problemas
   \end{objetivos}
\end{unit}

\begin{unit}{C\'onicas: (en forma vectorial)}{Granero, Edwards96}{12}
\begin{topicos}
      	\item Coordenadas Homog\'eneas o Absolutas en el plano
	\item Ecuaci\'on de la Recta en coordenada homog\'eneas
	\item Definici\'on de C\'onica y su Ecuaci\'on general interpretaci\'on geom\'etrica
	\item Polar de un punto y polo de una  rectas
	\item Intersecci\'on de una c\'onica con una recta
	\item Puntos singulares de una c\'onica: C\'onicas degeneradas
	\item Composici\'on de las c\'onicas degeneradas. (|A| = o)
	\item Clasificaci\'on de las c\'onicas mediante sus intersecci\'on con la recta $X_3 = 0$
	\item C\'onicas Imaginarias
	\item Clasificaci\'on General de las c\'onicas
	\item Rectas Tangentes a una c\'onica: As\'intotas
	\item Elementos principales de las c\'onicas no degeneradas
	\item Focos y Directrices
	\item Reducci\'on de la Ecuaci\'on General de las c\'onicas no degeneradas a formas c\'onicas
	\item Obtenci\'on de los Coeficientes de la forma can\'onica para la Elipse, hip\'erbolas y par\'abolas
	\item Determinaci\'on anal\'itica de las c\'onicas
	\item Haces lineales de c\'onicas
   \end{topicos}

   \begin{objetivos}
      \item Describir matem\'aticamente las c\'onicas
      \item Conocer y aplicar conceptos de c\'onicas en la soluci\'on de problemas
   \end{objetivos}
\end{unit}

\begin{bibliografia}
\bibfile{CM141}
\end{bibliografia}
\end{sumilla}
