\newcommand{\DocumentVersion}{V2.0}
\newcommand{\fecha}{\today}
\newcommand{\YYYY}{2010\xspace}
\newcommand{\Semester}{2010-1\xspace}
\newcommand{\city}{Arequipa\xspace}
\newcommand{\country}{Perú\xspace}
\newcommand{\dictionary}{Español\xspace}
\newcommand{\GraphVersion}{2\xspace}
\newcommand{\CurriculaVersion}{2\xspace}
\newcommand{\OutcomesList}{a,b,c,d,e,f,g,h,i,j,k,l,m,HU}

\newcommand{\University}{Sociedad Peruana de Computación\xspace}
\newcommand{\InstitutionURL}{http://www.spc.org.pe\xspace}
\newcommand{\underlogotext}{}

\newcommand{\FacultadName}{Computación\xspace}
\newcommand{\DepartmentName}{Sistemas de Información\xspace}
\newcommand{\SchoolFullName}{Programa Profesional de Sistemas de Información\xspace}
\newcommand{\SchoolFullNameBreak}{\SchoolFullName}
%\newcommand{\SchoolFullName}{Programa Profesional de Sistemas de Información\xspace}
\newcommand{\SchoolShortName}{Sistemas de Información\xspace}
\newcommand{\SchoolAcro}{PPSI\xspace}
\newcommand{\SchoolURL}{http://www.spc.org.pe/education/PCC/Peru/IS-SPC/}

\newcommand{\GradoAcademico}{Bachiller en Sistemas de Información}
\newcommand{\TituloProfesional}{Licenciado en Sistemas de Información}
\newcommand{\GradosyTitulos}%
{\begin{description}%
\item [Grado Académico: ] \GradoAcademico%
\item [Titulo Profesional: ] \TituloProfesional%
\end{description}%
}

\newcommand{\doctitle}{Propuesta de malla curricular para Programas Profesionales de \SchoolShortName {\Large\footnote{\SchoolURL}}\xspace}

\newcommand{\AbstractIntro}{Este documento representa el informe final de la
\University (\textit{\InstitutionURL}) para Programas Profesionales de \SchoolShortName.
}
\newcommand{\OtherKeyStones}{}

\newcommand{\profile}{%
El perfil profesional de este programa profesional puede ser mejor entendido a partir de
\OnlyMainDoc{la Fig. \ref{fig.cs} (Pág. \pageref{fig.cs})}\OnlyPoster{las figuras del lado derecho}.
Este profesional tiene a la computación como un medio para hacer que las organizaciones de todo tamaño funcionen
de forma mucho más eficiente.

Nuestro perfil profesional está orientado a ser generador de innovación a niovel organizacional gracias
al uso adecuado de las herramientas computacionales.
Nuestra formación profesional tiene 2 pilares fundamentales:
Un contenido de acuerdo a recomendaciones internacionales y una orientación marcada a la
innovación dentro de un entorno organizacional.
}

\newcommand{\HTMLFootnote}{Generado por <A HREF='http://socios.spc.org.pe/ecuadros/'>Ernesto Cuadros-Vargas</A> <ecuadros AT spc.org.pe>, <A HREF='http://www.spc.org.pe/'>Sociedad Peruana de Computacin, Per</A> <BR> basado en la Computing Curricula de <A HREF='http://www.computer.org/'>IEEE-CS</A>/<A HREF='http://www.acm.org/'>ACM</A>/<A HREF='http://home.aisnet.org/'>AIS</A>}

