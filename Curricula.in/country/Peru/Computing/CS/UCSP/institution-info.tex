\newcommand{\DocumentVersion}{2016}
\newcommand{\fecha}{\today}
\newcommand{\city}{Arequipa\xspace}
\newcommand{\country}{Perú\xspace}
\newcommand{\dictionary}{Español\xspace}
\newcommand{\SyllabusLangs}{Español,English}
\newcommand{\GraphVersion}{2\xspace}

%newcommand{\CurriculaVersion}{1\xspace} % Malla 2006: 1, Malla 2016: 2
%newcommand{\YYYY}{2006\xspace}          % Plan 2006
%newcommand{\Range}{7-10}                % Plan 2016 1-8, Plan 2006 7-10

\newcommand{\CurriculaVersion}{2016\xspace} % Malla 2006: 1, Malla 2016: 2 i.e ../Curricula.in/lang/Espanol/CS.tex/CS2016-dependencies.tex
\newcommand{\YYYY}{2016\xspace}          % Plan 2006, 2010, 2016
\newcommand{\Range}{1-2}                % Plan 2016 1-8, Plan 2006 7-10

\newcommand{\Semester}{2016-2\xspace}
\newcommand{\equivalences}{2006,2010} %  {2006,2010}

% convert ./fig/UCSP.jpg ./html/img3.png
% cp ./fig/big-graph-curricula.png ./html/img18.png
% convert ../Curricula2.0.out/Peru/CS-UCSP/cycle/2014-1/Plan2016/fig/UCSP.jpg ../Curricula2.0.out/Peru/CS-UCSP/cycle/2014-1/Plan2016/html/img3.png
% cp ../Curricula2.0.out/Peru/CS-UCSP/cycle/2014-1/Plan2016/fig/big-graph-curricula.png ../Curricula2.0.out/Peru/CS-UCSP/cycle/2014-1/Plan2016/html/img18.png

\newcommand{\OutcomesList}{a,b,c,d,e,f,g,h,i,j,k,l,m,n,ñ,o}
\newcommand{\logowidth}{20cm}

\newcommand{\University}{Universidad Católica San Pablo\xspace}
\newcommand{\InstitutionURL}{\htmladdnormallink{http://www.ucsp.edu.pe}{http://www.ucsp.edu.pe}\xspace}
\newcommand{\underlogotext}{}
\newcommand{\FacultadName}{}
\newcommand{\DepartmentName}{Ciencia de la Computación\xspace}
\newcommand{\SchoolFullName}{Escuela Profesional de Ciencia de la Computación\xspace}
\newcommand{\SchoolFullNameBreak}{Escuela Profesional de \\Ciencia de la Computación\xspace}
\newcommand{\SchoolShortName}{Ciencia de la Computación\xspace}
\newcommand{\SchoolAcro}{PPII\xspace}
\newcommand{\SchoolURL}{\href{http://cs.ucsp.edu.pe}{http://cs.ucsp.edu.pe}\xspace}

\newcommand{\GradoAcademico}{Bachiller en Ciencia de la Computación\xspace}
\newcommand{\TituloProfesional}{Licenciado en Ciencia de la Computación\xspace}
\newcommand{\GradosyTitulos}%
{\begin{description}%
\item [Grado Académico: ] \GradoAcademico\xspace y%
\item [Titulo Profesional: ] \TituloProfesional%
\end{description}%
}

\newcommand{\doctitle}{Plan Curricular \YYYY\xspace del \SchoolFullName\\ \SchoolURL}

\newcommand{\AbstractIntro}{Este documento representa el informe final de la nueva 
malla curricular \YYYY de la \SchoolFullName de la \University (\textit{\InstitutionURL}) 
en la ciudad de \city-\country.}

\newcommand{\OtherKeyStones}%
{Un pilar que merece especial consideración en el caso de la \University es el aspecto de 
valores humanos, básicos y cristianos debido a que forman parte fundamental 
de los lineamientos básicos de la existencia de la institución.\xspace}

\newcommand{\profile}{%
El perfil profesional puede ser mejor entendido a partir de
\OnlyMainDoc{la Fig. \ref{fig.cs} (Pág. \pageref{fig.cs})}\OnlyPoster{las figuras del lado derecho}. 
Este profesional tiene como objetivo principal ser el impulsor del desarrollo de nuevas 
tecnologí­as computacionales con calidad internacional que puedan ser útiles a nivel local, nacional e internacional.
Nuestro perfil profesional también está orientado a ser generador de puestos de empleo a través de la innovación permanente. 
Nuestra formación profesional tiene 3 pilares fundamentales: 
un contenido computacional de acuerdo a normas internacionales (CS2013), una orientación marcada a la innovación ambos enriquecidos por una sólida
Formación Humana.
}

\newcommand{\mission}{La Universidad Católica San Pablo es una comunidad académica animada por las orientaciones y vida de la Iglesia Católica que, 
a la luz de la fe y con el esfuerzo de la razón, busca la verdad y promueve la formación integral de la persona mediante actividades 
como la investigación, la enseñanza y la extensión, para contribuir con la 
configuración de la cultura conforme a la identidad y despliegue propios del ser humano.\xspace}
