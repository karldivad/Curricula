\documentclass[final]{article}
\input{current-institution}
\usepackage{\InStyAllDir/syllabus-final}
\usepackage{enumitem} 
\usepackage[backend=bibtex,style=alphabetic]{biblatex}
% \bibliographystyle{apalike}
\addbibresource{--BIBFILE--}

\newcommand{\INST}{}
\newcommand{\AREA}{}

\begin{document}

\begin{tabularx}{\textwidth}{p{3cm}cX}
\includegraphics[width=3cm]{\InLogosDir/\INST} 
&
\begin{minipage}{0.75\textwidth}
\begin{center}
{\noindent\Large\bf\University\\
\SchoolFullNameBreak \\
<<SYLLABUSOFCOURSE>> -- <<PERIOD>> \Semester}

% \section*{--COURSE_CODE--. --COURSE_NAME-- (--COURSE_TYPE--)}\label{sec:--COURSE_CODE--}%
\section*{}\label{sec:--COURSE_CODE--}%
\addcontentsline{toc}{subsection}{--COURSE_CODE--. --COURSE_NAME-- (--COURSE_TYPE--)}%
\end{center}
\end{minipage}
\end{tabularx}

\addtocounter{SyllabiSectionCount}{1}
{\noindent\bf \arabic{SyllabiSectionCount}. <<CourseCodeandName>>:} --COURSE_CODE--. --COURSE_NAME-- \label{sec:--COURSE_CODE--}

\addtocounter{SyllabiSectionCount}{1}
{\noindent\bf \arabic{SyllabiSectionCount}. <<Credits>>:} --CREDITS--

\addtocounter{SyllabiSectionCount}{1}
{\noindent\bf \arabic{SyllabiSectionCount}. <<HoursTheoryAndLab>>:} --HOURS--

\addtocounter{SyllabiSectionCount}{1}
{\noindent\bf \arabic{SyllabiSectionCount}. <<Professor>>}
--PROFESSOR_SHORT_CVS--
<<AttentionwithAppointment>>

\addtocounter{SyllabiSectionCount}{1} {\noindent\bf \arabic{SyllabiSectionCount}. <<Bibliography>>}
\defbibheading{bibliography}[\bibname]{}
\printbibliography

\addtocounter{SyllabiSectionCount}{1}
{\noindent\bf \arabic{SyllabiSectionCount}. <<InformationAbouttheCourse>>}
\begin{enumerate}[label=(\alph*)]
\item {\bf <<BriefDescription>>}
--JUSTIFICATION--
 
\item {\bf <<Prerequisites>>:}
--PREREQUISITES--
 
\item {\bf <<TypeOfCourse>>:} --COURSE_TYPE--
\end{enumerate}

\addtocounter{SyllabiSectionCount}{1}
{\noindent\bf \arabic{SyllabiSectionCount}. <<Competences>>}
--FULL_GOALS--

\addtocounter{SyllabiSectionCount}{1}
{\noindent\bf \arabic{SyllabiSectionCount}. <<ContributionToOutcomes>>}
--FULL_OUTCOMES--

\addtocounter{SyllabiSectionCount}{1}
{\noindent\bf \arabic{SyllabiSectionCount}. <<ListOfTopics>>}
\begin{enumerate}
--LIST_OF_TOPICS--
\end{enumerate}



% \addtocounter{SyllabiSectionCount}{1}
% \begin{center}
% \begin{tabularx}{\textwidth}{|X|}      \hline
% {\bf \arabic{SyllabiSectionCount}. <<CONTENTS>>}                      \\ \hline
% \end{tabularx}
% \end{center}
% 
% %%%%%%%%%%%%%%%%%%%%%%%%%%%%%%%%%%%%%%%%%%%%%%%%%%%%%%%%%%%%%%%%%%%%%%%%%%%%%%
% \setcounter{SyllabiUnitCount}{0}
% BEGINUNIT--
% % A new unit begins here
% \addtocounter{SyllabiUnitCount}{1}
% \begin{center}
% \begin{tabularx}{\textwidth}{|X|X|}                 \hline
% \multicolumn{2}{|l|}{{\bf <<UNIT>> \arabic{SyllabiUnitCount}: --UNIT_TITLE-- (--HOURS--)}} \\ \hline
% \multicolumn{2}{|l|}{{\bf Nivel Bloom: --BLOOM_LEVEL--}} \\ \hline
% {\bf GENERAL OBJECTIVE}  & {\bf CONTENT}                    \\ \hline
% UNIT_GOAL--
% & 
% UNIT_CONTENT--
% \\ \hline
% \multicolumn{2}{|l|}
% {\begin{minipage}{0.95\textwidth}
% {\bf Lecturas:} --CITATIONS--
% \end{minipage}
% }
% \\ \hline
% \end{tabularx}
% \end{center}
% 
% --ENDUNIT--
% %%%%%%%%%%%%%%%%%%%%%%%%%%%%%%%%%%%%%%%%%%%%%%%%%%%%%%%%%%%%%%%%%%%%%%%%%%%%%%

\addtocounter{SyllabiSectionCount}{1}
\begin{center}
\begin{tabularx}{\textwidth}{|X|}      \hline
\arabic{SyllabiSectionCount}. METODOLOG�A  \\ \hline
\begin{evaluation}
	\item El profesor del curso presentar� clases te�ricas de los temas se�alados en el programa propiciando la intervenci�n de los alumnos. 
	\item El profesor del curso presentar� demostraciones para fundamentar clases te�ricas.
	\item El profesor y los alumnos realizar�n pr�cticas
	\item Los alumnos deber�n asistir a clase habiendo le�do lo que el profesor va a presentar. 
	De esta manera se facilitar� la comprensi�n y los estudiantes estar�n en mejores condiciones de hacer consultas en clase.
\end{evaluation}
\\ \hline
\end{tabularx}
\end{center}

\addtocounter{SyllabiSectionCount}{1}
\begin{center}
\begin{tabularx}{\textwidth}{|X|}      \hline
\arabic{SyllabiSectionCount}. EVALUACIONES  \\ \hline
\begin{evaluation}
	\item[Evaluaci�n Permanente 1] : 20 \%
	\item[Examen Parcial] : 30 \%
	\item[Evaluaci�n Permanente 2] : 20 \%
	\item[Examen Final] : 30 \%
\end{evaluation}
\\ \hline
\end{tabularx}
\end{center}


\end{document}