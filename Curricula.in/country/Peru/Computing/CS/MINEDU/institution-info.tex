\newcommand{\DocumentVersion}{2010}
\newcommand{\fecha}{2 de Enero de 2017}
\newcommand{\city}{Lima\xspace}
\newcommand{\country}{Perú\xspace}
\newcommand{\dictionary}{Español\xspace}
\newcommand{\SyllabusLangs}{Español,English}
\newcommand{\GraphVersion}{2\xspace}

% newcommand{\CurriculaVersion}{1\xspace} % Malla 2006: 1, Malla 2010: 2
% newcommand{\YYYY}{2006\xspace}          % Plan 2006
% newcommand{\Range}{8-10}                % Plan 2010 1-8, Plan 2006 7-10
% newcommand{\SchoolShortName}{Ingeniería Informática\xspace}
% newcommand{\SchoolFullName}{Escuela Profesional de \SchoolShortName}
% newcommand{\SchoolFullNameBreak}{\FacultadName\\ Escuela Profesional de \\\SchoolShortName\xspace}

% Para UTEC 2017
%newcommand{\CurriculaVersion}{2016\xspace} % Malla 2006: 1, Malla 2010: 2
%newcommand{\YYYY}{2017\xspace}             % Plan 2006
%newcommand{\Range}{1-10}                   % Plan 2010 1-8, Plan 2006 7-10
%newcommand{\Semester}{2017-II\xspace}      

%Para UTEC 2018
\newcommand{\CurriculaVersion}{2018\xspace} % Malla 2006: 1, Malla 2010: 2
\newcommand{\YYYY}{2018\xspace}             % Plan 2006
\newcommand{\Range}{1-10}                   % Plan 2010 1-8, Plan 2006 7-10
\newcommand{\Semester}{2018-II\xspace}      

% convert ./fig/UTEC.jpg ./html/img3.png
% cp ./fig/big-graph-curricula.png ./html/img18.png
% convert ../Curricula2.0.out/Peru/CS-UTEC/cycle/2014-1/Plan2010/fig/UTEC.jpg ../Curricula2.0.out/Peru/CS-UTEC/cycle/2014-1/Plan2010/html/img3.png
% cp ../Curricula2.0.out/Peru/CS-UTEC/cycle/2014-1/Plan2010/fig/big-graph-curricula.png ../Curricula2.0.out/Peru/CS-UTEC/cycle/2014-1/Plan2010/html/img18.png

% newcommand{\OutcomesList}{a,b,c,d,e,f,g,h,i,j,k,l,m,HU,FH,TASDSH}
\newcommand{\OutcomesVersion}{V1}
\OutcomesList{V1}{a,b,c,d,e,f,g,h,i,j,k,l,m,n,ñ,o}
\OutcomesList{V2}{1,2,3,4,5,6,7,8,9,10}

\newcommand{\logowidth}{6.3cm}
\newcommand{\InstitutionURL}{\htmladdnormallink{http://www.utec.edu.pe}{http://www.utec.edu.pe}\xspace}

\newcommand{\UniversityEspanol}{Universidad de Ingeniería y Tecnología\xspace}
\newcommand{\UniversityEnglish}{University of Engineering and Technology\xspace}
%newcommand{\University}{\UniversityEspanol}
\newcommand{\University}{Universidad de Ingeniería y Tecnología\xspace}

\newcommand{\FacultadNameEspanol}{Facultad de Computación\\ \xspace}
\newcommand{\FacultadNameEnglish}{}
\newcommand{\FacultadName}{\FacultadNameEspanol}

\newcommand{\DepartmentNameEspanol}{Departamento de Ciencia de la Computación\xspace}
\newcommand{\DepartmentNameEnglish}{Department of Computer Science\xspace}
\newcommand{\DepartmentName}{\DepartmentNameEspanol}

\newcommand{\SchoolShortNameEspanol}{Ciencia de la Computación\xspace}
\newcommand{\SchoolShortNameEnglish}{Computer Science\xspace}
\newcommand{\SchoolShortName}{\SchoolShortNameEspanol}

\newcommand{\SchoolFullNameEspanol}{Escuela Profesional de \SchoolShortNameEspanol}
\newcommand{\SchoolFullNameEnglish}{School of \SchoolShortNameEnglish}
\newcommand{\SchoolFullName}{\SchoolFullNameEspanol}

\newcommand{\SchoolFullNameBreakEspanol}{Escuela Profesional de \\ \SchoolShortNameEspanol\xspace}
\newcommand{\SchoolFullNameBreakEnglish}{School of \SchoolShortNameEnglish\xspace}
\newcommand{\SchoolFullNameBreak}{\SchoolFullNameBreakEspanol}

\newcommand{\PosterTitle}{}

\newcommand{\SchoolAcro}{EPCC\xspace}
\newcommand{\SchoolURL}{\htmladdnormallink{http://cs.utec.edu.pe}{http://cs.utec.edu.pe}\xspace}
\newcommand{\underlogotext}{}

\newcommand{\GradoAcademico}{Bachiller en Ciencia de la Computación\xspace}
\newcommand{\TituloProfesional}{Licenciado en Ciencia de la Computación\xspace}
\newcommand{\GradosyTitulos}%
{\begin{description}%
\item [Grado Académico: ] \GradoAcademico\xspace y% 
\item [Titulo Profesional: ] \TituloProfesional%
\end{description}%
}

\newcommand{\doctitle}{Plan Curricular \YYYY\xspace de la \SchoolFullName\\ \SchoolURL}

\newcommand{\AbstractIntro}{Este documento representa el informe final de la nueva 
malla curricular \YYYY del \SchoolFullName de la \University (\textit{\InstitutionURL}) 
en la ciudad de \city-\country.}

\newcommand{\OtherKeyStones}{}


\newcommand{\profile}{%
El perfil profesional de este programa profesional puede ser mejor entendido a partir de
\OnlyMainDoc{la Fig. \ref{fig.cs} (Pág. \pageref{fig.cs})}\OnlyPoster{las figuras del lado derecho}. 
Este profesional tiene como centro de su estudio a la computación. Es decir, tiene a la computación 
como fin y no como medio. De acuerdo a la definición de esta área, este profesional está llamado 
directamente a ser un impulsor del desarrollo de nuevas técnicas computacionales que 
puedan ser útiles a nivel local, nacional e internacional.

Nuestro perfil profesional está orientado a ser generador de puestos de empleo a través de la innovación permanente. 
Nuestra formación profesional tiene 3 pilares fundamentales: 
Formación Humana, un contenido de acuerdo a normas internacionales y una orientación marcada a la innovación.
}

\newcommand{\mission}{Contribuir al desarrollo científico, tecnológico y técnico del país,  
formando profesionales competentes, orientados a la creación de nueva 
ciencia y tecnología computacional, como motor que impulse y consolide la industria 
del software en base a la investigación científica y tecnológica en 
áreas innovadoras formando, EN NUESTROS profesionales, un conjunto de habilidades y 
destrezas para la solución de problemas computacionales con un compromiso social.\xspace}
