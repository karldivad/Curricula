\newcommand{\DocumentVersion}{2010}
\newcommand{\fecha}{2 de Enero de 2017}
\newcommand{\city}{Lima\xspace}
\newcommand{\country}{Perú\xspace}
\newcommand{\dictionary}{Español\xspace}
\newcommand{\SyllabusLangs}{Español}
\newcommand{\GraphVersion}{2\xspace}

% newcommand{\CurriculaVersion}{1\xspace} % Malla 2006: 1, Malla 2010: 2
% newcommand{\YYYY}{2006\xspace}          % Plan 2006
% newcommand{\Range}{8-10}                % Plan 2010 1-8, Plan 2006 7-10
% newcommand{\SchoolShortName}{Ingeniería Informática\xspace}
% newcommand{\SchoolFullName}{Escuela Profesional de \SchoolShortName}
% newcommand{\SchoolFullNameBreak}{\FacultadName\\ Escuela Profesional de \\\SchoolShortName\xspace}

% Para PUCP 2017
\newcommand{\CurriculaVersion}{2016\xspace} % Malla 2006: 1, Malla 2010: 2
\newcommand{\YYYY}{2018\xspace}             % Plan 2006
\newcommand{\Range}{1-10}                   % Plan 2010 1-8, Plan 2006 7-10
\newcommand{\Semester}{2018-I\xspace}      

%Para PUCP 2018
%newcommand{\CurriculaVersion}{2018\xspace} % Malla 2006: 1, Malla 2010: 2
%newcommand{\YYYY}{2018\xspace}             % Plan 2006
%newcommand{\Range}{1-10}                   % Plan 2010 1-8, Plan 2006 7-10
%newcommand{\Semester}{2018-I\xspace}      

% convert ./fig/UTEC.jpg ./html/img3.png
% cp ./fig/big-graph-curricula.png ./html/img18.png
% convert ../Curricula2.0.out/Peru/CS-UTEC/cycle/2014-1/Plan2010/fig/UTEC.jpg ../Curricula2.0.out/Peru/CS-UTEC/cycle/2014-1/Plan2010/html/img3.png
% cp ../Curricula2.0.out/Peru/CS-UTEC/cycle/2014-1/Plan2010/fig/big-graph-curricula.png ../Curricula2.0.out/Peru/CS-UTEC/cycle/2014-1/Plan2010/html/img18.png

% newcommand{\OutcomesList}{a,b,c,d,e,f,g,h,i,j,k,l,m,HU,FH,TASDSH}
\newcommand{\OutcomesList}{a,b,c,d,e,f,g,h,i,j,k,l,m,n,ñ,o}

\newcommand{\logowidth}{20cm}
\newcommand{\InstitutionURL}{\htmladdnormallink{http://www.utec.edu.pe}{http://www.utec.edu.pe}\xspace}

\newcommand{\UniversityEspanol}{}
\newcommand{\UniversityEnglish}{Pontifical Catholic University of Peru\xspace}
\newcommand{\University}{Pontificia Universidad Católica del Perú\xspace}


\newcommand{\FacultadNameEspanol}{Facultad de Ciencias e Ingeniería\\ \xspace}
\newcommand{\FacultadNameEnglish}{}
\newcommand{\FacultadName}{\FacultadNameEspanol}

\newcommand{\DepartmentNameEspanol}{Departamento de Informática - Sección de Ciencia de la Computación\xspace}
\newcommand{\DepartmentNameEnglish}{Department of Informatics\xspace}
\newcommand{\DepartmentName}{\DepartmentNameEspanol}

\newcommand{\SchoolShortNameEspanol}{Ciencia de la Computación\xspace}
\newcommand{\SchoolShortNameEnglish}{Computer Science\xspace}
\newcommand{\SchoolShortName}{\SchoolShortNameEspanol}

\newcommand{\SchoolFullNameEspanol}{Escuela Profesional de \SchoolShortNameEspanol}
\newcommand{\SchoolFullNameEnglish}{School of \SchoolShortNameEnglish}
\newcommand{\SchoolFullName}{\SchoolFullNameEspanol}

\newcommand{\SchoolFullNameBreakEspanol}{Escuela Profesional de \\ \SchoolShortNameEspanol\xspace}
\newcommand{\SchoolFullNameBreakEnglish}{School of \SchoolShortNameEnglish\xspace}
\newcommand{\SchoolFullNameBreak}{\SchoolFullNameBreakEspanol}

\newcommand{\SchoolAcro}{CC\xspace}
\newcommand{\SchoolURL}{\htmladdnormallink{http://cs.pucp.edu.pe}{http://cs.pucp.edu.pe}\xspace}
\newcommand{\underlogotext}{}

\newcommand{\GradoAcademico}{Bachiller en Ciencia de la Computación\xspace}
\newcommand{\TituloProfesional}{Licenciado en Ciencia de la Computación\xspace}
\newcommand{\GradosyTitulos}%
{\begin{description}%
\item [Grado Académico: ] \GradoAcademico\xspace y% 
\item [Titulo Profesional: ] \TituloProfesional%
\end{description}%
}

\newcommand{\doctitle}{Plan Curricular \YYYY\xspace de la \SchoolFullName\\ \SchoolURL}

\newcommand{\AbstractIntro}{Este documento representa el informe final de la nueva 
malla curricular \YYYY del \SchoolFullName de la \University (\textit{\InstitutionURL}) 
en la ciudad de \city-\country.}

\newcommand{\OtherKeyStones}{}

\newcommand{\profile}{%
Un graduado de Ciencia de la Computación posee los conocimientos necesarios para diseñar sistemas informáticos complejos y 
críticos en términos de eficiencia, fiabilidad y seguridad. 
La corresponsabilidad social que obliga a exigir soluciones cada vez más eficientes, energética o económicamente por ejemplo, 
hace del informático con estas habilidades un profesional altamente valorado en ámbitos muy diversos. 
Por ejemplo, en áreas como la robótica y la optimización de procesos industriales; 
los productos financieros y la predicción en la banca; la planificación de infraestructuras en la administración pública; 
la experimentación científica y el tratamiento de imágenes en centros de investigación biomédica; o la programación de juegos y 
aplicaciones del web a la industria propiamente informática.

Además, la creciente exigencia de innovación frente a los nuevos retos requiere de profesionales entrenados para trabajar 
con rigor científico y que puedan integrarse en equipos multidisciplinarios de científicos e ingenieros. 
La valía del especialista en Ciencia de la Computación radica en su habilidad para innovar, y para detectar y garantizar 
los requerimientos críticos de un sistema informático complejo. 
Esta tendencia en la nueva industria informática viene liderada por las firmas de más prestigio de ámbito global.

La formación humanística, carácterística de la PUCP, se combina con una alta formación especializada para que el egresado de 
Ciencia de la Computación proponga soluciones a los diversos problemas de nuestro país teniendo como base el 
pensamiento computacional y apoyándose en las diversas tecnologías informáticas.
}

\newcommand{\mission}{Contribuir al desarrollo nacional como profesional que basa sus propuestas en el pensamiento computacional y 
aplica conocimiento de ciencias e ingenierías para tomar ventaja competitiva y otorgamiento de soluciones más adecuadas, 
todo ello en beneficio y bienestar de nuestra sociedad que cada vez se hace más dependiente de la tecnología.\xspace}
