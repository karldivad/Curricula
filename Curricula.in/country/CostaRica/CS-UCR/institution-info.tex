\newcommand{\DocumentVersion}{V2.0}
\newcommand{\fecha}{1 de Septiembre de 2010}
\newcommand{\YYYY}{2010\xspace}
\newcommand{\Semester}{2010-2\xspace}
\newcommand{\city}{San Jos�\xspace}
\newcommand{\country}{Costa Rica\xspace}
\newcommand{\dictionary}{Espa�ol\xspace}
\newcommand{\GraphVersion}{2\xspace}
\newcommand{\CurriculaVersion}{2\xspace}
\newcommand{\OutcomesList}{a,b,c,d,e,f,g,h,i,j,k,l,m,HU,FH,TASDSH}

\newcommand{\University}{Universidad de Costa Rica\xspace}
\newcommand{\InstitutionURL}{http://www.ucr.ac.cr\xspace}
\newcommand{\underlogotext}{}
\newcommand{\FacultadName}{Computaci�n\xspace}
\newcommand{\DepartmentName}{Ciencia de la Computaci�n\xspace}
\newcommand{\SchoolFullName}{Carrera de Ciencia de la Computaci�n}
\newcommand{\SchoolFullNameBreak}{Facultad de Computaci�n\\ Carrera de Ciencia de la Computaci�n\xspace}
\newcommand{\SchoolShortName}{Ciencia de la Computaci�n\xspace}
\newcommand{\SchoolAcro}{PPII\xspace}
\newcommand{\SchoolURL}{http://cs.ucr.ac.cr}

\newcommand{\GradoAcademico}{Licenciatura Universitaria\xspace}
\newcommand{\TituloProfesional}{Licenciatura en en Ciencia de la Computaci�n\xspace}
\newcommand{\GradosyTitulos}%
{\begin{description}%
\item [Grado: ] \GradoAcademico y%
\item [Titulo: ] \TituloProfesional%
\item [Duraci�n: ] Toda la carrera tiene una duraci�n de diez semestres de 15 semanas cada uno. La carrera est� constituida por un total de \ref{n} cursos
\end{description}%
}

\newcommand{\doctitle}{Plan Curricular \YYYY\xspace del \SchoolFullName\\ \SchoolURL}

\newcommand{\AbstractIntro}{Este documento representa el informe final de la nueva malla curricular \YYYY 
del \SchoolFullName de la \University (\textit{\InstitutionURL}) en la ciudad de \city-\country.}

\newcommand{\OtherKeyStones}{}

\newcommand{\profile}{%
El perfil profesional de este programa profesional puede ser mejor entendido a partir de
\OnlyMainDoc{la Fig. \ref{fig.cs} (P�g. \pageref{fig.cs})}\OnlyPoster{las figuras del lado derecho}. 
Este profesional tiene como centro de su estudio a la computaci�n. Es decir, tiene a la computaci�n 
como fin y no como medio. De acuerdo a la definici�n de esta �rea, este profesional est� llamado 
directamente a ser un impulsor del desarrollo de nuevas t�cnicas computacionales que 
puedan ser �tiles a nivel local, nacional e internacional.

Nuestro perfil profesional est� orientado a ser generador de puestos de empleo a trav�s de la innovaci�n permanente. 
Nuestra formaci�n profesional tiene 3 pilares fundamentales: 
Formaci�n Humana, un contenido de acuerdo a normas internacionales y una orientaci�n marcada a la innovaci�n.
}

\newcommand{\HTMLFootnote}{Generado por <A HREF='http://socios.spc.org.pe/ecuadros/'>Ernesto Cuadros-Vargas</A> <ecuadros AT spc.org.pe> (<A HREF='http://www.ucsp.edu.pe/'>Universidad Cat�lica San Pablo, Per�</A>), <A HREF='http://www.ucr.edu.cr/'>Universidad de Costa Rica, San Jos�-Costa Rica</A><BR>basado en el modelo de la <A HREF='http://www.spc.org.pe/'>Sociedad Peruana de Computaci�n</A> y en la Computing Curricula de <A HREF='http://www.computer.org/'>IEEE-CS</A>/<A HREF='http://www.acm.org/'>ACM</A>}


