\newcommand{\DocumentVersion}{2011}
\newcommand{\fecha}{\today}
\newcommand{\city}{Santiago de Chile\xspace}
\newcommand{\country}{Chile\xspace}
\newcommand{\dictionary}{Español\xspace}
\newcommand{\GraphVersion}{2\xspace}

\newcommand{\CurriculaVersion}{2\xspace} % Malla 2006: 1
\newcommand{\YYYY}{2012\xspace}          % Plan 2006
\newcommand{\Range}{1-10}                 %Plan 2010 1-6, Plan 2006 5-10
\newcommand{\Semester}{2012-2\xspace}

% convert ./fig/UCSP.jpg ./html/img3.png
% cp ./fig/big-graph-curricula.png ./html/img18.png
% cp ./fig/small-graph-curricula.png ./html/img19.png 

\newcommand{\OutcomesList}{a,b,c,d,e,f,g,h,i,j,k,l,m,HU,FH,TASDSH}
\newcommand{\logowidth}{7cm}

\newcommand{\University}{Universidad de Chile\xspace}
\newcommand{\InstitutionURL}{http://www.uchile.cl\xspace}
\newcommand{\underlogotext}{}
\newcommand{\FacultadName}{Facultad de Ciencias Físicas y Matemáticas\xspace}
\newcommand{\DepartmentName}{Ciencias de la Computación\xspace}
\newcommand{\SchoolFullName}{Escuela de Ingeniería Civil en Computación}
\newcommand{\SchoolFullNameBreak}{\FacultadName\\ Carrera de Ingeniería \\Civil en Computación\xspace}
\newcommand{\SchoolShortName}{Civil en Computación\xspace}
\newcommand{\SchoolAcro}{ICC\xspace}
\newcommand{\SchoolURL}{http://www.dcc.uchile.cl}

\newcommand{\AcademicDegreeIssued}{Licenciado en Ciencias de la Ingeniería (Mención Computación)\xspace}
\newcommand{\TitleIssued}{Ingeniero Civil en Computación\xspace}
\newcommand{\AcademicDegreeAndTitle}%
{\begin{description}%
\item [Grado Académico: ] \AcademicDegreeIssued\xspace y%
\item [Titulo Profesional: ] \TitleIssued%
\end{description}%
}

\newcommand{\doctitle}{Plan de Estudios \YYYY\xspace del \SchoolFullName\\ \SchoolURL}

\newcommand{\AbstractIntro}{Este documento presenta el informe preliminar de la nueva malla 
curricular \YYYY del \SchoolFullName de la \University (\textit{\InstitutionURL}) en la ciudad 
de \city-\country.}

\newcommand{\OtherKeyStones}{
Un pilar que merece especial consideración en el caso de la \University es el aspecto de formación en desarrollo 
del pensamiento, comunicació›n efectiva, desarrollo personal y formación ciudadana.\xspace}

\newcommand{\profile}{
El perfil de los egresados de la carrera de Ingeniería Civil en Computación (ICC) sigue los estándares de la iniciativa CDIO, 
que define un marco acerca de las habilidades fundamentales para los ingenieros de la próxima generación. 
Está orientado a un fuerte dominio de las matemáticas, ciencias básicas, ciencias de la ingeniería y comprensión global 
de los fundamentos de la Ciencia de la Computación, incluyendo conocimiento avanzado en una o más áreas, y capacidad 
para aplicar estos conocimientos de una manera rigurosa e integrada. 

Los egresados de esta carrera poseen un dominio de técnicas y herramientas modernas necesarias para el ejercicio 
de su profesión, y la habilidad para concebir, diseñar, implementar, operar, evaluar y controlar software, sistemas, 
componentes o procesos, en forma eficiente y creativa, que cumplan con las especificaciones demandadas por el contexto; 
considerando las restricciones económicas, ambientales, sociales, políticas, éticas, de salud y seguridad, 
de manufacturación y sustentabilidad.
}

\newcommand{\HTMLFootnote}{{Generado por <A HREF='http://socios.spc.org.pe/ecuadros/'>Ernesto Cuadros-Vargas</A> <ecuadros AT spc.org.pe>, <A HREF='http://www.spc.org.pe/'>Sociedad Peruana de Computación-Perú</A>, <A HREF='http://www.ucsp.edu.pe/'>Universidad Católica San Pablo, Arequipa-Perú</A><BR>basado en el modelo de la {\it Computing Curricula} de <A HREF='http://www.computer.org/'>IEEE-CS</A>/<A HREF='http://www.acm.org/'>ACM</A>}}
\newcommand{\Copyrights}{Generado por Ernesto Cuadros-Vargas (ecuadros AT spc.org.pe), Sociedad Peruana de Computación (http://www.spc.org.pe/), Universidad Católica San Pablo (http://www.ucsp.edu.pe) Perú basado en la {\it Computing Curricula} de IEEE-CS (http://www.computer.org) y ACM (http://www.acm.org/)}
