\newcommand{\DocumentVersion}{2011}
\newcommand{\fecha}{26 de Diciembre de 2011}
\newcommand{\YYYY}{2012\xspace}
\newcommand{\Semester}{2012-2\xspace}
\newcommand{\city}{Curicó\xspace}
\newcommand{\country}{Chile\xspace}
\newcommand{\dictionary}{Español\xspace}
\newcommand{\GraphVersion}{2\xspace}
\newcommand{\CurriculaVersion}{2\xspace}
\newcommand{\OutcomesList}{a,b,c,d,e,f,g,h,i,j,k,l,m,HU,FH,TASDSH}
\newcommand{\logowidth}{20cm}

\newcommand{\University}{Universidad de Talca\xspace}
\newcommand{\InstitutionURL}{http://www.utalca.cl\xspace}
\newcommand{\underlogotext}{}
\newcommand{\FacultadName}{Facultad de Ingenierí­a\xspace}
\newcommand{\DepartmentName}{Ciencias de la Computación\xspace}
\newcommand{\SchoolFullName}{Escuela de Ingeniería Civil en Computación}
\newcommand{\SchoolFullNameBreak}{\FacultadName\\ Carrera de Ingeniería \\Civil en Computación \\ (Informática)\xspace}
\newcommand{\SchoolShortName}{Civil en Computación\xspace}
\newcommand{\SchoolAcro}{ICC\xspace}
\newcommand{\SchoolURL}{http://dcc.utalca.cl}

\newcommand{\GradoAcademico}{Licenciado en Ciencias de la Ingeniería\xspace}
\newcommand{\TituloProfesional}{Ingeniero Civil en Computación\xspace}
\newcommand{\GradosyTitulos}%
{\begin{description}%
\item [Grado Académico: ] \GradoAcademico\xspace y%
\item [Titulo Profesional: ] \TituloProfesional%
\end{description}%
}

\newcommand{\doctitle}{Plan de Estudios \YYYY\xspace del \SchoolFullName\\ \SchoolURL}

\newcommand{\AbstractIntro}{Este documento presenta el informe preliminar de la nueva malla 
curricular \YYYY del \SchoolFullName de la \University (\textit{\InstitutionURL}) en la ciudad 
de \city-\country.}

\newcommand{\OtherKeyStones}{
Un pilar que merece especial consideración en el caso de la \University es el aspecto de formación en desarrollo 
del pensamiento, comunicació›n efectiva, desarrollo personal y formación ciudadana.\xspace}

\newcommand{\profile}{
Formar Ingenieros Civiles en Computación con una formación fundamental que complemente y sustente su preparación en 
Ciencias Básicas y de la Ingeniería, así como en distintas especialidades de la Computación.

El profesional estará habilitado para desempeñarse en proyectos de desarrollo de software, y en la aplicación de técnicas y estrategias de Ciencias de la Computación para la resolución de problemas complejos de Ingeniería.

Tendrá la capacidad de adaptarse a los cambios tecnológicos, teóricos y metodológicos que pudieran ocurrir, pues su formación no se restringe a herramientas computacionales específicas.

% Al término de su formación, los Ingenieros Civiles en Computación habrán desarrollado las siguientes competencias.
% 
% \paragraph{AREA:} FORMACIÓN FUNDAMENTAL
% 
% Competencias Instrumentales:
% \begin{itemize}
% \item Comunicar de modo pertinente en forma oral y escrita en situaciones diversas y propias de su
% formación profesional
% \item Aplicar herramientas de aprendizaje autónomo como estrategia para continuar aprendiendo
% \end{itemize}
% 
% Competencias Interpersonales:
% \begin{itemize}
% \item Lograr eficacia en el uso de habilidades sociales para establecer relaciones interpersonales
% adecuadas.
% \item Desempeñarse colaborativamente en equipos de trabajo mostrando liderazgo y emprendimiento en los ámbitos económico y social.
% \end{itemize}
% 
% Competencias Ciudadanas:
% \begin{itemize}
% \item Discernir en los ámbitos ético profesional, social, cultural, ambiental y ciudadano
% \end{itemize}
% 
% 
% \paragraph{ÁREA:} FORMACIÓN BÁSICA Y DISCIPLINARIA:
% 
% Dominio: Ingeniería de Software
% \begin{itemize}
% \item Aplicar Ingeniería de Requisitos
% \item Diseñar y Construir Sistemas de Software
% \item Modelar y Manipular Información
% \item Administrar Proyectos Informáticos
% \item Gestionar la Seguridad de Proyectos Informático
% \end{itemize}
% 
% Dominio: Ciencias de la Computación e Ingeniería
% \begin{itemize}
% \item Resolver problemas computacionales algorítmicamente
% \item Diseñar y evaluar el desempeño de algoritmos y estructuras de datos
% \item Construir un modelo de computación para una problemática particular
% \item Diseñar, configurar y administrar sistemas computacionales y redes de computadores
% \item Integrar conocimientos para la solución de problemas complejos de Ingeniería
% \end{itemize}
}

\newcommand{\HTMLFootnote}{Generado por <A HREF='http://socios.spc.org.pe/ecuadros/'>Ernesto Cuadros-Vargas</A> <ecuadros AT spc.org.pe>, <A HREF='http://www.spc.org.pe/'>Sociedad Peruana de Computación-Perú</A>, <A HREF='http://www.ucsp.edu.pe/'>Universidad Católica San Pablo, Arequipa-Perú</A><BR>basado en el modelo de la {\it Computing Curricula} de <A HREF='http://www.computer.org/'>IEEE-CS</A>/<A HREF='http://www.acm.org/'>ACM</A>}
\newcommand{\Copyrights}{Generado por Ernesto Cuadros-Vargas (ecuadros AT spc.org.pe), Sociedad Peruana de Computación (http://www.spc.org.pe/), Universidad Católica San Pablo (http://www.ucsp.edu.pe) Perú basado en la {\it Computing Curricula} de IEEE-CS (http://www.computer.org) y ACM (http://www.acm.org/)}
