\begin{syllabus}

\course{EG0006. Matemática III}{Obligatorio}{EG0006} % Common.pm

\begin{justification}
This course introduces the first concepts of linear algebra as well as numerical methods with an emphasis on problem solving with the Scilab open source libe package.
Mathematical theory is limited to fundamentals, while effective application for problem solving is privileged.
In each subject, a few methods of relevance for engineering are taught. Knowledge of these methods prepares students for the search for more advanced alternatives, if required.

\end{justification}

\begin{goals}
\item Ability to apply knowledge about Mathematics.
\item Ability to apply engineering knowledge.
\item Ability to apply the modern knowledge, techniques, skills and tools of modern engineering to the practice of engineering

\end{goals}

\begin{outcomes}{V1}
    \item \ShowOutcome{a}{3}
    \item \ShowOutcome{j}{3}
\end{outcomes}

\begin{competences}{V1}
    \item \ShowCompetence{C1}{a} 
    \item \ShowCompetence{C20}{j} 
    \item \ShowCompetence{C24}{j} 
\end{competences}

\begin{unit}{Introduction}{}{Anton,Chapra}{18}{C1}

  \begin{topics}
      \item Importance of linear algebra and numerical methods. Examples.
   \end{topics}

   \begin{learningoutcomes}
      \item Be able to understand the basic concepts and importance of Linear Algebra and Numerical Methods.
   \end{learningoutcomes}
\end{unit}

\begin{unit}{Linear Algebra}{}{Anton,Chapra}{14}{C1}
   \begin{topics}
    \item Elementary matrix algebra and determinants
    \item Null space and exact solutions of systems of linear equations Ax=b:
	  \begin{subtopics}
	    \item Tridiagonal and triangular systems and Gaussian elimination with and without pivoting.
	    \item LU factorization and Crout algorithm.
	  \end{subtopics}
    \item Basics on eigenvalues and eigenvectors:	  
	  \begin{subtopics}
	    \item Characteristic polynomials.
	    \item Algebraic and geometric multiplicities.
	  \end{subtopics}
    \item Least squares estimation.
    \item Linear transformations.
    \end{topics}

   \begin{learningoutcomes}
      \item Understanding the basics concepts of Linear Algebra.
      \item Solve properly linear transformations problems.
      \end{learningoutcomes}
\end{unit}

\begin{unit}{Numerical methods}{}{Anton,Chapra}{22}{C24}
   \begin{topics}
    \item Basics on solutions of systems of linear equations Ax=b: Jacobi and Gauss Seidel methods.
    \item Application of matrix factorizations to the solution of linear systems (singular value decomposition, QR, Cholesky) Numerical computation of null space, rank and condition number.
    \item Root finding:
	  \begin{subtopics}
	    \item Bisection.
	    \item Fixed-point iteration.
	    \item Newton-Raphson methods.
	  \end{subtopics}
    \item Basics on interpolation:
	  \begin{subtopics}
	    \item Newton and Lagrange polynomial interpolations
	    \item Spline interpolation
	  \end{subtopics}
    \item Basics on numerical differentiation and Taylor approximation
    \item Basics on numerical integration:
	  \begin{subtopics}
	    \item Trapezium, midpoint and Simpson rule
	    \item Gaussian quadrature
	  \end{subtopics}
    \item Basics on numerical solutions to ODEs:
	  \begin{subtopics}
	    \item Finite differences; Euler and Runge-Kutta methods
	    \item Converting higher order ODEs into a system of low order ODEs
	    \item Runge-Kutta methods for systems of equations
	    \item Single shooting method
	  \end{subtopics}
    \item Short introduction to optimization techniques: overview on linear programming, bounded linear systems, quadratic programming, gradient descent.
    \end{topics}

   \begin{learningoutcomes}
      \item Understanding the basics concepts of Numerical Methods.
      \item Applying the most frequent methods for the resolution of mathematical problems.
      \item Implementing and applying numerical algorithms for the solution of mathematical problems using the Scilab open-source computational package.
      \item Applying Scilab for the solution of mathematical problems and for plotting graphs.
   \end{learningoutcomes}
\end{unit}

\begin{coursebibliography}
\bibfile{BasicSciences/MA102}
\end{coursebibliography}

\end{syllabus}

