\begin{syllabus}

\course{CS240S. Compiladores}{Obligatorio}{CS240S}

\begin{justification}
Que el alumno conozca y comprenda los conceptos y principios
fundamentales de la teoría de compilación para realizar la
construcción de un compilador
\end{justification}

\begin{goals}
\item Conocer las técnicas básicas empleadas durante el proceso de generación intermedio, optimización y generación de código.
\item Aprender a implementar pequeños compiladores.
\end{goals}

\begin{outcomes}
\ExpandOutcome{a}{3}
\ExpandOutcome{b}{4}
\ExpandOutcome{j}{4}
\end{outcomes}

\begin{unit}{\PLOverviewDef}{Lou004LP,Pratt98}{8}{4}
   \PLOverviewAllTopics
   \PLOverviewAllObjectives
\end{unit}

\begin{unit}{\PLBasicLanguageTranslationDef}{Aho2008,Aho90,Teu98,Lou004CO,Appe002}{12}{3}
   \PLBasicLanguageTranslationAllTopics
   \PLBasicLanguageTranslationAllObjectives
\end{unit}

\begin{unit}{\PLLanguageTranslatioSystemsDef}{Aho2008,Aho90,Lou004CO,Teu98,Lem96,Appe002}{24}{2}
   \PLLanguageTranslatioSystemsAllTopics
   \PLLanguageTranslatioSystemsAllObjectives
\end{unit}

\begin{unit}{Paralelismo a nivel de instrucción}{Aho2008}{4}{2}
  \begin{topics}
     \item Arquitectura de procesadores.
     \item Restricciones de programación de código.
     \item Programación de bloques básicos.
     \item Programación de código global.
     \item Canalización por software.
  \end{topics}

  \begin{unitgoals}
     \item Describir la importancia y poder de la extracción de paralelismo de las secuencias de instrucciones.
     \item Explicar los conceptos de bloques básicos y código global.
     \item Distinguir los conceptos entre canalización de instrucciones por software.
  \end{unitgoals}
\end{unit}

\begin{unit}{Optimización para el paralelismo y la localidad}{Aho2008}{4}{2}
  \begin{topics}
     \item Conceptos básicos.
     \item Multiplicación de matrices.
     \item Espacios de iteraciones.
     \item Indices de arreglos afines.
     \item Análisis de dependencias de datos de arreglos.
     \item Búsqueda del paralelismo sin sincronización.
     \item Sincronización entre ciclos paralelos.
  \end{topics}

  \begin{unitgoals}
     \item Diseñar, codificar programas para cálculos paralelos.
     \item Identificar las propiedades básicas del paralelismo.
     \item Aplicar los fundamentos del paralelismo en la programación.
  \end{unitgoals}
\end{unit}



\begin{coursebibliography}
\bibfile{Computing/CS/CS240S}
\end{coursebibliography}

\end{syllabus}
