\begin{syllabus}

\course{FG101. Communication}{Obligatorio}{FG101}
% Source file: ../Curricula.in/lang/English/cycle/2021-I/Syllabi/GeneralEducation/FG101.tex

\begin{justification}
   To achieve an effective communication in the personal and professional field, 
   The adequate handling of the language in oral and written form is a priority. 
   It is therefore justified that the students know, understand and apply 
   the conceptual and operational aspects of their language, for the development 
   of their fundamental communication skills: listening, speaking, reading and writing.

   Consequently, the permanent exercise and the contribution of the contribute greatly 
   to academic training and, in the future in the course of their work
\end{justification}

\begin{goals}
\item Develop communication skills through the theory and practice of language that help students to overcome the academic demands of the undergraduate program and contribute to their humanistic formation and as human beings.
\end{goals}

--COMMON-CONTENT--

\begin{unit}{}{First Unit}{Real}{16}{C17,C20}
\begin{topics}
      \item The communication, definition, relevance. Elements. Process. Functions. Classification. Oral and written communication.
      \item The language: definition. Features and functions. Language: levels. System. Rule. Speaks. The linguistic sign: definition, characteristics.
      \item Multilingualism in Peru. Dialect variations in Peru.
      \item The word: definition, classes and structure. The monemas: lexema and morpheme. The morpheme: classes. Etymology.
      \item The Academic Article: Definition, structure, choice of topic, delimitation of the topic.
\end{topics}

\begin{learningoutcomes}
   \item Recognize and value communication as a process of understanding and exchanging messages, differentiating its elements, functions and classification[\Usage].
   \item Analyze the characteristics, functions and elements of language and language [\Usage].
   \item Identify the characteristics of multilingualism in Peru, valuing its idiomatic richness [\Usage].
   \item Identify the qualities of the word and its classes [\Usage].
\end{learningoutcomes}
\end{unit}

\begin{unit}{}{Second Unit}{Real, Gatti}{16}{C17, C24}
\begin{topics}
   \item Paragraph: Main, secondary and global idea.
   \item The text: definition, characteristics. Cohesion and coherence.
   \item Organization of the text: The reference (dejis); Anaphora, cataphora, ellipsis. Logical and textual connectors.
   \item Types of text: descriptive (processes), expository, argumentative.
   \item Functions of elocution in the text: generalization, identification, nominalization, classification, exemplification, definition.
   \item Discontinuous texts: graphs, tables and diagrams.
   \item Search for information. Information sources. References and citations. Record of information: index cards, notes, summaries, etc. Critical apparatus: concept and purpose. APA Standards or other.
\end{topics}
\begin{learningoutcomes}
   \item Writing expository texts highlighting the main and secondary idea. [\Usage].
   \item Write expository texts with adequate cohesion and coherence, making use of textual references and connectors. [\Usage].
   \item Interpreting discontinuous texts, assessing their importance for the understanding of the message. [\Usage].
\end{learningoutcomes}
\end{unit}

\begin{unit}{}{Third Unit}{Lobato}{12}{C17}
\begin{topics}
   \item Prayer: definition and classes. The enunciative, interrogative, imperative, exclamatory and optional sentence. The proposition and the sentence. The simple and compound sentence. Coordination and subordination. The syntagm: structure and classes: nominal, verbal, adjectival, prepositional, adverbial.
   \item Preparation of a glossary of technical terms, abbreviations and acronyms related to the specialty (permanent activity throughout the semester).
   \item Writing the academic article: Summary, key words, introduction, development, conclusions, bibliographyTechnology (APA standards or other required by the Professional School).
\end{topics}
\begin{learningoutcomes}
   \item Recognizing and analyzing sentence structure, assessing its importance and usefulness in writing texts.[\Usage].
   \item Register and use specialty-specific terminology. [\Usage].
\end{learningoutcomes}
\end{unit}
y
\begin{unit}{}{Fourth Unit}{Vivaldi}{12}{C17, C20, C24}
\begin{topics}
   \item Writing correspondence: letter - application, report, memorandum, resume.
   \item Oral speech: purposes, parts. Listening: purposes and conditions. Vices of diction: barbarism, solecism, cacophony, redundancy, amphibology, monotony. Prepositional regime.
   \item Group communication Process, dynamics, structure Forms (Techniques): Round table, panel, forum and debate
   \item Final review of the academic article. Presentation and oral presentation of intellectual production works.
\end{topics}
\begin{learningoutcomes}
   \item To write academic and functional texts taking into account the different moments of their production, their structure, purpose and formality. [\Usage].
   \item Demonstrate skills as a sender or receiver in different communication situations with language correction. [\Usage].
   \item Apply the different forms (techniques) of group communication recognizing their importance for problem solving, decision making or discussion. [\Usage].
\end{learningoutcomes}
\end{unit}

\begin{coursebibliography}
\bibfile{GeneralEducation/FG101}
\end{coursebibliography}

\end{syllabus}
