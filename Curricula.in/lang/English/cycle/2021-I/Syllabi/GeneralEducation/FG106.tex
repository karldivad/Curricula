\begin{syllabus}

\course{FG106. Theater}{Obligatorio}{FG106}
% Source file: ../Curricula.in/lang/English/cycle/2021-I/Syllabi/GeneralEducation/FG106.tex

\begin{justification}
	It helps students to identify themselves with the 'Academic Community' of the University, insofar as it provides them with natural channels of integration into their group and their Study Centre and allows them, from an alternative viewpoint, to visualise the inner worth of the people around them, while at the same time getting to know their own.
	It relates the university student, through experimentation, with a new language, a means of communication and expression that goes beyond the conceptualized verbal expression.
	It helps the student in his integral formation, developing in him corporal capacities. It stimulates positive attitudes, cognitive and affective skills. It enriches their sensitivity and awakens their solidarity.
	It disinhibits and socializes, relaxes and makes people happy, opening a path of knowledge of one's own being and the being of others.
\end{justification}

\begin{goals}
\item To contribute to the personal and professional formation of the student, recognizing, valuing and developing his body language, integrating him to his group, strengthening his personal security, enriching his intuition, his imagination and creativity, motivating him to open paths of search of knowledge of himself and communication with others through his sensibility, exercises of introspection and new ways of expression.
\end{goals}

--COMMON-CONTENT--

\begin{unit}{}{Art, Creativity and Theatre}{Majorana,PAVIS}{6}{C18,C24}
\begin{topics}
	\item What is Art? An experiential and personal one.
	\item The master key: creativity.
	\item The importance of the theatre in personal and professional training.
	\item Usefulness and focus of the theatrical art.
\end{topics}
\begin{learningoutcomes}
	\item Recognize the validity of Art and creativity in personal and social development [\Usage].
	\item To relate the student to his group, valuing the importance of human communication and the social collective [\Usage].
	\item Recognize basic notions of theater [\Usage].
\end{learningoutcomes}
\end{unit}

\begin{unit}{}{The Game: the actor's job}{Majorana,PAVIS}{6}{C17,C24}
\begin{topics}
	\item I play, then I exist.
	\item Child's play and dramatic play.
	\item Group integration games and creativity games.
	\item The theatrical sequence.a
\end{topics}
\begin{learningoutcomes}
	\item Recognize play as a fundamental tool of the theater. [\Usage].
	\item Internalizing and revaluing play as creative learning. [\Usage].
	\item To bring the student closer to the theatrical experience in a spontaneous and natural way. [\Usage].
\end{learningoutcomes}
\end{unit}

\begin{unit}{}{Body expression and dramatic use of the Object}{Majorana,PAVIS}{9}{C17, C18, C24}
\begin{topics}
	\item Awareness of the body.
	\item Awareness of space
	\item Time awareness
	\item Creation of individual and collective sequences: Body, space and time
	\item The dramatic use of the element: The theatrical game.
	\item Theatrical presentations with the use of the element.

\end{topics}
\begin{learningoutcomes}
	\item Experimenting with new forms of expression and communication. [\Usage].
	\item Know some mechanisms of control and body management. [\Usage].
	\item To provide paths for the student to creatively develop his imagination, his ability to relate to and capture auditory, rhythmic and visual stimuli. [\Usage].
	\item To know and develop the management of their own space and spatial relations . [\Usage].
	\item Experiencing different emotional states and new collective climates. [\Usage].
\end{learningoutcomes}
\end{unit}

\begin{unit}{}{Non-verbal communication in the theatre}{Majorana,PAVIS}{12}{C18, C24}
\begin{topics}
	\item Relaxation, concentration and breathing.
	\item Disinhibition and interaction with the group.
	\item Improvisation.
	\item Balance, weight, time and rhythm.
	\item Analysis of the movement. Types of movement.
	\item The theatrical presence.
	\item The dance, the theatrical choreography.

\end{topics}
\begin{learningoutcomes}
	\item Exercise in the management of non-verbal communication skills. [\Usage].
	\item Practice games and body language exercises, individually and in groups. [\Usage].
	\item To freely and creatively express their emotions and feelings and their vision of society through original representations in various languages. [\Usage].
	\item Knowing the types of action. [\Usage].
\end{learningoutcomes}
\end{unit}

\begin{unit}{}{Traces of the theatre in time (The theatre in history)}{Majorana,PAVIS}{3}{C24}
\begin{topics}
	\item The origin of the theatre, the Greek theatre and the Roman theatre.
	\item The medieval theatre, the comedy of art.
	\item From passion to reason: Romanticism and Enlightenment.
	\item The realistic theatre, epic theatre. Brech and Stanislavski.
	\item The theatre of the absurd, contemporary theatre and total theatre.
	\item Theater in Peru: Yuyashkani, La Tarumba, pataclaun, others.
\end{topics}
\begin{learningoutcomes}
	\item To know the influence that society has exerted on the theatre and the response of this art to different moments in history. [\Usage].
	\item To appreciate the value and contribution of the works of important playwrights. [\Usage].
	\item Analyzing the social context of theatrical art. [\Usage].
	\item Reflecting on Peruvian and Arequipa's Theatre. [\Usage].
\end{learningoutcomes}
\end{unit}

\begin{unit}{}{The Theatrical Assembly}{Majorana,PAVIS}{12}{C17,C18, C24}
\begin{topics}
	\item Theatrical appreciation. Expectation of one or more plays.
	\item Theatrical space.
	\item Construction of the character
	\item Creation and staging of a play.
	\item Public presentation of small plays using costumes, make-up, scenery, props and the dramatic use of the object.
\end{topics}
\begin{learningoutcomes}
	\item To use theatrical creation as a manifestation of one's own ideas and feelings before society. [\Usage].
	\item To apply the techniques practiced and the knowledge learned in a concrete theatrical appreciation and/or expression that links the role of education. [\Usage].
	\item Exchange experiences and make short presentations of theatrical exercises in groups, in front of an audience. [\Usage].
\end{learningoutcomes}
\end{unit}

\begin{coursebibliography}
\bibfile{GeneralEducation/FG101}
\end{coursebibliography}

\end{syllabus}
