\begin{syllabus}

\course{FG102. Study Methodology}{Obligatorio}{FG102}
% Source file: ../Curricula.in/lang/English/cycle/2021-I/Syllabi/GeneralEducation/FG102.tex

\begin{justification}
Students in vocational training need to improve their attitude towards academic work and demands. In addition, they should understand the mental process that occurs in the exercise of study to achieve learning, so they know where and how to make the most appropriate adjustments to their needs. They also need to master various forms of study, so that they can select the strategies best suited to their personal learning style and the nature of each subject. They also need to know and use ways to search for academic information and do creative work of a formal academic nature, so that they can apply them to their college work, making their effort successful.
\end{justification}

\begin{goals}
\item Develop in the student attitudes and skills that promote autonomy in learning, good academic performance and their training as a person and professional.
\end{goals}

\begin{outcomes}{V1}
    \item \ShowOutcome{d}{2}
    \item \ShowOutcome{h}{2}
    \item \ShowOutcome{l}{1}
\end{outcomes}

\begin{competences}{V1}
    \item \ShowCompetence{C19}{h}
    \item \ShowCompetence{C24}{h,d}
\end{competences}

\begin{unit}{}{First Unit: The university, intellectual work and organization}{bibliografiaTecnologia}{12}{C19, C24}
\begin{topics}
        \item The underlining.
        \item Stitch taking.
        \item Vocation, habits of university life.
        \item Human interaction.
        \item The will as a requirement for learning.
        \item Planning and time.
\end{topics}
\begin{learningoutcomes}
        \item To analyze the normative documentation of the University evaluating its importance for the coexistence and academic performance. [\Usage]
        \item Understand and value the demands of university life as part of personal and professional training.[\Usage]
        \item Properly plan your time based on your personal and academic goals.[\Usage]
        \item Develop a personal improvement plan based on self-knowledge.[\Usage]
\end{learningoutcomes}
\end{unit}

\begin{unit}{}{Second Unit}{Rodriguez, Pereza,Quintana}{12}{C19,C24}
\begin{topics}
   \item Summary. Notes in the margin. Mnemonics.
   \item Mental processes: Simple, complex. Fundamentals of meaningful learning.
   \item The steps or factors for learning. Laws of learning. Learning style questionnaire Identification of personal learning style.
   \item Academic reading. Levels of analysis of a text: central idea, main idea and secondary ideas. Meza de Vernet's model.
   \item Exams: Preparation. Guidelines and strategies before, during and after an exam. Emotional intelligence and exams.
   \item The sources of information. Critical device: concept and purpose. Vancouver standards. References and quotations.
\end{topics}
\begin{learningoutcomes}
        \item Identify mental processes by relating them to learning. [\Usage].
        \item Understand the learning process to determine your own style and incorporate it into your academic activity. [\Usage].
        \item Develop strategies for text analysis by enhancing reading comprehension. [\Usage].
        \item To design a strategic program to successfully face the exams.[\Usage].
\end{learningoutcomes}
\end{unit}

\begin{unit}{}{Third Unit}{Chaveza, Flores}{12}{C24}
\begin{topics}
        \item The concept maps. Characteristics and elements.
        \item Copyrights and plagiarism. Personal or moral rights. Economic rights. ``Copyrigth''.
        \item Self-esteem, Emotional Intelligence, Assertiveness and Resilience. Concepts, development and strengthening.
        \item Critical Apparatus: Vancouver Standards. Practical application.
        \item Generation of ideas. Strategies for organizing ideas, writing and reviewing.
\end{topics}
\begin{learningoutcomes}
        \item To apply the techniques of study taking into account their particularities and adapting them to the different situations demanded by the learning. [\Usage].
        \item Recognize the importance of respect for intellectual property. [\Usage].
        \item Recognize the importance of EQ, assertive behavior, self-esteem and resilience by valuing them as strengths for college performance. [\Usage].
\end{learningoutcomes}
\end{unit}

\begin{unit}{}{Fourth Unit}{Rodriguez, Chaveza}{12}{C19}
\begin{topics}
        \item Synoptic Table. The mind maps. Practice with the subject matter of the course.
        \item The personal method of study.
        \item The cooperative learning: definition, study groups, organization, members' roles.
        \item Guidelines to conform efficient and harmonic groups.
        \item The personal study method. Reinforcement of study techniques.
        \item Presentation and exposition of works of intellectual production.
        \item The debate and the argumentation.
\end{topics}
\begin{learningoutcomes}
        \item To apply the techniques of study taking into account their particularities and adapting them to the different situations demanded by the learning. [\Usage].
        \item Assume management of behaviors and attitudes for cooperative learning and performance in work teams. [\Usage].
        \item Formulate a personal study method project, according to your style and needs, including techniques and strategies. [\Usage].
\end{learningoutcomes}
\end{unit}

\begin{coursebibliography}
\bibfile{GeneralEducation/FG101}
\end{coursebibliography}

\end{syllabus}
