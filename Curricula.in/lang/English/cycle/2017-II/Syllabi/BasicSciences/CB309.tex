\begin{syllabus}

\course{CB309. Bioinformática}{Obligatorio}{CB309}

\begin{justification}
The use of computational methods in the biological sciences has become one of the key tools for the field of molecular biology, being a fundamental part of research in this area.
\\
In Molecular Biology, there are several applications that involve both DNA, protein analysis or sequencing of the human genome, which depend on computational methods. Many of these problems are really complex and deal with large data sets.
\\
This course can be used to see concrete use cases of several areas of knowledge of Computer Science such as Programming Languages (PL), Algorithms and Complexity (AL), Probabilities and Statistics, Information Management (IM), Intelligent Systems (IS).
\end{justification}

\begin{goals}
\item That the student has a solid knowledge of molecular biological problems that challenge computing.
\item That the student is able to abstract the essence of the various biological problems to pose solutions using their knowledge of Computer Science
\end{goals}

\begin{outcomes}
    \item \ShowOutcome{a}{2}
    \item \ShowOutcome{b}{3}
    \item \ShowOutcome{l}{3}
\end{outcomes}

\begin{competences}
    \item \ShowCompetence{C1}{a,b} 
    \item \ShowCompetence{C3}{b,l}
    \item \ShowCompetence{C5}{a,b}
\end{competences}

\begin{unit}{Introduction to Molecular Biology}{}{Clote2000,Setubal1997}{4}{CS1}
\begin{topics}
\item Review of organic chemistry: molecules and macromolecules, sugars, nucleic acids, nucleotides, RNA, DNA, proteins, amino acids and levels of structure in proteins.
\item The Dogma of Life: From DNA to Proteins, Transcription, Translation, Protein Synthesis.
\item Genome study: Maps and sequences, specific techniques
\end{topics}
   \begin{learningoutcomes}
      \item  Achive  a general knowledge of the most important topics in Molecular Biology. [\Familiarity]
	  \item Understand that biological problems are a challenge to the computational world. [\Assessment]
   \end{learningoutcomes}
\end{unit}

\begin{unit}{Sequence Comparison}{}{Clote2000,Setubal1997,Pevzner2000}{4}{CS2}
\begin{topics}
\item Sequences of nucleotides and amino acid sequences.
\item Sequence alignment, paired alignment problem, exhaustive search, Dynamic programming, global alignment, local alignment, gaps penalty
\item Comparison of multiple sequences: sum of pairs, complexity analysis by dynamic programming, alignment heuristics, star algorithm, progressive alignment algorithms.
\end{topics}
\begin{learningoutcomes}
\item  Understand and solve the problem of aligning a pair of sequences. [\Usage]
\item  Understand and solve the problem of multiple sequence alignment. [\Usage]
\item  Know the various algorithms for aligning existing sequences in the  literature . [\Familiarity]
\end{learningoutcomes}
\end{unit}

\begin{unit}{Phylogenetic Trees}{}{Clote2000,Setubal1997,Pevzner2000}{4}{CS2}
\begin{topics}
\item Phylogeny: Introduction and phylogenetic relations
\item Phylogenetic trees: definition, type of trees, problem of search and reconstruction of trees
\item Reconstruction methods: parsimony methods, distance methods, maximum likelihood methods, confidence of reconstructed trees
\end{topics}

\begin{learningoutcomes}
\item Understand the concept of phylogeny, phylogenetic trees and the methodological difference between biology and molecular biology. [\Familiarity]
\item Understand the problem of the reconstruction of phylogenetic trees, to know and apply the main algorithms for the reconstruction of phylogenetic trees. [\Assessment]
\end{learningoutcomes}
\end{unit}

\begin{unit}{DNA Sequence Assembling}{}{Setubal1997,Aluru2006}{4}{CS2}
\begin{topics}
\item Biological basis: ideal case, difficulties, alternative methods for DNA sequencing
\item Formal Assembly Models: Shortest Common Superstring, Reconstruction, Multicontig
\item Algorithms for sequence assembly: representation of overlaps, paths to create superstrings, voracious algorithm, acyclic graphs.
\item Assembly heuristics: search for overlays, ordering fragments, alignments and consensus.
\end{topics}

\begin{learningoutcomes}
\item Understand the computational challenge of the Sequence Assembly problem. [\Familiarity]
\item Understand the principle of formal model for assembly. [\Assessment]
\item Know the main heuristics for the problem of assembjale of DNA sequences[\Usage]
\end{learningoutcomes}
\end{unit}

\begin{unit}{Secondary and tertiary structures}{}{Setubal1997,Clote2000,Aluru2006}{4}{CS2}
   \begin{topics}
    \item Molecular structures: primary, secondary, tertiary, quaternary.
    \item Prediction of secondary structures of RNA: formal model, pair energy, structures with independent bases, solution with Dynamic Programming, structures with loops.
    \item {\it Protein folding}: Estructuras en proteinas, problema de protein folding.
    \item {\it Protein Threading}: Definitions, Branch \ Bound Algorithm, Branch \ Bound for protein threading.
    \item {\it Structural Alignment}: Definitions, DALI algorithm
   \end{topics}
   \begin{learningoutcomes}
     \item Know the protein structures and the necessity of computational methods for the prediction of the geometry. [\Familiarity]
	   \item Know the algorithms for solving prediction problems of secondary structures RNA, and structures in proteins. [\Assessment]
   \end{learningoutcomes}
\end{unit}

\begin{unit}{Probabilistic Models in Molecular Biology}{}{Durbin1998,Clote2000,Aluru2006,Krogh1994}{4}{CS2}
   \begin{topics}
    \item Probability: Random Variables, Markov Chains, Metropoli-Hasting Algorithm, Markov Random Fields, and Gibbs Sampler, Maximum Likelihood.
    \item Hidden Markov Models (HMM),  parameter estimation, Viterbi algorithm and Baul-Welch method, Application in paired and multiple alignments, Motifs detection in proteins, in eukaryotic DNA, in sequences families.
		\item Probabilistic phylogeny: probabilistic models of evolution, likelihood of alignments, likelihood for inference, comparison of probailistic and non-probabilistic methods
   \end{topics}
   \begin{learningoutcomes}
      \item  Review concepts of Probabilistic Models and understand their importance in Computational Molecular Biology. [\Assessment]
	  \item Know and apply Hidden Markov Models for various analyzes in Molecular Biology.. [\Usage]
		\item Know the application of probabilistic models in Phylogeny and to compare them with non-probabilistic models[\Assessment]
   \end{learningoutcomes}
\end{unit}



\begin{coursebibliography}
\bibfile{BasicSciences/CB309}
\end{coursebibliography}

\end{syllabus}
