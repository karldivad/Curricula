\begin{syllabus}

\course{FG210. Moral}{Obligatorio}{FG210}

\begin{justification}
La ética-moral comienza cuando se trata de elegir un sentido correcto de realización humana en su línea propia, un sentido capaz de desarrollar en plenitud sus posibilidades.
El problema de dar sentido a la vida es fundamental en el ser humano, ya que lo acompaña durante toda su existencia, y la ética-moral interpela a la persona a vivir según su fin último. En este sentido, la ética-moral busca la realización del hombre en la elección correcta de dicho fin.
\end{justification}

\begin{goals}
\item Formar la conciencia del estudiante para que pueda conducirse con criterio moralmente correcto en los ámbitos personal y profesional.

\end{goals}

\begin{outcomes}
    \item \ShowOutcome{e}{2}
    \item \ShowOutcome{ñ}{2}
\end{outcomes}

\begin{competences}
    \item \ShowCompetence{C10}{e}
    \item \ShowCompetence{C20}{e}
    \item \ShowCompetence{C21}{e, ñ}
    \item \ShowCompetence{C22}{ñ}
\end{competences}

\begin{unit}{}{Primera Unidad: La Ética Filosófica}{Lewis, Bourmaud, RodriguezL, AristotelesE}{9}{C10,C20}
\begin{topics}
    \item Presentación del curso.
    \item Lo ético y moral. La ética como rama de la filosofía.
    \item La necesidad de la metafísica.
    \item La experiencia moral.
    \item El problema del relativismo y su solución.
	
\end{topics}
\begin{learningoutcomes}
	\item Incorporar una primera noción de la ética y la moral, junto con los problemas que buscan resolver.[\Familiarity]
\end{learningoutcomes}
\end{unit}

\begin{unit}{}{Segunda Unidad: La acción moral}{SanchezM,Genta}{15}{C10,C20}
\begin{topics}
    \item Caracterización del actuar humano.
    \item Libertad, conciencia y voluntariedad. Distintos niveles de libertad. Factores que afectan la voluntariedad.
    \item El papel de la afectividad en la moralidad.
    \item La felicidad como fin último del ser humano.

\end{topics}
\begin{learningoutcomes}
	\item Analizar el acto humano, presentando sus condiciones y especificando su moralidad.[\Familiarity]
\end{learningoutcomes}
\end{unit}

\begin{unit}{}{Tercera Unidad: La vida virtuosa}{Piper,Droste,Lego}{12}{C21,C22}
\begin{topics}
    \item ¿Qué se entiende por virtud?
    \item La virtud moral: caracterización y modo de adquisición; el carácter dinámico de la virtud.
    \item Relación entre las distintas virtudes éticas. Las virtudes cardinales. Los vicios.
\end{topics}
\begin{learningoutcomes}
	\item Reflexionar respecto al ideal filosófico y moral de la vida virtuosa desde la práctica estable de bien y el rechazo constante de lo dañino.[\Familiarity]
\end{learningoutcomes}
\end{unit}

\begin{unit}{}{Cuarta Unidad: Lo éticamente correcto y su conocimiento}{ReydeCastro2010,SanchezM,Genta}{9}{C10,C20,C21}
\begin{topics}
    \item La corrección en lo ético.
    \item El conocimiento de lo éticamente correcto.
    \item La llamada ``recta razón'' y la ``verdad práctica''.
    \item Las leyes morales: ley natural y ley positiva.
    \item La conciencia moral: definición, tipos, deformaciones.
    \item La valoración moral de las acciones concretas.
\end{topics}

\begin{learningoutcomes}
	\item Discernir las nociones de recta razón, conciencia moral, y moral natural, remarcando la necesidad de la ley moral natural como el parámetro de conducta.[\Familiarity]
\end{learningoutcomes}
\end{unit}



\begin{coursebibliography}
\bibfile{GeneralEducation/FG101}
\end{coursebibliography}

\end{syllabus}
