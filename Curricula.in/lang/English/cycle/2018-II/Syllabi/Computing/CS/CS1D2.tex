\begin{syllabus}

<<<<<<< HEAD
\course{CS1701208. Estructuras Discretas II}{Obligatorio}{CS1701208} % Common.pm
=======
\course{CS1D2. Estructuras Discretas II}{Obligatorio}{CS1D2} % Common.pm
>>>>>>> 1b5a069032d66f0a8c046c7388bce355761e1a06

\begin{justification}
In order to understand the advanced computational techniques, the students must have a strong knowledge of the
Various discrete structures, structures that will be implemented and used in the laboratory in the programming language..
\end{justification}

\begin{goals}
\item That the student is able to model computer science problems using graphs and trees related to data structures
\item That the student applies efficient travel strategies to be able to search data in an optimal way
\end{goals}

\begin{outcomes}{V1}
    \item \ShowOutcome{a}{1}
    \item \ShowOutcome{b}{2}	
    \item \ShowOutcome{i}{1}
\end{outcomes}

\begin{outcomes}{V2}
    \item \ShowOutcome{1}{2}
    \item \ShowOutcome{2}{2}	
    \item \ShowOutcome{6}{2}
\end{outcomes}

\begin{competences}{V1}
    \item \ShowCompetence{C1}{a}
    \item \ShowCompetence{C5}{b}
    \item \ShowCompetence{CS2}{i}
\end{competences}

\begin{competences}{V2}
    \item \ShowCompetence{C1}{1}
    \item \ShowCompetence{C5}{2}
    \item \ShowCompetence{CS2}{6}
\end{competences}

\begin{unit}{Digital Logic and Data Representation}{}{Rosen2007,Grimaldi03}{10}{C1,C20}
    \begin{topics}
     \item Reticles: Types and properties.
     \item Boolean algebras.
     \item Boolean Functions and Expressions.
     \item Representation of Boolean Functions: Normal Disjunctive and Conjunctive Form.
     \item Logical gates.
     \item Circuit Minimization.
    \end{topics}
 
    \begin{learningoutcomes}
     \item Explain the importance of Boolean algebra as a unification of set theory and propositional logic [\Assessment].
     \item Explain the algebraic structures of reticulum and its types [\Assessment].
     \item Explain the relationship between the reticulum and the ordinate set and the wise use to show that a set is a reticulum [\Assessment].
     \item Explain the properties that satisfies a Boolean algebra [\Assessment].
     \item Demonstrate if a terna formed by a set and two internal operations is or not Boolean algebra [\Assessment].
     \item Find the canonical forms of a Boolean function  [\Assessment].
     \item Represent a Boolean function as a Boolean circuit using logic gates  [\Assessment].
     \item Minimize a Boolean function. [\Assessment].
     \end{learningoutcomes}
  \end{unit}

\begin{unit}{\DSBasicsofCounting}{}{Grimaldi97}{40}{C1} 
    \begin{topics}
        \item \DSBasicsofCountingTopicCounting
        \item \DSBasicsofCountingTopicThePigeonhole
        \item \DSBasicsofCountingTopicPermutations
        \item \DSBasicsofCountingTopicSolving
        \item \DSBasicsofCountingTopicBasic
   \end{topics}
   \begin{learningoutcomes}
        \item \DSBasicsofCountingLOApplyCounting [\Familiarity]
        \item \DSBasicsofCountingLOApplyThe[\Familiarity]
        \item \DSBasicsofCountingLOComputePermutations[\Familiarity]
        \item \DSBasicsofCountingLOMap[\Familiarity]
        \item \DSBasicsofCountingLOSolveA[\Familiarity]
        \item \DSBasicsofCountingLOAnalyzeA[\Familiarity]
        \item \DSBasicsofCountingLOPerformComputations[\Familiarity]
   \end{learningoutcomes}
\end{unit}

\begin{unit}{\DSGraphsandTrees}{}{Johnsonbaugh99}{40}{C1}
    \begin{topics}
        \item \DSGraphsandTreesTopicTrees
	\item \DSGraphsandTreesTopicUndirected
	\item \DSGraphsandTreesTopicDirected
	\item \DSGraphsandTreesTopicWeighted
	\item \DSGraphsandTreesTopicSpanning
	\item \DSGraphsandTreesTopicGraph
   \end{topics}
   \begin{learningoutcomes}
	\item \DSGraphsandTreesLOIllustrate[\Familiarity]
	\item \DSGraphsandTreesLODemonstrateDifferent[\Familiarity]
	\item \DSGraphsandTreesLOModel[\Familiarity]
	\item \DSGraphsandTreesLOShowHow[\Familiarity]
	\item \DSGraphsandTreesLOExplainHowA [\Familiarity]
	\item \DSGraphsandTreesLODetermineIf [\Familiarity]
   \end{learningoutcomes}
\end{unit}

\begin{coursebibliography}
\bibfile{Computing/CS/CS1D1}
\end{coursebibliography}

\end{syllabus}
