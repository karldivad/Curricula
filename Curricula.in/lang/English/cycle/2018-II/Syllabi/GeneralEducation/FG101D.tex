\begin{syllabus}

\course{FG1702229. Ecología y Conservación Ambiental}{Obligatorio}{FG1702229} % Common.pm

\begin{justification}
During the plenary sessions, there will be lectures related to the methodology of Design Thinking as well as its use and importance in the creation processes. Also, during these sessions we will have presentations on entrepreneurships and startups related to engineering or technology.
During lab sessions, students form teams that maintain during the cycle. With the guidance of the teacher and through the methodology of Design Thinking developed in the plenary sessions, students will have to present innovative solutions to real problems inspired by the United Nations "Global Challenges".
The students will have a Digital Log which will be constantly reviewed by the teachers in charge. In it will be the advances, processes and referents of the group project. The course culminates with the presentations of the proposals put forward by the groups.
\end{justification}

\begin{goals}
\item Ability to design and carry out experiments.
\item Ability to analyze information.
\item Ability to design a system, a component or a process to meet the desired needs within realistic constraints (Level 1)
\item Teamwork Ability.
\item Ability to lead a team.
\item Oral communication skills (Level 1)
\item Written communication skills (Level 1)
\item Understand the impact of engineering solutions in a global, economic, environmental and societal context.
\end{goals}
    
\begin{outcomes}{V1}
    \item \ShowOutcome{n}{2}
    \item \ShowOutcome{ñ}{2}
\end{outcomes}

\begin{competences}{V1}
    \item \ShowCompetence{C20}{n,ñ}
\end{competences}

\begin{unit}{Global Challenges.}{}{Curedale12,Upton15}{12}{4}
   \begin{topics}
      \item Methodology of Design Thinking (DT).
      \item DT Steps.
      \item Technique and use of Brainstorming.
      \item Knowledge of the user, empathy and use of archetypes.
      \item Types of research, differences and uses.
      \item Strategies for gathering from Insights.
      \item Ideation methods.
      \item Introduction tool Prototyping.
      \item Introducction to User Experience.
      \item Testing and Iteration Strategies.
      \item Uses of Storytelling.
   \end{topics}
   \begin{learningoutcomes}
      \item Flexibility and Adaptability: Students learn to work in a team in a flexible and variable environment with constant challenges.
   \end{learningoutcomes}
\end{unit}

\begin{coursebibliography}
\bibfile{GeneralEducation/FG101D}
\end{coursebibliography}

\end{syllabus}
