\begin{syllabus}

\course{FG101. Comunicación}{Obligatorio}{FG101}

\begin{justification}
Para lograr una eficaz comunicación en el ámbito personal y profesional, es prioritario el manejo adecuado de la Lengua en forma oral y escrita. Se justifica, por lo tanto, que los alumnos de la Universidad Católica San Pablo conozcan, comprendan y apliquen los aspectos conceptuales y operativos de su idioma, para el desarrollo de sus habilidades comunicativas fundamentales: Escuchar, hablar, leer y escribir.
En consecuencia el ejercicio permanente y el aporte de los fundamentos contribuyen grandemente en la formación académica y, en el futuro, en el desempeño de su profesión.
El curso de Comunicación I corresponde a la Línea de Comunicación en el área de Humanidades y se desarrolla en el programa profesional de E y Telecomunicaciones. Es de carácter teórico-práctico y constituye prerrequisito del curso de comunicación II. Su propósito fundamental es desarrollar las habilidades de redacción, expresión oral, comprensión e interpretación de textos. Los objetivos del curso responden a la necesidad de mejorar y facilitar la interacción social. La organización temática comprende las  áreas de producción de textos, expresión oral, comprensión e interpretación de textos; para ello aborda los siguientes contenidos: elementos de lingüística (comunicación, el lenguaje, lengua, lingüística), redacción (textos descriptivos, narrativos, expositivos, argumentativos y documentos administrativos), análisis de la organización de los textos escritos, elaboración de indicadores de comprensión, estudio de la palabra morfológica y sintácticamente, el discurso y el estilo oral.
\end{justification}

\begin{goals}
\item Desarrollar competencias comunicativas a través de la teoría y práctica del lenguaje que ayuden al estudiante en su formación como profesional y persona humana. [\Usage]
\end{goals}

\begin{outcomes}
   \item \ShowOutcome{d}{2}
   \item \ShowOutcome{f}{2}
\end{outcomes}

\begin{competences}
    \item \ShowCompetence{C17}{f}
    \item \ShowCompetence{C18}{d}
\end{competences}

\begin{unit}{}{Primera Unidad}{espanola2010,magallanes2000}{}{C17,C18}
\begin{topics}
    \item La Comunicación.
		\subitem Elementos.
		\subitem Proceso.
		\subitem Funciones.
		\subitem Clasificación.
	\item El lenguaje.
		\subitem Características.
		\subitem Funciones.
		\subitem Lengua:niveles.
		\subitem Habla.
		\subitem Sistema.
		\subitem Norma.
    \item La Lingüística.
		\subitem Enfoques.
		\subitem Disciplinas.
		\subitem Singno Lingüístico: características.
    \end{topics}
	
\begin{learningoutcomes}
   \item Identificar la comunicación como un proceso de producción, comprensión e intercambio de mensajes.[\Familiarity]
   \item Caracterizar el lenguaje, la lengua y el habla. [\Familiarity]
   \item Identificar y valorar la lingüística como fundamento en el proceso de comunicación.[\Familiarity]
\end{learningoutcomes}

\end{unit}

\begin{unit}{}{Segunda Unidad}{Espanola2010,gatti2007}{}{C17,C18}
\begin{topics}
	\item La palabra.
		\subitem El morfema: clases.
		\subitem El sustantivo, adjetivo, artículo, pronombre, verbo, adverbio, preposición.
	\item Componente sintáctico.
		\subitem La oración enunciativa, interrogativa, imperativa, exclamativa, optativa.
		\subitem La proposición y la frase. 
		\subitem El sintagma: estructura y clases: nominal, verbal, adjetival, preposicional, adverbial.
	\item Indicadores de comprensión (organizadores gráficos: mapas conceptual, cuadro sinóptico, gráfico de causa - efecto).
		\subitem La paráfrasis.
		\subitem El comentario.
\end{topics}
\begin{learningoutcomes}
   \item Identificar las cualidades de la palabra,  sus clases y funciones sintácticas. [\Familiarity]
   \item Reconocer y analizar  la estructura oracional valorando su importancia y utilidad en la comprensión y redacción de texto. [\Familiarity]
   \item Emplear organizadores gráficos para la verificación de la comprensión de textos. [\Usage]
\end{learningoutcomes}
\end{unit}

\begin{unit}{}{Tercera Unidad}{sanchez2006,Cueva2004}{}{C17,C18}
\begin{topics}
	\item Organización del texto.
		\subitem La referencia (deixis).
		\subitem Anáfora, catáfora, elipsis.
		\subitem Las conexiones lógicas y textuales.
	\item Estructura del párrafo y del texto: ideas principales  e ideas secundarias.
	\item Funciones de elocución en el texto.
		\subitem Generalización, identificación, nominalización, clasificación, ejemplificación, definición.
	\item Etapas de la redacción.
	\item Tipos de texto.
		\subitem Descriptivo: Descripción de lugares y objetos.
		\subitem Narrativo: La anécdota.
		\subitem Expositivo: de divulgación (artículo). 
		\subitem Argumentativo: El ensayo.
	\item Redacción de documentación administrativa.
		\subitem La carta.
		\subitem La solicitud.
		\subitem El oficio.
\end{topics}

\begin{learningoutcomes}
   \item Analizar textos escritos empleando estrategias de lectura y redacción según el tipo de texto. [\Familiarity]
   \item Utilizar técnicas de redacción  según las necesidades y el tipo de texto empleando estrategias para distintas situaciones. [\Usage]
\end{learningoutcomes}
\end{unit}

\begin{unit}{}{Cuarta Unidad}{vivaldi2000,magallanes2000}{}{C17,C18}
\begin{topics}
	\item El estilo oral.
	\item Habilidades del comunicador.
	\item El discurso oral: 
		\subitem Propósitos.
		\subitem Partes.
	\item Escuchar: condiciones, propósitos.
	\item Prácticas discursivas orales
\end{topics}
\begin{learningoutcomes}
   \item Aplicar habilidades como emisor o receptor en distintas situaciones de comunicación.[\Usage]
\end{learningoutcomes}
\end{unit}

\begin{coursebibliography}
\bibfile{GeneralEducation/FG101}
\end{coursebibliography}

\end{syllabus}
