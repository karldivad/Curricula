\begin{syllabus}

\course{FG1701215. Comunicación Integral}{Obligatorio}{FG1701215} % Common.pm

\begin{justification}
In order to achieve an effective communication in the personal and professional field, the proper management of the Language in oral and written form is a priority. It is therefore justified that the students know, understand and apply the conceptual and operative concepts of their language, for the development of their fundamental communicative skills: Listening, speaking, reading and writing.
Consequently the permanent exercise and the contribution of fundamentals contribute greatly in the academic formation and in the future, in the performance of its profession.
\end{justification}

\begin{goals}
\item Develop communicative skills through the theory and practice of language that help the student to overcome the academic requirements of the undergraduate and contribute to his humanistic training and as a human person.
\end{goals}

\begin{outcomes}{V1}
   \item \ShowOutcome{f}{2}
   \item \ShowOutcome{h}{2}
   \item \ShowOutcome{n}{2}
\end{outcomes}

\begin{competences}{V1}
    \item \ShowCompetence{C17}{f,h,n}
    \item \ShowCompetence{C20}{f,n}
    \item \ShowCompetence{C24}{f,h}
\end{competences}

\begin{unit}{Course Presentation}{}{Real}{16}{C17,C20}
  \begin{topics}
      \item Oral and written communication I.
  \end{topics}

  \begin{learningoutcomes}
   \item That the student has an approximation to some characteristics of the formal writing.
  \end{learningoutcomes}
\end{unit}

\begin{unit}{Characteristics of Academic Writing}{}{Real}{16}{C17,C20}
  \begin{topics}
      \item Characteristics of Academic Writing
  \end{topics}

  \begin{learningoutcomes}
   \item The student will be able to differentiate between a Formal or Informal text.
  \end{learningoutcomes}
\end{unit}

\begin{unit}{Reading strategies}{}{Real}{16}{C17,C20}
  \begin{topics}
      \item Characteristics of reading strategies .
  \end{topics}

  \begin{learningoutcomes}
   \item The student will be able to develop a reading and summary scheme.
  \end{learningoutcomes}
\end{unit}

\begin{unit}{Structure of the Text}{}{Real}{16}{C17,C20}
  \begin{topics}
      \item Parts of text.
      \item Identification of structure in texts.
  \end{topics}

  \begin{learningoutcomes}
   \item .%ToDo
  \end{learningoutcomes}
\end{unit}

\begin{unit}{Structure of Paragraphs}{}{Real}{16}{C17,C20}
  \begin{topics}
      \item Parts of paragraph .
      \item Paragraph scheme.
  \end{topics}

  \begin{learningoutcomes}
   \item .%ToDo
  \end{learningoutcomes}
\end{unit}

\begin{unit}{Characteristics of the paragraph}{}{Real}{16}{C17,C20}
  \begin{topics}
      \item Paragraph Writing Exercises.
      \item General indications for the Final Work.
      \item Co-evaluation on paragraph in FORUM.
  \end{topics}

  \begin{learningoutcomes}
   \item .%ToDo
  \end{learningoutcomes}
\end{unit}

\begin{unit}{Synthesis Scheme}{}{Real}{16}{C17,C20}
  \begin{topics}
      \item Scheme and Summary.
  \end{topics}

  \begin{learningoutcomes}
   \item .%ToDo
  \end{learningoutcomes}
\end{unit}

\begin{unit}{Argumentative vs. expository}{}{Real}{16}{C17,C20}
  \begin{topics}
      \item Characteristics and parts of expository text.
      \item Writing process.
  \end{topics}

  \begin{learningoutcomes}
   \item .%ToDo
  \end{learningoutcomes}
\end{unit}

\begin{unit}{Writing process}{}{Real}{16}{C17,C20}
  \begin{topics}
      \item Delimitation of theme and production scheme.
      \item Preparation consultancy
      \item Theme, outline, production (parts and sub-parts) .
  \end{topics}

  \begin{learningoutcomes}
   \item .%ToDo
  \end{learningoutcomes}
\end{unit}

\begin{unit}{Virtual Presentation}{}{Real}{16}{C17,C20}
  \begin{topics}
      \item Delimited topic.
      \item Justification.
      \item Scheme.
      \item Report of Sources.
  \end{topics}

  \begin{learningoutcomes}
   \item .%ToDo
  \end{learningoutcomes}
\end{unit}

\begin{unit}{Writing Process}{}{Real}{16}{C17,C20}
  \begin{topics}
      \item Function and types (APA 6th edition).
  \end{topics}

  \begin{learningoutcomes}
   \item .%ToDo
  \end{learningoutcomes}
\end{unit}

\begin{unit}{Quotes}{}{Real}{16}{C17,C20}
  \begin{topics}
      \item Function and types.
      \item Bibliography.
  \end{topics}

  \begin{learningoutcomes}
   \item .%ToDo
  \end{learningoutcomes}
\end{unit}

\begin{unit}{Types of paragraphs}{}{Real}{16}{C17,C20}
  \begin{topics}
      \item Types of paragraphs.
      \item Group work in class.
  \end{topics}

  \begin{learningoutcomes}
   \item .%ToDo
  \end{learningoutcomes}
\end{unit}

\begin{unit}{Approximation to characteristics of the oral presentation}{}{Real}{16}{C17,C20}
  \begin{topics}
      \item Approximation to characteristics of the oral presentation.
      \item Writing Exercises.
  \end{topics}

  \begin{learningoutcomes}
   \item .%ToDo
  \end{learningoutcomes}
\end{unit}

\begin{unit}{Comparative Paragraph}{}{Real}{16}{C17,C20}
  \begin{topics}
      \item Establishment of Criteria
      \item Advance 2-Consulting
  \end{topics}

  \begin{learningoutcomes}
   \item .%ToDo
  \end{learningoutcomes}
\end{unit}

\begin{unit}{Characteristics of oral exposition and paragraph types}{}{Real}{16}{C17,C20}
  \begin{topics}
      \item Coevaluations: enumerative and comparative paragraphs.
      \item FORUM: characteristics of orality in an academic context.
  \end{topics}

  \begin{learningoutcomes}
   \item .%ToDo
  \end{learningoutcomes}
\end{unit}

\begin{unit}{Full text writing}{}{Real}{16}{C17,C20}
  \begin{topics}
      \item Writing full text with quotations.
  \end{topics}

  \begin{learningoutcomes}
   \item .%ToDo
  \end{learningoutcomes}
\end{unit}

\begin{unit}{Theory: writing of complete texts}{}{Real}{16}{C17,C20}
  \begin{topics}
      \item Consulting
      \item Indications for the third advance.
  \end{topics}

  \begin{learningoutcomes}
   \item .%ToDo
  \end{learningoutcomes}
\end{unit}

\begin{unit}{Exposition}{}{Real}{16}{C17,C20}
  \begin{topics}
      \item Exhibition Feedback.
      \item Evaluations.
  \end{topics}

  \begin{learningoutcomes}
   \item .%ToDo
  \end{learningoutcomes}
\end{unit}

\begin{coursebibliography}
\bibfile{GeneralEducation/FG250}
\end{coursebibliography}

\end{syllabus}
