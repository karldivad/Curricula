\begin{syllabus}

\course{CB101. �lgebra y Geometr�a}{Obligatorio}{CB101}

\begin{justification}
Este curso busca proporcionar a los y las estudiantes ciertos marcos referenciales con los cuales analizar las disyuntivas que se pueden presentar en su ejercicio profesional. El curso pone en pr�ctica constante el razonamiento cr�tico y responsable de los  y las estudiantes, siendo esta una competencia fundamental para los procesos de toma de decisi�n que asumiremos como profesionales y ciudadanos.
El pensamiento �tico tiene una implicancia directa en el campo de las ciencias y las ingenier�as. La informaci�n y los dispositivos tecnol�gicos son herramientas de poder que deben ponerse al servicio de la humanidad.  Seg�n esto, la dignidad humana y la buena convivencia son las dos fronteras que deben regular y orientar el avance tecnol�gico y la diseminaci�n de informaci�n. Los y las estudiantes 
se aproximar�n a  la tecnolog�a y la ingenier�a como cuestiones que deben responder a las necesidades de �ndole �tica y comunitaria de nuestra sociedad. 
Finalmente, nos encontramos en un momento de la historia de la humanidad en que nos enfrentamos a desaf�os nuevos como la degradaci�n ambiental, el transhumanismo o el ciberterrorismo, por nombrar algunos. Los estudiantes de UTEC deben estar en la capacidad de saber abordar problemas complejos y poder proponer soluciones sostenibles para el futuro del mundo y de nuestra especie. 



\end{justification}

\begin{goals}
\item Introducir a los estudiantes al pensamiento cr�tico y �tico aplicado a su campo profesional. 
\item Fortalecer en el estudiante la capacidad de pensar interdisciplinariamente. 
\item Desarrollar la competencia de mirar un fen�meno desde varias disciplinas y perspectivas genera en la persona empat�a y respeto a la diversidad de opini�n.
\item Capacidad de trabajo en equipo.
\item Capacidad para identificar problemas.
\item Comprende las responsabilidades profesional y �tica.
\item Capacidad de comunicaci�n oral.
\item Comprende el impacto de las soluciones de la ingenier�a en un contexto global, econ�mico, ambiental y de la sociedad.
\item Tiene inter�s por conocer sobre temas actuales de la sociedad peruana y del mundo.
\item Capacidad de comunicaci�n escrita.

\end{goals}

\begin{outcomes}
\ExpandOutcome{a}{3}
\ExpandOutcome{i}{2}
\ExpandOutcome{j}{4}
\end{outcomes}

\begin{unit}{Sistemas de coordenadas. La recta.}{Lehmann05}{12}{4}
   \begin{topics}
      \item 
      \item 
   \end{topics}
   \begin{unitgoals}
      \item 
   \end{unitgoals}
\end{unit}

\begin{unit}{C�nicas y Coordenadas polares}{Lehmann05}{24}{3}
   \begin{topics}
      \item 
      \item 
   \end{topics}

   \begin{unitgoals}
      \item 
      \item
      \item 
      \end{unitgoals}
\end{unit}

\begin{unit}{Sistemas de ecuaciones. Matrices y determinantes}{Strang03,Grossman96}{24}{3}
   \begin{topics}
      \item 
      \item 
      \item 
      \end{topics}

   \begin{unitgoals}
      \item 
      \item 
      \item 
     
   \end{unitgoals}
\end{unit}

\begin{unit}{Vectores en $R^2$ y vectores en $R^3$}{Grossman96}{30}{3}
   \begin{topics}
      \item 
      \item 
   \end{topics}

   \begin{unitgoals}
      \item 
      \item 
      \item 
   \end{unitgoals}
\end{unit}



\begin{coursebibliography}
\bibfile{GeneralH/GH0010}
\end{coursebibliography}

\end{syllabus}
