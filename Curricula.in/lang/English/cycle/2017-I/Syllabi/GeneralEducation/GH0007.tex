\begin{syllabus}

\course{CB101. Álgebra y Geometría}{Obligatorio}{CB101}

\begin{justification}
This course aims to provide students with a real-life hands-on experience in the first steps within a business life cycle, through which an idea becomes a formal business model.
It is the first of a set of three courses designed to accompany students as they transform an idea into a prospective business or business, from idea to review of current business strategy.
\end{justification}

\begin{goals}
   \item Ability to analyze information.
   \item Interpretation of information and results.
   \item Teamwork Ability.
   \item Ethics.
   \item Oral communication.
   \item Written communication.
   \item Graphic communication.
   \item Understand the need to learn continuously
\end{goals}

\begin{outcomes}
    \item \ShowOutcome{n}{2}
    \item \ShowOutcome{ñ}{2}
\end{outcomes}

\begin{competences}
    \item \ShowCompetence{C24}{n,ñ}
\end{competences}


\begin{unit}{}{El ciclo de vida empresarial: desde la idea hasta la revisión de su estrategia.}{Fitzpatrick13}{12}{4}
   \begin{topics}
      \item .
   \end{topics}
   \begin{learningoutcomes}
      \item .
   \end{learningoutcomes}
\end{unit}

\begin{unit}{}{El proceso de ideación y la visión del cliente}{Osterwalder10}{24}{3}
   \begin{topics}
      \item . 
   \end{topics}

   \begin{learningoutcomes}
      \item .
      \end{learningoutcomes}
\end{unit}

\begin{unit}{}{¿Cómo construir y mantener equipos eficaces?}{Osterwalder10}{24}{3}
   \begin{topics}
      \item . 
      \end{topics}

   \begin{learningoutcomes}
      \item .
   \end{learningoutcomes}
\end{unit}

\begin{unit}{}{Ejecución de LEAN: lo básico}{Osterwalder10}{30}{3}
   \begin{topics}
      \item .
   \end{topics}

   \begin{learningoutcomes}
      \item . 
   \end{learningoutcomes}
\end{unit}

\begin{unit}{}{Diseño de un modelo de negocio: herramientas de diseño y lienzo.}{Fitzpatrick13}{30}{3}
   \begin{topics}
      \item .
   \end{topics}

   \begin{learningoutcomes}
      \item .
   \end{learningoutcomes}
\end{unit}

\begin{unit}{}{Generación de Modelos de Negocio: la Lona Modelo de Negocio (Osterwalder).}{Osterwalder10}{30}{3}
   \begin{topics}
      \item . 
   \end{topics}

   \begin{learningoutcomes}
      \item . 
   \end{learningoutcomes}
\end{unit}

\begin{unit}{}{Ingeniería de Venture: usar las habilidades de la informática para construir un modelo de negocio efectivo.}{Fitzpatrick13}{30}{3}
   \begin{topics}
      \item .
   \end{topics}

   \begin{learningoutcomes}
      \item .
   \end{learningoutcomes}
\end{unit}

\begin{unit}{}{Herramientas de investigación de mercado primario y nichos de mercado.}{Grossman96}{30}{3}
   \begin{topics}
      \item .
   \end{topics}

   \begin{learningoutcomes}
      \item . 
   \end{learningoutcomes}
\end{unit}

\begin{unit}{}{La Importancia del Capital: Humano, Financiero e Intelectual.}{Grossman96}{30}{3}
   \begin{topics}
      \item .
   \end{topics}

   \begin{learningoutcomes}
      \item . 
   \end{learningoutcomes}
\end{unit}

\begin{unit}{}{Técnicas de monetización y financiamiento.}{Fitzpatrick13}{30}{3}
   \begin{topics}
      \item . 
   \end{topics}

   \begin{learningoutcomes}
      \item. 
   \end{learningoutcomes}
\end{unit}


\begin{unit}{}{Comunicación eficaz: crear una presentación de un modelo de negocio de impacto.}{Fitzpatrick13}{30}{3}
   \begin{topics}
      \item . 
   \end{topics}

   \begin{learningoutcomes}
      \item .
   \end{learningoutcomes}
\end{unit}
\begin{coursebibliography}
\bibfile{GeneralEducation/GH0007}
\end{coursebibliography}

\end{syllabus}
