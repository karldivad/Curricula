\begin{syllabus}

\course{GH1013. Criticism of Modernity}{Obligatorio}{GH1013}

\begin{justification}
The development of the course serves three objectives: firstly, to understand modernity from three perspectives: scientific, social and artistic-cultural; secondly, to revise the fundamental concepts that constituted the modern imaginary from the perspective of their cultural, scientific and institutional works; and, thirdly, to discuss and talk about
current criticisms of modernity and its principles from the new philosophical, sociological and artistic theories of the twentieth and the twenty-first century. These three objectives are organized around the three horizons that make up the main intention of our course: the modern scientific horizon, the socio-political horizon and the artistic horizon. From this triple horizon, we will expose, discuss and participate in the reconstruction of the imaginary and the modern mentality, whose multiple contributions
and developments still constitute and develop our contemporary imaginary. In fact, it is not possible to interpret our actuality and its consequences, if not on the base of the social, scientific and scientific products that are projected by modernity. However, our course is not only limited to historical revision; in addition if offers an essential
component of any historical understanding: the analysis and the revision of the boundaries and scope of modernity. From this revision, we will better understand the contemporary consequences of modern projects and how they were surpassed, continued or have agreed to the rewriting of their intentions- not from a modern, but from a postmodern perspective.
\end{justification}

\begin{goals}
\item Identify the fundamental concepts of Modernity.
\item Understand the historical-cultural situation of the twentieth and the twenty-first century called Postmodernity.
\end{goals}

\begin{outcomes}
    \item \ShowOutcome{d}{2} % Multidiscip teams
    \item \ShowOutcome{e}{2} % ethical, legal, security and social implications
    \item \ShowOutcome{f}{2} % communicate effectively
    \item \ShowOutcome{n}{2} % Apply knowledge of the humanities
    \item \ShowOutcome{o}{2} % TASDSH
\end{outcomes}

\begin{competences}
    \item \ShowCompetence{C10}{d,n,o}
    \item \ShowCompetence{C17}{f}
    \item \ShowCompetence{C18}{f}
    \item \ShowCompetence{C21}{e}
\end{competences}

\begin{unit}{Criticism of Modernity}{}{Danto99}{12}{4}
   \begin{topics}
      \item What does being modern mean?
      \item The Scientific Revolutions
      \item The Mathematical Model of Nature: Copernico and Newton
      \item New Specific Revolutions: The peak of the Twentieth Century
      \item Nature and Art
      \item Birth of the Subject
      \item Disenchantment and Demystification. Secularization
      \item The Origin of Art Criticism
      \item The End Of Art: Of Urinaries And Bicycle Wheels    
   \end{topics}
   \begin{learningoutcomes}
      \item Value and critically revise the general artistic and cultural developments of Modernity from contemporary considerations.
   \end{learningoutcomes}
\end{unit}




\begin{coursebibliography}
\bibfile{GeneralH/GH1013}
\end{coursebibliography}

\end{syllabus}
