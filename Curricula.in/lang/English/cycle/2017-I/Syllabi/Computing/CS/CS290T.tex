\begin{syllabus}

\course{CS290T. Ingeniería de Software I}{Obligatorio}{CS290T}

\begin{justification}
La taréa de desarrollar software, excepto para aplicaciones sumamente simples, exige la ejecución de un desarrollo bien definido. Los profesionales de esta área requieren un alto grado de conocimiento de los diferentes modelos e proceso de desarrollo, para que sean capaces de elegir el más idóneo para cada proyecto de desarrollo. Por otro lado, el desarrollo de sistemas de mediana y gran escala requiere del uso de bibliotecas de patrones y componentes y del dominio de técnicas relacionadas al diseño basado en componentes
\end{justification}

\begin{goals}
\item Brindar al alumno un marco teórico y práctico para el desarrollo de software bajo estándares de calidad.
\item Familiarizar al alumno con los procesos de modelamiento y construcción de software a través del uso de herramientas CASE.
\item Los alumnos debe ser capaces de seleccionar Arquitecturas y Plataformas tecnológicas ad-hoc a los escenarios de implementación.
\item Aplicar el modelamiento basado en componentes y fin de asegurar variables como calidad, costo y time-to-market en los procesos de desarrollo.
\item Brindar a los alumnos mejores prácticas para la verificación y validación del software
\end{goals}

\begin{outcomes}
    \item \ExpandOutcome{b}{2}
    \item \ExpandOutcome{c}{2}
	\item \ExpandOutcome{f}{2}
	\item \ExpandOutcome{i}{3}
	\item \ExpandOutcome{k}{2}
\end{outcomes}

\begin{competences}
    \item \Competence{C7}{b,k} 
    \item \Competence{C8}{b,c,k} 
    \item \Competence{C11}{c}
	\item \Competence{C12}{c,i}
	\item \Competence{C13}{c,i}
	\item \Competence{C18}{k}
	\item \Competence{CS1}{c}
	\item \Competence{CS2}{b,c}
	\item \Competence{CS4}{b,c,i}
	\item \Competence{CS5}{b,c,i}
	\item \Competence{CS10}{i,k}
\end{competences}

\begin{unit}{\SERequirementsEngineering}{}{Pressman2005, Sommerville2008, Larman2008}{18}{C7, C11, CS2}
	\begin{topics}
		\item \SERequirementsEngineeringTopicDescribing
		\item \SERequirementsEngineeringTopicProperties
		\item \SERequirementsEngineeringTopicSoftwareRequirements
		\item \SERequirementsEngineeringTopicDescribingSystem
		\item \SERequirementsEngineeringTopicNonFunctional
		\item \SERequirementsEngineeringTopicEvaluationAnd
		\item \SERequirementsEngineeringTopicRequirements
		\item \SERequirementsEngineeringTopicAcceptability
		\item \SERequirementsEngineeringTopicPrototyping
		\item \SERequirementsEngineeringTopicBasicConcepts
		\item \SERequirementsEngineeringTopicRequirementsSpecification
		\item \SERequirementsEngineeringTopicRequirementsValidation
		\item \SERequirementsEngineeringTopicRequirementsTracing
	\end{topics}
	\begin{learningoutcomes}
		\item \SERequirementsEngineeringLOListTheOfCase[\Assessment]
		\item \SERequirementsEngineeringLODescribeHowEngineering[\Assessment]
		\item \SERequirementsEngineeringLOInterpret[\Assessment]
		\item \SERequirementsEngineeringLODescribeTheOfTechniques[\Assessment]
		\item \SERequirementsEngineeringLOListTheOfModel[\Assessment]
		\item \SERequirementsEngineeringLOIdentifyBoth[\Assessment]
		\item \SERequirementsEngineeringLOConductAA[\Assessment]
		\item \SERequirementsEngineeringLOApplyKey[\Assessment]
		\item \SERequirementsEngineeringLOCompareTheAnd[\Assessment]
		\item \SERequirementsEngineeringLOUseAFormal[\Assessment]
		\item \SERequirementsEngineeringLOTranslateInto[\Assessment]
		\item \SERequirementsEngineeringLOCreateAA[\Assessment]
		\item \SERequirementsEngineeringLODifferentiateBetweenBackward[\Assessment]
	\end{learningoutcomes}
\end{unit}

\begin{unit}{\SESoftwareDesign}{}{Pressman2005, Sommerville2008, Larman2008}{18}{C5, C7, C8, CS10}
	\begin{topics}
		\item \SESoftwareDesignTopicSystemDesign%
		\item \SESoftwareDesignTopicDesignParadigms%
		\item \SESoftwareDesignTopicStructuralAnd%
		\item \SESoftwareDesignTopicDesignPatterns%
		\item \SESoftwareDesignTopicRelationships%
		\item \SESoftwareDesignTopicSoftwareArchitectureConcepts%
		\item \SESoftwareDesignTopicTheUse%
		\item \SESoftwareDesignTopicInternal%
		\item \SESoftwareDesignTopicInternalDesignQualitiesAndV%
		\item \SESoftwareDesignTopicMeasurement%
		\item \SESoftwareDesignTopicTradeoffsBetweenDifferent%
		\item \SESoftwareDesignTopicApplicationFrameworks%
		\item \SESoftwareDesignTopicMiddleware%
		\item \SESoftwareDesignTopicPrinciplesOfSecure%
	\end{topics}
	\begin{learningoutcomes}
		\item \SESoftwareDesignLOArticulateDesign [\Familiarity] %
		\item \SESoftwareDesignLOUseAToSimple [\Usage] %
		\item \SESoftwareDesignLOConstructModels [\Usage] %
		\item \SESoftwareDesignLOWithin [\Familiarity] %
		\item \SESoftwareDesignLOForASuitable [\Usage] %
		\item \SESoftwareDesignLOCreateAppropriate [\Usage] %
		\item \SESoftwareDesignLOExplainTheTheA [\Assessment] %
		\item \SESoftwareDesignLOForThe [\Familiarity] %
		\item \SESoftwareDesignLOGiven [\Familiarity] %
		\item \SESoftwareDesignLOInvestigateThe [\Assessment] %
		\item \SESoftwareDesignLOApplySimple [\Usage] %
		\item \SESoftwareDesignLODescribeARefactoring [\Familiarity] %
		\item \SESoftwareDesignLOSelectSuitable [\Usage] %
		\item \SESoftwareDesignLOExplainHowMight [\Familiarity] %
		\item \SESoftwareDesignLODesignAA [\Usage] %
		\item \SESoftwareDesignLODiscussAnd [\Usage] %
		\item \SESoftwareDesignLOApplyModels [\Usage] %
		\item \SESoftwareDesignLOAnalyzeAFrom [\Assessment] %
		\item \SESoftwareDesignLOAnalyzeAFromOf [\Assessment] %
		\item \SESoftwareDesignLOExplainTheObjects [\Familiarity] %
		\item \SESoftwareDesignLOApplyComponent [\Usage] %
		\item \SESoftwareDesignLORefactorAn [\Usage] %
		\item \SESoftwareDesignLOStateAnd [\Familiarity] %
	\end{learningoutcomes}
\end{unit}

\begin{unit}{\SESoftwareConstruction}{}{Pressman2005, Sommerville2008, Larman2008}{24}{C4, C5, C7, C8, CS2}
	\begin{topics}
		\item \SESoftwareConstructionTopicCoding
		\begin{subtopic}
			\item Prácticas de codificación defensive
			\item Prácticas de codificación segura
			\item Utilizando mecanismos de manejo de excepciones para hacer el programa más robusto, tolerante a fallas
		\end{subtopic}
		\item \SESoftwareConstructionTopicCodingStandards
		\item \SESoftwareConstructionTopicIntegration
		\item \SESoftwareConstructionTopicDevelopment
		\begin{subtopic} 
			\item Análisis de cambio impacto 
			\item Cambio de actualización 
		\end{subtopic}
		\item \SESoftwareConstructionTopicPotential
		\begin{subtopic} 
			\item Buffer y otros tipos de desbordamientos 
			\item Condiciones elemento Race 
			\item Inicialización incorrecta, incluyendo la elección de los privilegios 
			\item Entrada Comprobación 
			\item Suponiendo éxito y corrección 
			\item La validación de las hipótesis 
		\end{subtopic}
	\end{topics}
	\begin{learningoutcomes}
		\item \SESoftwareConstructionLODescribeTechniques[\Assessment]
		\item \SESoftwareConstructionLOBuild[\Assessment]
		\item \SESoftwareConstructionLODescribeSecure[\Assessment]
		\item \SESoftwareConstructionLOSelectAndDefined[\Assessment]
		\item \SESoftwareConstructionLOCompareAndStrategies[\Assessment]
		\item \SESoftwareConstructionLODescribeTheAnalyzing[\Assessment]
		\item \SESoftwareConstructionLODescribeTheAnalyzingChanges[\Assessment]
		\item \SESoftwareConstructionLORewrite[\Assessment]
		\item \SESoftwareConstructionLOWriteAThatNon[\Assessment]
	\end{learningoutcomes}
\end{unit}



\begin{coursebibliography}
\bibfile{Computing/CS/CS290}
\end{coursebibliography}

\end{syllabus}

%\end{document}
