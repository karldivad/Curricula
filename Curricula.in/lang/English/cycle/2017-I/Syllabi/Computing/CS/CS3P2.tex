\begin{syllabus}

\course{CS3P2. Cloud Computing}{Obligatorio}{CS3P2} % Common.pm

\begin{justification}
In order to understand the advanced computational techniques, the students must have a strong knowledge of the
various discrete structures, structures that will be implemented and used in the laboratory in the programming language.
\end{justification}

\begin{goals}
\item That the student is able to model computer science problems using graphs and trees related to data structures.
\item That the student apply efficient travel strategies to be able to search data in an optimal way.
\end{goals}

%% (1) familiar  (2)usar (3)evaluar
\begin{outcomes}{V1}
    \item \ShowOutcome{a}{2} % Apply computational and math knowledge appropriate to the discipline.
    \item \ShowOutcome{b}{2} % Analyze problems identify and define the appropriate computational requirements for their solution
    \item \ShowOutcome{i}{2} % Use current techniques and tools necessary for the practice of computing
    \item \ShowOutcome{j}{2} % Apply the mathematical base, principles of algorithms and the theory of the Science of the Computing in the modeling and design of computational systems in such a way that it demonstrates understanding of the points of balance involved in the chosen option.
\end{outcomes}

\begin{competences}{V1}
    \item \ShowCompetence{C2}{a} % Ability to have a critical and creative perspective to identify and solve problems using computational thinking
    \item \ShowCompetence{C4}{b}
    \item \ShowCompetence{C16}{i} % Ability to identify advanced computing topics and understanding the boundaries of the discipline.
    \item \ShowCompetence{CS2}{i} % Identify and analyze the criteria and specifications appropriate to the specific problems, and plan strategies for their solution.
    \item \ShowCompetence{CS3}{j} % Analyze the degree to which a computer-based system complies with  the criteria defined for its current use and future development.
    \item \ShowCompetence{CS6}{j} % Evaluate systems in terms of quality attributes in general and the possible advantages and disadvantages that arise in the given problem.
\end{competences}

\begin{unit}{\PDDistributedSystems}{}{coulouris}{15}{C2, C4}
\begin{topics}%
    \item \PDDistributedSystemsTopicFaults
    \item \PDDistributedSystemsTopicDistributed
    \item \PDDistributedSystemsTopicDistributedSystem
    \item \PDDistributedSystemsTopicDistributedService
    \item \PDDistributedSystemsTopicCore
\end{topics}
\begin{learningoutcomes}%
    \item \PDDistributedSystemsLODistinguishNetwork~[\Familiarity] %
    \item \PDDistributedSystemsLOExplainWhySuch~[\Familiarity] %
    \item \PDDistributedSystemsLOWriteAPerforms~[\Usage] %
    \item \PDDistributedSystemsLOMeasure~[\Usage] %
    \item \PDDistributedSystemsLOExplainWhySystem~[\Familiarity] %
    \item \PDDistributedSystemsLOImplementAForSpell~[\Usage] %
    \item \PDDistributedSystemsLOExplainTheOverhead~[\Familiarity] %
    \item \PDDistributedSystemsLODescribeTheAssociated~[\Familiarity] %
    \item \PDDistributedSystemsLOGiveExamplesFor~[\Usage] %
\end{learningoutcomes}%
\end{unit}

\begin{unit}{\PDCloudComputing}{}{dongarra, buyya}{15}{C2, C4}
\begin{topics}
    \item Overview of it Cloud Computing.
    \item History.
    \item Overview of the technologies involved.
    \item Benefits, risks and economic aspects.
    \item \PDCloudComputingTopicCloud
    \item \PDCloudComputingTopicInternet
\end{topics}
\begin{learningoutcomes}
    \item Explain the concept of Cloud Computing. [\Familiarity]
    \item List some technologies related to Cloud Computing. [\Familiarity]
    \item \PDCloudComputingLOExplainStrategies~[\Familiarity] %
    \item Discuss the advantages and disadvantages of the Cloud Computing paradigm.  [\Familiarity]
    \item Express the economic benefits as well as the characteristics and risks of the Cloud paradigm for business and cloud providers.   [\Familiarity]
    \item Differentiate between service models.   [\Usage]
\end{learningoutcomes}
\end{unit}

\begin{unit}{Centros de Procesamiento de Datos}{}{dongarra, buyya}{10}{C16}
\begin{topics}
    \item Overview of a data processing center.
    \item Design Considerations.
    \item Comparison of large data processing centers.
\end{topics}
\begin{learningoutcomes}
    \item Describe the evolution of Data Centers. [\Familiarity]
    \item Sketch the architecture in detail of the data center. [\Familiarity]
    \item Indicate design considerations and discuss their impact.  [\Familiarity]
\end{learningoutcomes}
\end{unit}

\begin{unit}{\PDCloudComputing}{}{dongarra, buyya}{20}{CS2, CS3}
\begin{topics}
    \item \PDCloudComputingTopicVirtualization
    \item Security, resources, and failures isolation .
    \item Storage as a Service.
    \item Elasticity.
    \item Xen y WMware.
    \item Amazon EC2.
\end{topics}
\begin{learningoutcomes}
    \item \PDCloudComputingTopicVirtualization. [\Familiarity]
    \item \PDCloudComputingLOExplainTheDisadvantages. [\Familiarity]
    \item Identify the reasons why virtualization is becoming enormously useful, especially in the cloud. [\Familiarity]
    \item Explain different types of isolation such as failure, resources and security provided by virtualization and used by the cloud. [\Familiarity]
    \item Explain the complexity that management can have in terms of abstraction levels and well-defined interfaces and their applicability for virtualization in the cloud.  [\Familiarity]
    \item Define Virtualization and Identify Different Types of Virtual Machines. [\Familiarity]
    \item Identify CPU virtualization conditions, recognize the difference between full virtualization and  paravirtualization,explain emulation as a major technique for CPU virtualization and examine virtual CPU planning in Xen. [\Familiarity]
    \item Sketching the difference between the classic OS virtual memory and memory virtualization.Explain multiple levels of page mapping as opposed to memory virtualization. Define over-commitment memory and illustrate VMware memory ballooning as a claiming technique for virtualized systems with over-committed memory. [\Familiarity]
\end{learningoutcomes}
\end{unit}

\begin{unit}{\PDCloudComputing}{}{dongarra, buyya}{12}{CS2, CS3}
\begin{topics}
    \item \PDCloudComputingTopicCloudBased
    \item Overview of Storage Technologies.
    \item Fundamentals concepts of cloud storage.
    \item Amazon S3 y EBS.
    \item Distributed File System.
    \item Database System  NoSQL.
\end{topics}
\begin{learningoutcomes}
    \item Describe the general organization of data and storage. [\Familiarity]
    \item Identify the problems of scalability and administration of the big data. Discuss several abstractions in storage. [\Familiarity]
    \item Compare and contrast different types of file system. Compare and contrast the Hadoop Distributed File System (HDFS) and the Virtual Parallel File System (PVFS).  [\Usage]
    \item Compare and contrast different types of databases. Discuss the advantages and disadvantages of NoSQL databases. [\Usage]
    \item Discuss storage concepts in the cloud. [\Familiarity]
\end{learningoutcomes}
\end{unit}

\begin{unit}{Modelos de Programación}{}{dongarra, buyya, graphlab, pregel, giraph}{12}{CS6}
\begin{topics}
    \item Overview of cloud computing-based programming models.
    \item Programming Model MapReduce.
    \item Programming model for graph-based applications.
\end{topics}
\begin{learningoutcomes}
    \item Explain the fundamental aspects of parallel and distributed programming models. [\Familiarity]
    \item Differences between programming models: MapReduce, Pregel, GraphLab and Giraph.. [\Usage]
    \item Explain the main concepts in the MapReduce programming model.  [\Usage]
\end{learningoutcomes}
\end{unit}

\begin{coursebibliography}
\bibfile{Computing/CS/CS3P2}
\end{coursebibliography}

\end{syllabus}
