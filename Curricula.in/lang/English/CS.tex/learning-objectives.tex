\section{Objetivos de Aprendizaje ({\it Learning Outcomes})}
Cada KU dentro de un KA enumera tanto un conjunto de temas y los resultados de aprendizaje ({\it Learning Outcomes})
que los estudiantes deben alcanzar en lo que respecta a los temas especificados. 
Resultados de aprendizaje no son de igual tama�o y no tienen una asignaci�n uniforme de horas curriculares; 
temas con el mismo n�mero de horas pueden tener muy diferentes n�meros de los resultados del aprendizaje asociados.

Cada resultado de aprendizaje tiene un nivel asociado de dominio. 
En la definici�n de los diferentes niveles que dibujamos de otros enfoques curriculares, 
especialmente la taxonom�a de Bloom, que ha sido bien explorado dentro de la ciencia de la computaci�n. 
En este documento no se aplic� directamente los niveles de Bloom en parte porque varios de 
ellos son impulsados por contexto pedag�gico, que introducir�a demasiada pluralidad en un documento de este tipo; 
en parte porque tenemos la intenci�n de los niveles de dominio para ser indicativa y 
no imponer restricci�n te�rica sobre los usuarios de este documento.

Nosotros usamos tres niveles de dominio esperados que son:
\begin{description}
 \item [Nivel 1 \Familiarity ({\it Familiarity})]: \LearningOutcomesTxtEsFamiliarity
 \item [Nivel 2 \Usage ({\it Usage})]: \LearningOutcomesTxtEsUsage
 \item [Nivel 3 \Assessment ({\it Assessment})]: \LearningOutcomesTxtEsAssessment
\end{description}

% As a concrete, although admittedly simplistic, example of these levels of mastery, we consider
% the notion of iteration in software development, for example for-loops, while-loops, and iterators.
% At the level of ``Familiarity,'' a student would be expected to have a definition of the concept of
% iteration in software development and know why it is a useful technique. In order to show
% mastery at the ``Usage'' level, a student should be able to write a program properly using a form
% of iteration. Understanding iteration at the ``Assessment'' level would require a student to
% understand multiple methods for iteration and be able to appropriately select among them for
% different applications.

Por ejemplo, para evaluar los niveles de dominio, consideremos la noci�n de iteraci�n en el desarrollo de software (for, while e iteradores). 
En el plano de la ``familiaridad'', se espera que un estudiante tenga una definici�n del 
concepto de iteraci�n en el desarrollo de software y saber por qu� esta t�cnica es �til. 

Con el fin de mostrar el dominio del nivel ``Uso'', el estudiante debe ser capaz de 
escribir un programa adecuadamente usando una forma de iteraci�n. 

En el nivel de ``Evaluaci�n'', en la iteraci�n se requerir�a que un estudiante comprenda m�ltiples m�todos de iteraci�n y 
que sea capaz de seleccionar apropiadamente entre ellos para diferentes aplicaciones.

