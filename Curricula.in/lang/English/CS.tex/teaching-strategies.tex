\section{Estrategias de Ense�anza-Aprendizaje}
El avance progresivo de las tecnolog�as y redes m�viles, habilita nuevas herramientas para ser exploradas y 
experimentadas en escenarios de aprendizaje. Es as�, como el uso de dispositivos m�viles, a�ade nuevas 
dimensiones al proceso de ense�anza aprendizaje, como la movilidad y personalizaci�n. La evoluci�n de los 
componentes de E-Learning (aprendizaje soportado por medios electr�nicos) para el aprendizaje m�vil y aprendizaje ubicuo, 
abre el espacio conceptual y t�cnico para el desarrollo de la Internet de los objetos (IoT), de sus siglas en ingl�s de {\it Internet of Things}, en el aprendizaje.

La tendencia generalizada de parte de los estudiantes a aceptar positivamente las herramientas que 
impliquen una novedad en el proceso de ense�anza-aprendizaje, nos indica que la inclusi�n de elementos nuevos, es favorable para el �nimo frente al aprendizaje.

Teniendo en cuenta que las actividades intensivas en cuanto al uso de dispositivos especializados, 
deben ser cuidadosamente dise�adas, sobre todo en cuanto a imprudencias o excesos de parte de los usuarios, 
lo cual puede llevar a fallas y posibles p�rdidas de informaci�n, entonces podemos hacer uso de IoT 
para hacer m�s agradable el proceso de aprendizaje.

Por lo que, consideramos implementar los siguientes conceptos estrat�gicos:

\begin{itemize}
\item Horizontalidad educativa.
\item Material audiovisual, aprendamos con documentales y con todo lo que nos da hoy la tecnolog�a.
\item interactividad grupal.
\item Motivaci�n de la curiosidad.
\item Fomento de la creatividad.
\item Pensamiento lateral.
\item Uso adecuado de las tecnolog�as.
\item Autoaprendizaje guiado.
\item Estimulaci�n de la inteligencia en lugar de estimular la memoria.
\item Diversi�n incorporada, mientras aprendemos � ense�amos.
\end{itemize}