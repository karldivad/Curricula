\begin{BKL2}{IT1}{IT1. Arquitectura de Computadores}
Horas:

(begin_topics)
-Representación de datos fundamentales: no numéricos, numéricos (enteros, reales, errores, precisión)
(begin_subtopics)
-Representación básica en máquina de datos numéricos
-Representación básica en máquina de datos no numéricos
-Precisión de la representación de números enteros y de punto flotante
-Errores en aritmética de computador y asuntos relacionados de portabilidad
-Conceptos básicos de arquitectura de computadores
(end_subtopics)
(begin_goals)
(end_goals)
(end_topics)

(begin_topics)
-Representación física de información digital: p.e. datos, texto, imágenes, voz, video.
(begin_subtopics)
(end_subtopics)
(begin_goals)
(end_goals)
(end_topics)

(begin_topics)
-Architecturas de CPU: CPU, memoria, registros, modos de direccionamiento, conjuntos de instrucciones.
(begin_subtopics)
-Organización básica; von Neumann, diagrama de bloques, bus de datos, bus de control.
-Instrucciones y modos de direccionamiento: conjuntos de instrucciones y tipos.
-Instrucciones y modos de direccionamiento: lenguaje de máquina.
-Modos de direccionamiento.
-Unidad de control; {\it fetch} y ejecución de instrucciones, {\it fetch} de operadores.
-CISC, RISC.
-Organización del computador.
-Sistemas de memoria.
(end_subtopics)
(begin_goals)
(end_goals)
(end_topics)

(begin_topics)
-Componentes de sistemas de computadores: bus, controladores, sistemas de almacenamiento, dispositivos periféricos.
(begin_subtopics)
-Periféricos: E/S e interrupciones.
-Periféricos: métodos de control de E/S, interrupciones.
-Periféricos: almacenamiento externo, organización física y {\it drives}.
-Almacenamiento auxiliar, cinta, óptico.
-Sistemas de almacenamiento, jerarquía.
-Organización principal de la memoria, operaciones de bus, tiempos de ciclo.
-Memoria caché, lectura/escritura.
-Memoria virtual.
-Interfaces entre computadores y otros dispositivos (sensores, efectuadores, etc.)
(end_subtopics)
(begin_goals)
(end_goals)
(end_topics)

(begin_topics)
-Arquitectura de multiprocesadores
(begin_subtopics)
-Arquitecturas de sistemas (multi-procesamiento simple y procesamiento distribuido, pilas)
-Tecnologías cliente-servidor.
(end_subtopics)
(begin_goals)
(end_goals)
(end_topics)

(begin_topics)
-Lógica y sistemas discretos.
(begin_subtopics)
-Elementos lógicos y teoría de {\it switching}; conceptos e implementación de minimización.
-Demoras y peligros de propagación.
-Demultiplexers, multiplexers, decodificadores, codificadores, aditores, substractores.
-ROM, PROM, EPROM, EAPROM, RAM.
-Análisis y síntesis de circuitos síncronos, asíncronos vs. síncronos.
-Notación de transferencia de registros, condicional e incondicional.
-Máquinas de estado, redes de conducción, señales de transferencia de carga.
-Tri-estados y estructuras de bus.
-Diagramas de bloque, diagramas de tiempo, lenguaje de transferencia.
(end_subtopics)
(begin_goals)
(end_goals)
(end_topics)
\end{BKL2}

\begin{BKL2}{IT2}{IT2. Algoritmos y Estructuras de Datos}
Horas:

(begin_topics)
-Problemas formales y resolución de problemas.
(begin_subtopics)
-Estrategias de resolución de problemas usando algoritmos voraces.
-Estrategias de resolución de problemas usando algoritmos divide y vencerás.
-Estrategias de resolución de problemas usando algoritmos de {\it back- tracking}.
-Proceso de diseño de software; desde la especificación a la implementación.
-Establecimiento del problema y determinación algorítmica.
-Estrategias de implementación ({\it top-down}, {\it bottom-up}; equipo vs. individual).
-Conceptos de verificación formal.
-Models de computación formales.
(end_subtopics)
(begin_goals)
(end_goals)
(end_topics)

(begin_topics)
-Estructuras de datos básicas: listas, arreglos, cadenas, registros, conjuntos, listas enlazadas, pilas, colas, árboles.
(begin_subtopics)
(end_subtopics)
(begin_goals)
(end_goals)
(end_topics)

(begin_topics)
-Estructuras de datos complejas: p.e., de datos, texto, voz, imagen, video, hipermedia.
(begin_subtopics)
(end_subtopics)
(begin_goals)
(end_goals)
(end_topics)

(begin_topics)
-Tipos abstractos de datos.
(begin_subtopics)
-Propósito e implementación de tipos abstractos de datos.
-Especificaciones informales.
-Especificaciones formales, pre-condiciones y post-condiciones, algebraicas.
-Módulos, cohesión, acoplamiento; diagramas de flujo de datos y conversión a jerarquías.
-Correctitud, verificación y validación: pre- y post-condiciones, invariantes.
-Estructuras de control; selección, iteración, recursión; tipos de datos y sus usos.
(end_subtopics)
(begin_goals)
(end_goals)
(end_topics)

(begin_topics)
-Estructuras de archivos: secuencial, de acceso directo, hashing, indexados.
(begin_subtopics)
-Archivos (estructura, métodos de acceso): distribución de archivos, conceptos de archivos fundamentales.
-Archivos (estructura, métodos de acceso): contenidos y estructuras de directorios, nombramiento.
-Archivos (estructura, métodos de acceso): vista general de seguridad del sistema, métodos de seguridad.
(end_subtopics)
(begin_goals)
(end_goals)
(end_topics)

(begin_topics)
-Estructuras y algoritmos de ordenamiento y búsqueda.
(begin_subtopics)
-Algoritmos de ordenamiento ({\it shell sort}, {\it bucket sort}, {\it radix sort}, {\it quick sort}), edición.
-Algoritmos de búsqueda (búsqueda serial, búsqueda binaria y árboles de búsqueda binaria).
-Búsqueda, hashing, resolución de colisiones.
(end_subtopics)
(begin_goals)
(end_goals)
(end_topics)

(begin_topics)
-Eficiencia de algoritmos, complejidad y métricas.
(begin_subtopics)
-Análisis asintótico.
-Balance entre tiempo y espacio en algoritmos.
-Clases de complejidad P, NP, P-space; problemas tratables e intratables.
-Análisis de límite inferior.
-NP-completitud.
-Algoritmos de ordenamiento $O(n^2)$.
-Algoritmos de ordenamiento $O(n\log n)$.
-{\it Backtracking}, {\it parsing}, simulaciones discretas, etc.
-Fundamentos de análisis de algoritmos.
(end_subtopics)
(begin_goals)
(end_goals)
(end_topics)

(begin_topics)
-Algoritmos recursivos.
(begin_subtopics)
-Conexión de algoritmos recursivos con inducción matemática.
-Comparación de algoritmos iterativos y recursivos.
(end_subtopics)
(begin_goals)
(end_goals)
(end_topics)

(begin_topics)
-Redes neuronales y algoritmos genéticos.
(begin_subtopics)
(end_subtopics)
(begin_goals)
(end_goals)
(end_topics)

(begin_topics)
-Consideraciones avanzadas.
(begin_subtopics)
-Funciones computables: modelos de funciones computables de máquinas de Turing.
-Problemas de decisión: problemas enumerables recursivos; indecibilidad.
-Modelo de arquitecturas paralelas.
-Algoritmos de arquitecturas paralelas.
-Problemas matemáticos: problemas bien acondicionados y mal acondicionados.
-Problemas matemáticos: aproximaciones iterativas a problemas matemáticos.
-Problemas matemáticos: clasificación de error, computacional, representacional.
-Problemas matemáticos: aplicaciones de métodos de aproximación interativa.
-Límites de computación: computabilidad e intratabilidad algorítmica.
(end_subtopics)
(begin_goals)
(end_goals)
(end_topics)
\end{BKL2}

\begin{BKL2}{IT3}{IT3. Lenguajes de Programación}
Horas:
 
(begin_topics)
-Estructuras de lenguajes de programación fundamentales; comparación de lenguajes y aplicaciones.
(begin_subtopics)
(end_subtopics)
(begin_goals)
(end_goals)
(end_topics)

 
(begin_topics)
-Lenguajes de nivel de máquina y ensamblador.
(begin_subtopics)
(end_subtopics)
(begin_goals)
(end_goals)
(end_topics)

 
(begin_topics)
-Lenguajes procedurales.
(begin_subtopics)
-Ventajas y desventajas de la programación procedural.
-Declaraciones básicas de tipos; operaciones aritméticas y asignación; condicionales.
-Procedimientos, funciones y parámetros; arreglos y registros.
(end_subtopics)
(begin_goals)
(end_goals)
(end_topics)

 
(begin_topics)
-Lenguajes no procedurales: lógicos, funcional y basados en eventos.
(begin_subtopics)
(end_subtopics)
(begin_goals)
(end_goals)
(end_topics)

 
(begin_topics)
-Lenguajes de cuarta generación.
(begin_subtopics)
(end_subtopics)
(begin_goals)
(end_goals)
(end_topics)

 

(begin_topics)
-Extensiones orientadas a objetos para lenguajes.
(begin_subtopics)
(end_subtopics)
(begin_goals)
(end_goals)
(end_topics)

 

(begin_topics)

-Lenguajes de programación, diseño, implementación y comparación.
(begin_subtopics)

-Historia de los primeros lenguajes.

-Evolución de los lenguajes procedurales.

-Evolución de los lenguajes no procedurales.

-Computadores virtuales.

-Tipos de datos elementales y estructurados.

-Creación y aplicación de tipos de datos definidos por el usuario.

-Expresiones, orden de evaluación y efectos secundarios.

-Subproigramas y corutinas como abstracciones de expresiones y declaraciones.

-Manejo de excepciones.

-Mecanismos para compartir y restringir el acceso a datos.

-Ámbito estático vs. dinámico, timepo de vida, visibilidad.

-Mecanismos de paso de parámetros; referencia, valor, nombre, resultado, etc.

-Variedades de disciplinas de prueba de tipos y sus mecanismos.

-Aplicación de almacenamiento basado en pilas.

-Aplicación de almacenamiento basado en {\it heaps}.

-Autómatas de estado finito para modelos restringidos de computación.

-Aplicación de expresiones regulares al análisis de lenguajes de programación.

-Uso de gramáticas libres de contexto y de autómatas de pila.

-Equivalencia entre gramáticas libres de contexto y autómatas de pila.

-Uso de autómatas de pila en el {\it parsing} de lenguajes de programación.

-Proceso de traducción de lenguajes, compiladores a interpretadores.

-Semántica de los lenguajes de programación.

-Paradigmas y lenguajes de programación funcional.

-Construcciones programación paralela.

-Lenguajes procedurales: problemas de implementación; mejora del rendimiento.

-Compiladores y traductores.

-Lenguajes de muy alto nivel: SQL, lenguajes de cuarta generación.

-Diseño orientado a objetos, lenguajes y programación.

-Lenguajes de programación lógica: LISP, PROLOG; programación orientada a lógica.

-Generadores de código.

-Shells de sistemas expertos.

-Lenguajes de diseño de software.

(end_subtopics)
(begin_goals)
(end_goals)
(end_topics)

\end{BKL2}
 

\begin{BKL2}{IT4}{IT4. Sistemas Operativos}
Horas:
 
(begin_topics)

-Arquitectura, objetivos y estructura de un sistema operativo; métodos de estructuración, modelos por capas.

(begin_subtopics)
(end_subtopics)
(begin_goals)
(end_goals)
(end_topics)

 

(begin_topics)

-Interacción del sistema operativo con la arquitectura de hardware.

(begin_subtopics)
(end_subtopics)
(begin_goals)
(end_goals)
(end_topics)

 

(begin_topics)

-Administración de procesos: procesos concurrentes, sincronización.

(begin_subtopics)

-Tareas, procesos, interrupciones.

-Estructuras, listas de espera, bloques de control de procesos.

-Ejecución concurrente de procesos.

-Acceso compartido, condiciones de ejecución.

-{\it Deadlock}; causas, condiciones, prevención.

-Modelos y mecanismos (p.e., {\it busy waiting}, {\it spin locks}, algoritmo de Deker).

-{\it Switching} preferente y no preferente.

-{\it Schedulers} y políticas de {\it scheduling}.
(end_subtopics)
(begin_goals)
(end_goals)
(end_topics)

 

(begin_topics)
-Administración de memoria.
(begin_subtopics)

-Memoria física y registros.

-{\it Overlays}, {\it swapping}, particiones.

-Páginas y segmentos.

-Política de posicionamiento y reposicionamiento.

-{\it Thrashing}, {\it working sets}.

-Listas libres, {\it layout}; servidores, interrupciones; recuperación de fallos.

-Protección de memoria, administración de la recuperación.

(end_subtopics)
(begin_goals)
(end_goals)
(end_topics)

 

(begin_topics)

-Asignación y programación de recursos.

(begin_subtopics)

-{\it Suites} de protocolos (comuniación y redes); {\it streams} y datagramas.

-Internet {\it working} y {\it routing}; servidores y servicios.

-Tipos de sistemas operativos: de usuario simple, multi-usuario, de red.

-Sincronización y temporización en sistemas distribuidos y de tiempo real.

-Consideraciones especiales en sistemas de tiempo real; fallas, riesgos y recuperación.

-Utilidades de sistemas operativos.

-Evolución del hardware; fuerzas y restricciones económicas.

-Arquitectura de los sistemas de tiempo real y sistemas empotrados.

-Consideraciones especiales en sistemas de tiempo real empotrados: requerimientos {\it hard-timing}..

(end_subtopics)
(begin_goals)
(end_goals)
(end_topics)

 

(begin_topics)

-Administración de almacenamiento secundario.
(begin_subtopics)
(end_subtopics)
(begin_goals)
(end_goals)
(end_topics)

 

(begin_topics)

-Sistemas de archivos y de directorios.
(begin_subtopics)
(end_subtopics)
(begin_goals)
(end_goals)
(end_topics)

 

(begin_topics)

-Protección y seguridad.
(begin_subtopics)
(end_subtopics)
(begin_goals)
(end_goals)
(end_topics)

 

(begin_topics)

-Sistemas operativos distribuidos.
(begin_subtopics)
(end_subtopics)
(begin_goals)
(end_goals)
(end_topics)

 

(begin_topics)
-Soporte del sistema operativo para interacción humano-computador: p.e., GUI, video interactivo.
(begin_subtopics)
(end_subtopics)
(begin_goals)
(end_goals)
(end_topics)

 

(begin_topics)
-Interoperatividad y compatibilidad de sistemas operativos: p.e., sistemas abiertos.
(begin_subtopics)
(end_subtopics)
(begin_goals)
(end_goals)
(end_topics)

 

(begin_topics)

-Utilidades de los sistemas operativos, herramientas, comandos y programación {\it shell}.
(begin_subtopics)
(end_subtopics)
(begin_goals)
(end_goals)
(end_topics)

 

(begin_topics)

-Administración y gerenciamiento de sistemas.
(begin_subtopics)

-{\it Bootstrapping} del sistema/carga inicial de programa.

-Generación del sistema.

-Configuración del sistema.

-Análisis, evaluación y monitoreo de rendimiento.

-Optimización y {\it tuning} del sistema.

-Funciones de administración del sistema: copias de seguridad, securidad y protección.

(end_subtopics)
(begin_goals)
(end_goals)
(end_topics)

\end{BKL2}
 


\begin{BKL2}{IT5}{IT5. Telecomunicaciones}
Horas:
 
(begin_topics)

-Estándares, modelos y tendencias internacionales en telecomunicaciones.
(begin_subtopics)

-Redes de computadoras y control: topologías, portadores comunes, equipos.

-Diseño y administración de redes: arquitecturas de red (ISO, SNA, DNA).

(end_subtopics)
(begin_goals)
(end_goals)
(end_topics)

 

(begin_topics)

-Transmisión de datos: media, técnicas de señalización, impedimento de transmisión, codificación, error.

(begin_subtopics)

-Tecnologías de sistemas de comunicaciones: medios de transmisión, analógico-digital.
(end_subtopics)
(begin_goals)
(end_goals)
(end_topics)

 

(begin_topics)

-Configuración de línea: control de rror, control de flujo, multiplexado.
(begin_subtopics)
(end_subtopics)
(begin_goals)
(end_goals)
(end_topics)

 

(begin_topics)

-Redes de área local.
(begin_subtopics)

-Topologías, control de acceso al medio, multiplexado.

-Redes de área local y WANs: topología, {\it gateways}, usos (funciones y oficina).

-Determinación de requerimientos, monitoreo y control del rendimiento, aspectos económicos.

-Arquitectura de los sistemas distribuidos.

-Aspectos de hardware de los sitemas distribuidos.

(end_subtopics)
(begin_goals)
(end_goals)
(end_topics)


(begin_topics)
-Redes de área amplia: técnicas de {\it switching}, de {\it broadcast}, {\it routing}.
(begin_subtopics)
(end_subtopics)
(begin_goals)
(end_goals)
(end_topics)

 

(begin_topics)
-Arquitecturas y protocolos de redes.
(begin_subtopics)
(end_subtopics)
(begin_goals)
(end_goals)
(end_topics)

 

(begin_topics)
-{\it Internetworking}
(begin_subtopics)
(end_subtopics)
(begin_goals)
(end_goals)
(end_topics)

 

(begin_topics)
-Configuración de redes, análisis y monitoreo de rendimiento.
(begin_subtopics)
(end_subtopics)
(begin_goals)
(end_goals)
(end_topics)

 

(begin_topics)

-Seguridad de redes: encriptación, firmas digitales, autenticación.

(begin_subtopics)
(end_subtopics)
(begin_goals)
(end_goals)
(end_topics)

 

(begin_topics)

-Redes de alta velocidad: p.e. ISDN, SMDS, ATM, FDDI de banda ancha.

(begin_subtopics)
(end_subtopics)
(begin_goals)
(end_goals)
(end_topics)

 

(begin_topics)

-Tecnologías emergentes: ATM, ISDN, redes de satélites, redes ópticas, etc., voz, datos y videos integrados.

(begin_subtopics)
(end_subtopics)
(begin_goals)
(end_goals)
(end_topics)

 

(begin_topics)

-Aplicación: p.e., cliente-servidor, EDI, EFT, redes de teléfonos, e-mail, multimedia, video.
(begin_subtopics)

-Métodos y transmisión de información gráfica y de video.

(end_subtopics)
(begin_goals)
(end_goals)
(end_topics)

 

(begin_topics)

-Protocolos de sistemas abiertos.
(begin_subtopics)

-Protocolos de transporte.

-Protocolos de soporte de aplicaciones: encriptación; compromiso, concurrencia.
(end_subtopics)
(begin_goals)
(end_goals)
(end_topics)

 

(begin_topics)

-Distribución de información.
(begin_subtopics)

-Structura de redes.

-Tecnología cliente-servidor/cliente-servidor delgada.

-Redes, {\it routing}, análisis de desempeño.

-Sistemas de comunicaciones.
(end_subtopics)
(begin_goals)
(end_goals)
(end_topics)

\end{BKL2}
 

\begin{BKL2}{IT6}{IT6. Bases de Datos}

Horas:
 
(begin_topics)
-DBMS: características, funciones y arquitectura.
(begin_subtopics)
-DBMS (características, funciones, arquitectura); componentes de un sistema de bases de datos.

-DBMS: vista general de álgebra relacional.

-Diseño lógico (diseño independiente de DBMS): ER, orientado a objetos.

(end_subtopics)
(begin_goals)
(end_goals)
(end_topics)

 

(begin_topics)

-Modelos de datos: relacional, jerárquico, de red, de objetos, de objetos semánticos.
(begin_subtopics)

-Teminología de modelado de datos relacional.

-Modelado conceptual (p.e., entidad-relación, orientado a objtos).

-Interconversión entre tipos de modelos (p.e. jerárquico a relacional).
(end_subtopics)
(begin_goals)
(end_goals)
(end_topics)

 

(begin_topics)
-Normalización
(begin_subtopics)
(end_subtopics)
(begin_goals)
(end_goals)
(end_topics)

 

(begin_topics)

-Integridad (referencial, de item de datos, de intra-relaciones): representación de relaciones; de entidad y referencial.
(begin_subtopics)
(end_subtopics)
(begin_goals)
(end_goals)
(end_topics)

 

(begin_topics)

-Lenguajes de definición de datos (lenguajes de definición de esquemas, herramientas de desarrollo gráficas).
(begin_subtopics)
(end_subtopics)
(begin_goals)
(end_goals)
(end_topics)

 

(begin_topics)

-Interfaz de aplicaciones.
(begin_subtopics)

-Función soportada por sistemas de bases de datos típicos; métodos de acceso, seguridad.

-DML, consulta, QBE, SQL, etc.: lenguaje de consultas de bases de datos; definitción de datos, consulta.

-Interfaces de aplicación y de usuario (DML, consulta, QBE, SQL).

-Objetos de pantalla basados en eventos (botones, listas, etc.)

-Procesamiento físico de transacciones; consideraciones cliente-servidor.

-Distribución de consideraciones de procesamiento del cliente y del servidor.
(end_subtopics)
(begin_goals)
(end_goals)
(end_topics)

 

(begin_topics)

-Procesadores inteligentes de consultas y organización de consultas, herramientas OLAP.
(begin_subtopics)
(end_subtopics)
(begin_goals)
(end_goals)
(end_topics)

 

(begin_topics)

-Bases de datos distribuidas, repositorios y {\it warehouses}
(begin_subtopics)
(end_subtopics)
(begin_goals)
(end_goals)
(end_topics)

 

(begin_topics)
-Productos DBMS: desarrollos recientes en sistemas de bases de datos (p.e., hipertexto, hipermedia, ópticos).
(begin_subtopics)
(end_subtopics)
(begin_goals)
(end_goals)
(end_topics)

 

(begin_topics)

-Máquinas y servidores de bases de datos.
(begin_subtopics)
(end_subtopics)
(begin_goals)
(end_goals)
(end_topics)

 

(begin_topics)

-Administración de datos y bases de datos.
(begin_subtopics)

-Administración de datos.

-Administración de bases de datos: impacto social de las bases de datos; seguridad y privacidad.

-Propiedad y controles de acceso para sistema de datos y de aplicaciones.

-Modelos de acceso basados en roles y capacidades.

-Replicación.

-Planeamiento de la capacidad del sistema.

-Planeamiento y administración de redundancia, seguridad y copias de seguridad.

(end_subtopics)
(begin_goals)
(end_goals)
(end_topics)

 

(begin_topics)

-Diccionarios, enciclopedias y repositorios de datos.
(begin_subtopics)
(end_subtopics)
(begin_goals)
(end_goals)
(end_topics)

 

(begin_topics)

-Recuperación de información: p.e. herramientas de Internet, procesamiento de imágenes, hipermedia.
(begin_subtopics)
(end_subtopics)
(begin_goals)
(end_goals)
(end_topics)

\end{BKL2}



\begin{BKL2}{IT7}{IT7. Inteligencia Artificial}
Horas:
 
(begin_topics)

-Representación del conocimiento.
(begin_subtopics)

-Historia, contexto y límites de la inteligencia artificial; el test de Turing.
(end_subtopics)
(begin_goals)
(end_goals)
(end_topics)

 

(begin_topics)

-Ingeniería del conocimiento.
(begin_subtopics)
(end_subtopics)
(begin_goals)
(end_goals)
(end_topics)

 

(begin_topics)

-Proceso de inferencia.
(begin_subtopics)

-Estrategias de control básicas (p.e., por profundidad y por amplitud).

-Razonamiento hacia adelante y hacia atrás.

-Búsqueda heurística.

-Sistemas expertos.
(end_subtopics)
(begin_goals)
(end_goals)
(end_topics)

 

(begin_topics)

-Otras técnicas: lógica difusa, razonamiento basado en casos, lenguaje natural y reconocimiento del habla.
(begin_subtopics)
(end_subtopics)
(begin_goals)
(end_goals)
(end_topics)

 

(begin_topics)

-Sistemas basados en conocimiento.
(begin_subtopics)

-Lenguaje natural, habla y visión.

-Reconocimiento de patrones.

-Aprendizaje de máquina.

-Robótica.

-Redes neuronales.

(end_subtopics)
(begin_goals)
(end_goals)
(end_topics)

\end{BKL2}



\begin{BKL2}{OMC1}{OMC1. Teoría General de Organizaciones}
Horas:
 
(begin_topics)

-Modelos organizacionales jerárquicos y de flujo.
(begin_subtopics)
(end_subtopics)
(begin_goals)
(end_goals)
(end_topics)

 

(begin_topics)

-Grupos de trabajo organizacionales.
(begin_subtopics)
(end_subtopics)
(begin_goals)
(end_goals)
(end_topics)

 

(begin_topics)

-Envergadura organizacional: usuario simple, grupo de trabajo, equipo, empresa, global.

(begin_subtopics)
(end_subtopics)
(begin_goals)
(end_goals)
(end_topics)

 

(begin_topics)

-Rol de Sistemas de Información dentro de la empresa: estratégico, táctico y operativo.
(begin_subtopics)
(end_subtopics)
(begin_goals)
(end_goals)
(end_topics)

 

(begin_topics)

-Efecto de Sistemas de Información en la estructura organizacional; Sistemas de Información y mejora continua.

(begin_subtopics)
(end_subtopics)
(begin_goals)
(end_goals)
(end_topics)

 

(begin_topics)

-Estructura organizacional: centralizada, descentralizada, matriz.

(begin_subtopics)
(end_subtopics)
(begin_goals)
(end_goals)
(end_topics)

 

(begin_topics)

-Aspectos organizacionales para el uso de sistemas de software en organizaciones.
(begin_subtopics)
(end_subtopics)
(begin_goals)
(end_goals)
(end_topics)

(begin_topics)
-Procesos en la organización.
(begin_subtopics)
-Vista estratégica de los procesos organizacionales; conceptos de eficiencia y efectividad organizacional.
-Integración de las áreas funcionales de la organización.
-Procesos relacionados a los objetivos financieros, usuario final y orientados al producto.
-Innovación de procesos: análisis, modelado y simulación. 
(end_subtopics)
(begin_goals)
(end_goals)
(end_topics)

(begin_topics)
-Modelado y simulación de procesos de negocio.
(begin_subtopics)
-Automatización del proceso de negocios.
-Utilización de diagramas de actividad y de la Notación de Modelado de Procesos de Negocio (\emph{Business Process Modeling Notation} - BPMN - ) para el modelado del proceso de negocio.
-Herramientas para el modelado del proceso de negocio.
-Rediseño de tareas; impacto de la automatización en las prácticas de trabajo.
-Alcanzando seguridad y conformidad del proceso.
-Monitoreo y control de procesos.
-Administración de la cadena de abastecimiento (\emph{Supply Chain Management} - SCM). 
-Administración de la relación con los clientes (\emph{Customer Relationship Management} - CRM).
-Sistemas de gerenciamiento empresarial (\emph{Enterprise Management Systems} - ERP).
-El proceso continuo: de procesos estructurados a no estructurados.
(end_subtopics)
(begin_goals)
(end_goals)
(end_topics)

(begin_topics)
-Una vista integral de la firma y su relación con proveedores y clientes.
(begin_subtopics)
(end_subtopics)
(begin_goals)
(end_goals)
(end_topics)

\end{BKL2}



\begin{BKL2}{OMC2}{OMC2. Gerenciamiento de Sistemas de Información}
Horas:
 
(begin_topics)

-Planeamiento de Sistemas de Información.
(begin_subtopics)

-Alineamiento del planeamiento de Sistemas de Información con el planeamiento empresarial.

-Planeamiento estratégico de Sistemas de Información.

-Paneamiento de corto alcance de Sistemas de Información.

-Reingeniería.

-Mejora continua.
(end_subtopics)
(begin_goals)
(end_goals)
(end_topics)

 

(begin_topics)
-Control de la función de Sistemas de Información: p.e., auditoría, {\it outsourcing}.
(begin_subtopics)
(end_subtopics)
(begin_goals)
(end_goals)
(end_topics)

 

(begin_topics)

-Administración y contratación de recursos humanos.
(begin_subtopics)

-Planeamiento de habilidades.

-Administración del rendimiento del personal.

-Educación y entrenamiento.

-Estructuras de competencia, cooperación y premios.

(end_subtopics)
(begin_goals)
(end_goals)
(end_topics)

 

(begin_topics)
-Estructuras funcionales de Sistemas de Información -- internas vs. {\it outsourcing}.
(begin_subtopics)
(end_subtopics)
(begin_goals)
(end_goals)
(end_topics)

 

(begin_topics)

-Determinación de las metas y objetivos de la organización de Sistemas de Información.
(begin_subtopics)
(end_subtopics)
(begin_goals)
(end_goals)
(end_topics)

 

(begin_topics)

-Administración de Sistema de Información como un negocio: p.e., definición del cliente, definición de la misión de Sistemas de Información, aspectos críticos del éxito de Sitemas de Información.
(begin_subtopics)
(end_subtopics)
(begin_goals)
(end_goals)
(end_topics)

 

(begin_topics)
-Oficial de Información en Jefe (\emph{Chief Information Officer} - CIO) y funciones del personal.
(begin_subtopics)
(end_subtopics)
(begin_goals)
(end_goals)
(end_topics)

 

(begin_topics)
-Sistemas de Información como una función de servicio: evaluación del desempeño -- externo e interno, {\it marketing} de servicios.
(begin_subtopics)
(end_subtopics)
(begin_goals)
(end_goals)
(end_topics)

 

(begin_topics)

-Administración financiera de Sistemas de Información.
(begin_subtopics)
(end_subtopics)
(begin_goals)
(end_goals)
(end_topics)

 

(begin_topics)

-Uso estratégico de Sistemas de Información: p.e., ventajas competitivas y Sistemas de Información, proceso de reingeniería, Sistemas de Información y calidad.

(begin_subtopics)
(end_subtopics)
(begin_goals)
(end_goals)
(end_topics)

 

(begin_topics)
-Trabajo del conocimiento, computación de usuario final: soporte, roles, productividad y actividades.
(begin_subtopics)
(end_subtopics)
(begin_goals)
(end_goals)
(end_topics)

 

(begin_topics)

-Política de Sistemas de Información y formulación y comunicación de procesos operativos.
(begin_subtopics)
(end_subtopics)
(begin_goals)
(end_goals)
(end_topics)

 

(begin_topics)

-Copias de seguridad, planeamiento y recuperación de desastres.

(begin_subtopics)
(end_subtopics)
(begin_goals)
(end_goals)
(end_topics)

 

(begin_topics)
-Administración de tecnologías emergentes.
(begin_subtopics)
(end_subtopics)
(begin_goals)
(end_goals)
(end_topics)

 

(begin_topics)
-Administración de sub-funciones.
(begin_subtopics)
-Administración de telecomunicaciones.
-Administración de arquitecturas de computadores.
-Administración de sistemas de soporte a decisión de grupos.
-Administración de datos.
-Sistemas de aplicación y propiedad de datos.
-Optmización del ambiente para la creatividad-
-Administración de la calidad: p.e., ingeniería de calidad; equipos de control de calidad.
-Administración de las relaciones de consultoría, {\it outsourcing}.
-Administración para la contención de recursos.
-Asustos operativos asociados con la instalación, operación, transición y mantenimiento de sistemas.
-Actividades y disciplinas de controlque soportan la evolución del software.
-Actividades de ingeniería de software: desarrollo, control, administración, operaciones.
(end_subtopics)
(begin_goals)
(end_goals)
(end_topics)

 

(begin_topics)

-Seguridad y control, virus e integridad de sistemas.
(begin_subtopics)
-Cómo la información es comprometida incluyendo el acceso sin autorización, modificación de la información, bloqueo de servicios, virus.
-Crimen computacional, terrorismo y guerras cibernéticas.
-Virus computacionales, gusanos, caballos de troya.
-Fraude en Internet, leyendas urbanas.
-Correo no deseado (\emph{spam}), avisos publicitarios (\emph{adware}) y mensajes instantáneos no deseados (\emph{spIM}).
-Suplantación y \emph{phishing}.
-Medidas de seguridad computacionales incluyendo tecnológicas (acceso físico restringido, firewalls, encriptación y controles de auditoría) y métodos humanos (legal, ético y gerencia efectiva.)
-Planeamiento de seguridad computacional, incluyendo evaluación de riesgo, evaluación de políticas, implementación, entrenamiento y auditoría.
(end_subtopics)
(begin_goals)
(end_goals)
(end_topics)

(begin_topics)
-Administración de operaciones del computador: p.e. administración de cinta/DASD, {\it scheduling}.
(begin_subtopics)
(end_subtopics)
(begin_goals)
(end_goals)
(end_topics)

\end{BKL2}



\begin{BKL2}{OMC3}{OMC3. Teoría de Decisiones}
Horas:
 
(begin_topics)

-Medición y modelado.
(begin_subtopics)
(end_subtopics)
(begin_goals)
(end_goals)
(end_topics)

 

(begin_topics)

-Decisiones bajo certeza, incerteza, riesgo.

(begin_subtopics)
(end_subtopics)
(begin_goals)
(end_goals)
(end_topics)

 

(begin_topics)
-Información costo/valor, valor competitivo de Sistemas de Información.
(begin_subtopics)
-Motivación/propiedad del trabajo.
(end_subtopics)
(begin_goals)
(end_goals)
(end_topics)

 

(begin_topics)
-Modelos de decisión y Sistemas de Información: optimización, satisfacción.
(begin_subtopics)
(end_subtopics)
(begin_goals)
(end_goals)
(end_topics)

 

(begin_topics)

-Proceso de decisión de grupo.
(begin_subtopics)
(end_subtopics)
(begin_goals)
(end_goals)
(end_topics)

\end{BKL2}



\begin{BKL2}{OMC4}{OMC4. Comportamiento Organizacional}
Horas:
 
(begin_topics)

-Teoría de diseño del trabajo.
(begin_subtopics)
(end_subtopics)
(begin_goals)
(end_goals)
(end_topics)

 

(begin_topics)

-Diversidad cultural.
(begin_subtopics)
(end_subtopics)
(begin_goals)
(end_goals)
(end_topics)

 

(begin_topics)

-Dinámicas de grupo.
(begin_subtopics)
(end_subtopics)
(begin_goals)
(end_goals)
(end_topics)

 

(begin_topics)

-Trabajo en equipo, liderazgo y motivación.
(begin_subtopics)
(end_subtopics)
(begin_goals)
(end_goals)
(end_topics)

 

(begin_topics)

-Uso de influencias, poder y política.
(begin_subtopics)
(end_subtopics)
(begin_goals)
(end_goals)
(end_topics)

 

(begin_topics)

-Estilos cognitivos.
(begin_subtopics)
(end_subtopics)
(begin_goals)
(end_goals)
(end_topics)

 

(begin_topics)

-Negociación y estilos de negociación.
(begin_subtopics)
(end_subtopics)
(begin_goals)
(end_goals)
(end_topics)

 

(begin_topics)
-Construcción de consenso.
(begin_subtopics)
(end_subtopics)
(begin_goals)
(end_goals)
(end_topics)

(begin_topics)
-La organización virtual.
(begin_subtopics)
(end_subtopics)
(begin_goals)
(end_goals)
(end_topics)

(begin_topics)
-Implicaciones de la globalización.
(begin_subtopics)
(end_subtopics)
(begin_goals)
(end_goals)
(end_topics)

\end{BKL2}



\begin{BKL2}{OMC7}{OMC7. Manejo del Proceso de Cambio}
Horas:

 

(begin_topics)

-Razones para la resistencia al cambio.
(begin_subtopics)
(end_subtopics)
(begin_goals)
(end_goals)
(end_topics)

 

(begin_topics)

-Estrategias para motivar el cambio.
(begin_subtopics)
(end_subtopics)
(begin_goals)
(end_goals)
(end_topics)

 

(begin_topics)

-Planeamiento para el cambio.
(begin_subtopics)
(end_subtopics)
(begin_goals)
(end_goals)
(end_topics)

 

(begin_topics)

-Administración del cambio.
(begin_subtopics)
(end_subtopics)
(begin_goals)
(end_goals)
(end_topics)

 

(begin_topics)

-Modelado de procesos y sistemas.
(begin_subtopics)
(end_subtopics)
(begin_goals)
(end_goals)
(end_topics)

 

(begin_topics)

-Experimentación com