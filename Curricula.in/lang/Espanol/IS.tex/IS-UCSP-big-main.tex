\documentclass[twosides,openright,a4paper,final]{book}
\usepackage[printonlyused]{acronym}
\usepackage[spanish]{babel}

\usepackage{amsmath}
\usepackage{amssymb}
\usepackage{graphics}
\usepackage{graphicx}
\usepackage{hlundef} 
\usepackage{ae}

\usepackage{tabularx}

\usepackage{xspace}
\usepackage{pstricks}
\usepackage{pst-tree}
\usepackage{pst-node}
\usepackage{xspace}

\usepackage{watermark}
\usepackage{lscape}
\usepackage{longtable}
\usepackage{fancyhdr}

\usepackage{paralist}
\usepackage{color}
\usepackage{html}
\usepackage{multirow}
\usepackage{rotating}

\usepackage{xkeyval}

\usepackage[top=2.5cm,bottom=3.5cm,left=3cm,right=2.5cm]{geometry}
\usepackage{base-tex/CS-bok-macros}

\newcommand{\fecha}{\today}
\newcommand{\YYYY}{2009\xspace}
\newcommand{\Semester}{2009-I\xspace}

\newcommand{\Universidad}{Universidad Católica San Pablo\xspace}
\newcommand{\InstitutionURL}{http://www.ucsp.edu.pe\xspace}
\newcommand{\underlogotext}{}

\newcommand{\FacultadName}{Ingenierías\xspace}
\newcommand{\DepartmentName}{Sistemas de Información\xspace}
\newcommand{\SchoolFullName}{Programa Profesional de Sistemas de Información\xspace}

\newcommand{\SchoolShortName}{Sistemas de Información\xspace}
\newcommand{\SchoolAcro}{PPSI\xspace}

\newcommand{\GradoAcademico}{Bachiller en Sistemas de Información}
\newcommand{\TituloProfesional}{Licenciado en Sistemas de Información}
\newcommand{\GradosyTitulos}
{\begin{description}
\item [Grado Académico: ] \GradoAcademico~y
\item [Titulo Profesional: ] \TituloProfesional
\end{description}
}

\newcommand{\SchoolURL}{http://www.ucsp.edu.pe\xspace}
\newcommand{\doctitle}{Proyecto de creación del \SchoolFullName {\Large\footnote{\SchoolURL}}\xspace}
\newcommand{\city}{Arequipa\xspace}
\newcommand{\AbstractIntro}{Este documento representa el informe final de la nueva malla curricular \YYYY del \SchoolFullName de la \institution (\textit{\InstitutionURL}) en la ciudad de \city-Perú.}

\newcommand{\OtherKeyStones}
{Un pilar que merece especial consideración en el caso de la \Universidad es el aspecto de valores humanos, básicos y cristianos debido a que forman parte fundamental de los lineamientos básicos de la existencia de la institución.\xspace}




\newcommand{\OnlyUNI}[1]{\xspace}
\newcommand{\NotUNI}[1]{#1\xspace}

\newcommand{\OnlyUTEA}[1]{\xspace}
\newcommand{\NotUTEA}[1]{#1\xspace}

\newcommand{\OnlyUNU}[1]{\xspace}
\newcommand{\NotUNU}[1]{#1\xspace}

\newcommand{\OnlyUCSP}[1]{#1\xspace}
\newcommand{\NotUCSP}[1]{\xspace}

\newcommand{\OnlySPC}[1]{\xspace}
\newcommand{\NotSPC}[1]{#1\xspace}

\newcommand{\OnlyPUCP}[1]{\xspace}
\newcommand{\NotPUCP}[1]{#1\xspace}

\newcommand{\OnlyUTP}[1]{\xspace}
\newcommand{\NotUTP}[1]{#1\xspace}

\newcommand{\OnlyUNSA}[1]{\xspace}
\newcommand{\NotUNSA}[1]{#1\xspace}

\newcommand{\OnlyCS}[1]{\xspace}
\newcommand{\NotCS}[1]{#1\xspace}

\newcommand{\OnlyIS}[1]{#1\xspace}
\newcommand{\NotIS}[1]{\xspace}

\begin{document}

\newcommand{\UNOxUNOxUNO}{Representación de datos fundamentales: no numéricos, numéricos (enteros, reales, errores, precisión)}

\newcommand{\UNOxUNOxUNOxUNO}{Representación básica en máquina de datos numéricos}

\newcommand{\UNOxUNOxUNOxDOS}{Representación básica en máquina de datos no numéricos}

\newcommand{\UNOxUNOxUNOxTRES}{Precisión de la representación de números enteros y de punto flotante}

\newcommand{\UNOxUNOxUNOxCUATRO}{Errores en aritmética de computador y asuntos relacionados de portabilidad}

\newcommand{\UNOxUNOxUNOxCINCO}{Conceptos básicos de arquitectura de computadores}

\newcommand{\UNOxUNOxDOS}{Representación física de información digital: p.e. datos, texto, imágenes, voz, video.}

\newcommand{\UNOxUNOxTRES}{Architecturas de CPU: CPU, memoria, registros, modos de direccionamiento, conjuntos de instrucciones.}

\newcommand{\UNOxUNOxTRESxUNO}{Organización básica; von Neumann, diagrama de bloques, bus de datos, bus de control.}

\newcommand{\UNOxUNOxTRESxDOS}{Instrucciones y modos de direccionamiento: conjuntos de instrucciones y tipos.}

\newcommand{\UNOxUNOxTRESxTRES}{Instrucciones y modos de direccionamiento: lenguaje de máquina.}

\newcommand{\UNOxUNOxTRESxCUATRO}{Modos de direccionamiento.}

\newcommand{\UNOxUNOxTRESxCINCO}{Unidad de control; {\it fetch} y ejecución de instrucciones, {\it fetch} de operadores.}

\newcommand{\UNOxUNOxTRESxSEIS}{CISC, RISC.}

\newcommand{\UNOxUNOxTRESxSIETE}{Organización del computador.}

\newcommand{\UNOxUNOxTRESxOCHO}{Sistemas de memoria.}

\newcommand{\UNOxUNOxCUATRO}{Componentes de sistemas de computadores: bus, controladores, sistemas de almacenamiento, dispositivos periféricos.}

\newcommand{\UNOxUNOxCUATROxUNO}{Periféricos: E/S e interrupciones.}

\newcommand{\UNOxUNOxCUATROxDOS}{Periféricos: métodos de control de E/S, interrupciones.}

\newcommand{\UNOxUNOxCUATROxTRES}{Periféricos: almacenamiento externo, organización física y {\it drives}.}

\newcommand{\UNOxUNOxCUATROxCUATRO}{Almacenamiento auxiliar, cinta, óptico.}

\newcommand{\UNOxUNOxCUATROxCINCO}{Sistemas de almacenamiento, jerarquía.}

\newcommand{\UNOxUNOxCUATROxSEIS}{Organización principal de la memoria, operaciones de bus, tiempos de ciclo.}

\newcommand{\UNOxUNOxCUATROxSIETE}{Memoria caché, lectura/escritura.}

\newcommand{\UNOxUNOxCUATROxOCHO}{Memoria virtual.}

\newcommand{\UNOxUNOxCUATROxNUEVE}{Interfaces entre computadores y otros dispositivos (sensores, efectuadores, etc.)}

\newcommand{\UNOxUNOxCINCO}{Arquitectura de multiprocesadores}

\newcommand{\UNOxUNOxCINCOxUNO}{Arquitecturas de sistemas (multi-procesamiento simple y procesamiento distribuido, pilas)}

\newcommand{\UNOxUNOxCINCOxDOS}{Tecnologías cliente-servidor.}

\newcommand{\UNOxUNOxSEIS}{Lógica y sistemas discretos.}

\newcommand{\UNOxUNOxSEISxUNO}{Elementos lógicos y teoría de {\it switching}; conceptos e implementación de minimización.}

\newcommand{\UNOxUNOxSEISxDOS}{Demoras y peligros de propagación.}

\newcommand{\UNOxUNOxSEISxTRES}{Demultiplexers, multiplexers, decodificadores, codificadores, aditores, substractores.}

\newcommand{\UNOxUNOxSEISxCUATRO}{ROM, PROM, EPROM, EAPROM, RAM.}

\newcommand{\UNOxUNOxSEISxCINCO}{Análisis y síntesis de circuitos síncronos, asíncronos vs. síncronos.}

\newcommand{\UNOxUNOxSEISxSEIS}{Notación de transferencia de registros, condicional e incondicional.}

\newcommand{\UNOxUNOxSEISxSIETE}{Máquinas de estado, redes de conducción, señales de transferencia de carga.}

\newcommand{\UNOxUNOxSEISxOCHO}{Tri-estados y estructuras de bus.}

\newcommand{\UNOxUNOxSEISxNUEVE}{Diagramas de bloque, diagramas de tiempo, lenguaje de transferencia.}

\newcommand{\UNOxDOSxUNO}{Problemas formales y resolución de problemas.}

\newcommand{\UNOxDOSxUNOxUNO}{Estrategias de resolución de problemas usando algoritmos voraces.}

\newcommand{\UNOxDOSxUNOxDOS}{Estrategias de resolución de problemas usando algoritmos divide y vencerás.}

\newcommand{\UNOxDOSxUNOxTRES}{Estrategias de resolución de problemas usando algoritmos de {\it back- tracking}.}

\newcommand{\UNOxDOSxUNOxCUATRO}{Proceso de diseño de software; desde la especificación a la implementación.}

\newcommand{\UNOxDOSxUNOxCINCO}{Establecimiento del problema y determinación algorítmica.}

\newcommand{\UNOxDOSxUNOxSEIS}{Estrategias de implementación ({\it top-down}, {\it bottom-up}; equipo vs. individual).}

\newcommand{\UNOxDOSxUNOxSIETE}{Conceptos de verificación formal.}

\newcommand{\UNOxDOSxUNOxOCHO}{Models de computación formales.}

\newcommand{\UNOxDOSxDOS}{Estructuras de datos básicas: listas, arreglos, cadenas, registros, conjuntos, listas enlazadas, pilas, colas, árboles.}

\newcommand{\UNOxDOSxTRES}{Estructuras de datos complejas: p.e., de datos, texto, voz, imagen, video, hipermedia.}

\newcommand{\UNOxDOSxCUATRO}{Tipos abstractos de datos.}

\newcommand{\UNOxDOSxCUATROxUNO}{Propósito e implementación de tipos abstractos de datos.}

\newcommand{\UNOxDOSxCUATROxDOS}{Especificaciones informales.}

\newcommand{\UNOxDOSxCUATROxTRES}{Especificaciones formales, pre-condiciones y post-condiciones, algebraicas.}

\newcommand{\UNOxDOSxCUATROxCUATRO}{Módulos, cohesión, acoplamiento; diagramas de flujo de datos y conversión a jerarquías.}

\newcommand{\UNOxDOSxCUATROxCINCO}{Correctitud, verificación y validación: pre- y post-condiciones, invariantes.}

\newcommand{\UNOxDOSxCUATROxSEIS}{Estructuras de control; selección, iteración, recursión; tipos de datos y sus usos.}

\newcommand{\UNOxDOSxCINCO}{Estructuras de archivos: secuencial, de acceso directo, hashing, indexados.}

\newcommand{\UNOxDOSxCINCOxUNO}{Archivos (estructura, métodos de acceso): distribución de archivos, conceptos de archivos fundamentales.}

\newcommand{\UNOxDOSxCINCOxDOS}{Archivos (estructura, métodos de acceso): contenidos y estructuras de directorios, nombramiento.}

\newcommand{\UNOxDOSxCINCOxTRES}{Archivos (estructura, métodos de acceso): vista general de seguridad del sistema, métodos de seguridad.}

\newcommand{\UNOxDOSxSEIS}{Estructuras y algoritmos de ordenamiento y búsqueda.}

\newcommand{\UNOxDOSxSEISxUNO}{Algoritmos de ordenamiento ({\it shell sort}, {\it bucket sort}, {\it radix sort}, {\it quick sort}), edición.}

\newcommand{\UNOxDOSxSEISxDOS}{Algoritmos de búsqueda (búsqueda serial, búsqueda binaria y árboles de búsqueda binaria).}

\newcommand{\UNOxDOSxSEISxTRES}{Búsqueda, hashing, resolución de colisiones.}

\newcommand{\UNOxDOSxSIETE}{Eficiencia de algoritmos, complejidad y métricas.}

\newcommand{\UNOxDOSxSIETExUNO}{Análisis asintótico.}

\newcommand{\UNOxDOSxSIETExDOS}{Balance entre tiempo y espacio en algoritmos.}

\newcommand{\UNOxDOSxSIETExTRES}{Clases de complejidad P, NP, P-space; problemas tratables e intratables.}

\newcommand{\UNOxDOSxSIETExCUATRO}{Análisis de límite inferior.}

\newcommand{\UNOxDOSxSIETExCINCO}{NP-completitud.}

\newcommand{\UNOxDOSxSIETExSEIS}{Algoritmos de ordenamiento $O(n^2)$.}

\newcommand{\UNOxDOSxSIETExSIETE}{Algoritmos de ordenamiento $O(n\log n)$.}

\newcommand{\UNOxDOSxSIETExOCHO}{{\it Backtracking}, {\it parsing}, simulaciones discretas, etc.}

\newcommand{\UNOxDOSxSIETExNUEVE}{Fundamentos de análisis de algoritmos.}

\newcommand{\UNOxDOSxOCHO}{Algoritmos recursivos.}

\newcommand{\UNOxDOSxOCHOxUNO}{Conexión de algoritmos recursivos con inducción matemática.}

\newcommand{\UNOxDOSxOCHOxDOS}{Comparación de algoritmos iterativos y recursivos.}

\newcommand{\UNOxDOSxNUEVE}{Redes neuronales y algoritmos genéticos.}

\newcommand{\UNOxDOSxUNOCERO}{Consideraciones avanzadas.}

\newcommand{\UNOxDOSxUNOCEROxUNO}{Funciones computables: modelos de funciones computables de máquinas de Turing.}

\newcommand{\UNOxDOSxUNOCEROxDOS}{Problemas de decisión: problemas enumerables recursivos; indecibilidad.}

\newcommand{\UNOxDOSxUNOCEROxTRES}{Modelo de arquitecturas paralelas.}

\newcommand{\UNOxDOSxUNOCEROxCUATRO}{Algoritmos de arquitecturas paralelas.}

\newcommand{\UNOxDOSxUNOCEROxCINCO}{Problemas matemáticos: problemas bien acondicionados y mal acondicionados.}

\newcommand{\UNOxDOSxUNOCEROxSEIS}{Problemas matemáticos: aproximaciones iterativas a problemas matemáticos.}

\newcommand{\UNOxDOSxUNOCEROxSIETE}{Problemas matemáticos: clasificación de error, computacional, representacional.}

\newcommand{\UNOxDOSxUNOCEROxOCHO}{Problemas matemáticos: aplicaciones de métodos de aproximación interativa.}

\newcommand{\UNOxDOSxUNOCEROxNUEVE}{Límites de computación: computabilidad e intratabilidad algorítmica.}

\newcommand{\UNOxTRESxUNO}{Estructuras de lenguajes de programación fundamentales; comparación de lenguajes y aplicaciones.}

\newcommand{\UNOxTRESxDOS}{Lenguajes de nivel de máquina y ensamblador.}

\newcommand{\UNOxTRESxTRES}{Lenguajes procedurales.}

\newcommand{\UNOxTRESxTRESxUNO}{Ventajas y desventajas de la programación procedural.}

\newcommand{\UNOxTRESxTRESxDOS}{Declaraciones básicas de tipos; operaciones aritméticas y asignación; condicionales.}

\newcommand{\UNOxTRESxTRESxTRES}{Procedimientos, funciones y parámetros; arreglos y registros.}

\newcommand{\UNOxTRESxCUATRO}{Lenguajes no procedurales: lógicos, funcional y basados en eventos.}

\newcommand{\UNOxTRESxCINCO}{Lenguajes de cuarta generación.}

\newcommand{\UNOxTRESxSEIS}{Extensiones orientadas a objetos para lenguajes.}

\newcommand{\UNOxTRESxSIETE}{Lenguajes de programación, diseño, implementación y comparación.}

\newcommand{\UNOxTRESxSIETExUNO}{Historia de los primeros lenguajes.}

\newcommand{\UNOxTRESxSIETExDOS}{Evolución de los lenguajes procedurales.}

\newcommand{\UNOxTRESxSIETExTRES}{Evolución de los lenguajes no procedurales.}

\newcommand{\UNOxTRESxSIETExCUATRO}{Computadores virtuales.}

\newcommand{\UNOxTRESxSIETExCINCO}{Tipos de datos elementales y estructurados.}

\newcommand{\UNOxTRESxSIETExSEIS}{Creación y aplicación de tipos de datos definidos por el usuario.}

\newcommand{\UNOxTRESxSIETExSIETE}{Expresiones, orden de evaluación y efectos secundarios.}

\newcommand{\UNOxTRESxSIETExOCHO}{Subproigramas y corutinas como abstracciones de expresiones y declaraciones.}

\newcommand{\UNOxTRESxSIETExNUEVE}{Manejo de excepciones.}

\newcommand{\UNOxTRESxSIETExUNOCERO}{Mecanismos para compartir y restringir el acceso a datos.}

\newcommand{\UNOxTRESxSIETExUNOUNO}{Ámbito estático vs. dinámico, timepo de vida, visibilidad.}

\newcommand{\UNOxTRESxSIETExUNODOS}{Mecanismos de paso de parámetros; referencia, valor, nombre, resultado, etc.}

\newcommand{\UNOxTRESxSIETExUNOTRES}{Variedades de disciplinas de prueba de tipos y sus mecanismos.}

\newcommand{\UNOxTRESxSIETExUNOCUATRO}{Aplicación de almacenamiento basado en pilas.}

\newcommand{\UNOxTRESxSIETExUNOCINCO}{Aplicación de almacenamiento basado en {\it heaps}.}

\newcommand{\UNOxTRESxSIETExUNOSEIS}{Autómatas de estado finito para modelos restringidos de computación.}

\newcommand{\UNOxTRESxSIETExUNOSIETE}{Aplicación de expresiones regulares al análisis de lenguajes de programación.}

\newcommand{\UNOxTRESxSIETExUNOOCHO}{Uso de gramáticas libres de contexto y de autómatas de pila.}

\newcommand{\UNOxTRESxSIETExUNONUEVE}{Equivalencia entre gramáticas libres de contexto y autómatas de pila.}

\newcommand{\UNOxTRESxSIETExDOSCERO}{Uso de autómatas de pila en el {\it parsing} de lenguajes de programación.}

\newcommand{\UNOxTRESxSIETExDOSUNO}{Proceso de traducción de lenguajes, compiladores a interpretadores.}

\newcommand{\UNOxTRESxSIETExDOSDOS}{Semántica de los lenguajes de programación.}

\newcommand{\UNOxTRESxSIETExDOSTRES}{Paradigmas y lenguajes de programación funcional.}

\newcommand{\UNOxTRESxSIETExDOSCUATRO}{Construcciones programación paralela.}

\newcommand{\UNOxTRESxSIETExDOSCINCO}{Lenguajes procedurales: problemas de implementación; mejora del rendimiento.}

\newcommand{\UNOxTRESxSIETExDOSSEIS}{Compiladores y traductores.}

\newcommand{\UNOxTRESxSIETExDOSSIETE}{Lenguajes de muy alto nivel: SQL, lenguajes de cuarta generación.}

\newcommand{\UNOxTRESxSIETExDOSOCHO}{Diseño orientado a objetos, lenguajes y programación.}

\newcommand{\UNOxTRESxSIETExDOSNUEVE}{Lenguajes de programación lógica: LISP, PROLOG; programación orientada a lógica.}

\newcommand{\UNOxTRESxSIETExTRESCERO}{Generadores de código.}

\newcommand{\UNOxTRESxSIETExTRESUNO}{Shells de sistemas expertos.}

\newcommand{\UNOxTRESxSIETExTRESDOS}{Lenguajes de diseño de software.}

\newcommand{\UNOxCUATROxUNO}{Arquitectura, objetivos y estructura de un sistema operativo; métodos de estructuración, modelos por capas.}

\newcommand{\UNOxCUATROxDOS}{Interacción del sistema operativo con la arquitectura de hardware.}

\newcommand{\UNOxCUATROxTRES}{Administración de procesos: procesos concurrentes, sincronización.}

\newcommand{\UNOxCUATROxTRESxUNO}{Tareas, procesos, interrupciones.}

\newcommand{\UNOxCUATROxTRESxDOS}{Estructuras, listas de espera, bloques de control de procesos.}

\newcommand{\UNOxCUATROxTRESxTRES}{Ejecución concurrente de procesos.}

\newcommand{\UNOxCUATROxTRESxCUATRO}{Acceso compartido, condiciones de ejecución.}

\newcommand{\UNOxCUATROxTRESxCINCO}{{\it Deadlock}; causas, condiciones, prevención.}

\newcommand{\UNOxCUATROxTRESxSEIS}{Modelos y mecanismos (p.e., {\it busy waiting}, {\it spin locks}, algoritmo de Deker).}

\newcommand{\UNOxCUATROxTRESxSIETE}{{\it Switching} preferente y no preferente.}

\newcommand{\UNOxCUATROxTRESxOCHO}{{\it Schedulers} y políticas de {\it scheduling}.}

\newcommand{\UNOxCUATROxCUATRO}{Administración de memoria.}

\newcommand{\UNOxCUATROxCUATROxUNO}{Memoria física y registros.}

\newcommand{\UNOxCUATROxCUATROxDOS}{{\it Overlays}, {\it swapping}, particiones.}

\newcommand{\UNOxCUATROxCUATROxTRES}{Páginas y segmentos.}

\newcommand{\UNOxCUATROxCUATROxCUATRO}{Política de posicionamiento y reposicionamiento.}

\newcommand{\UNOxCUATROxCUATROxCINCO}{{\it Thrashing}, {\it working sets}.}

\newcommand{\UNOxCUATROxCUATROxSEIS}{Listas libres, {\it layout}; servidores, interrupciones; recuperación de fallos.}

\newcommand{\UNOxCUATROxCUATROxSIETE}{Protección de memoria, administración de la recuperación.}

\newcommand{\UNOxCUATROxCINCO}{Asignación y programación de recursos.}

\newcommand{\UNOxCUATROxCINCOxUNO}{{\it Suites} de protocolos (comuniación y redes); {\it streams} y datagramas.}

\newcommand{\UNOxCUATROxCINCOxDOS}{Internet {\it working} y {\it routing}; servidores y servicios.}

\newcommand{\UNOxCUATROxCINCOxTRES}{Tipos de sistemas operativos: de usuario simple, multi-usuario, de red.}

\newcommand{\UNOxCUATROxCINCOxCUATRO}{Sincronización y temporización en sistemas distribuidos y de tiempo real.}

\newcommand{\UNOxCUATROxCINCOxCINCO}{Consideraciones especiales en sistemas de tiempo real; fallas, riesgos y recuperación.}

\newcommand{\UNOxCUATROxCINCOxSEIS}{Utilidades de sistemas operativos.}

\newcommand{\UNOxCUATROxCINCOxSIETE}{Evolución del hardware; fuerzas y restricciones económicas.}

\newcommand{\UNOxCUATROxCINCOxOCHO}{Arquitectura de los sistemas de tiempo real y sistemas empotrados.}

\newcommand{\UNOxCUATROxCINCOxNUEVE}{Consideraciones especiales en sistemas de tiempo real empotrados: requerimientos {\it hard-timing}..}

\newcommand{\UNOxCUATROxSEIS}{Administración de almacenamiento secundario.}

\newcommand{\UNOxCUATROxSIETE}{Sistemas de archivos y de directorios.}

\newcommand{\UNOxCUATROxOCHO}{Protección y seguridad.}

\newcommand{\UNOxCUATROxNUEVE}{Sistemas operativos distribuidos.}

\newcommand{\UNOxCUATROxUNOCERO}{Soporte del sistema operativo para interacción humano-computador: p.e., GUI, video interactivo.}

\newcommand{\UNOxCUATROxUNOUNO}{Interoperatividad y compatibilidad de sistemas operativos: p.e., sistemas abiertos.}

\newcommand{\UNOxCUATROxUNODOS}{Utilidades de los sistemas operativos, herramientas, comandos y programación {\it shell}.}

\newcommand{\UNOxCUATROxUNOTRES}{Administración y gerenciamiento de sistemas.}

\newcommand{\UNOxCUATROxUNOTRESxUNO}{{\it Bootstrapping} del sistema/carga inicial de programa.}

\newcommand{\UNOxCUATROxUNOTRESxDOS}{Generación del sistema.}

\newcommand{\UNOxCUATROxUNOTRESxTRES}{Configuración del sistema.}

\newcommand{\UNOxCUATROxUNOTRESxCUATRO}{Análisis, evaluación y monitoreo de rendimiento.}

\newcommand{\UNOxCUATROxUNOTRESxCINCO}{Optimización y {\it tuning} del sistema.}

\newcommand{\UNOxCUATROxUNOTRESxSEIS}{Funciones de administración del sistema: copias de seguridad, securidad y protección.}

\newcommand{\UNOxCINCOxUNO}{Estándares, modelos y tendencias internacionales en telecomunicaciones.}

\newcommand{\UNOxCINCOxUNOxUNO}{Redes de computadoras y control: topologías, portadores comunes, equipos.}

\newcommand{\UNOxCINCOxUNOxDOS}{Diseño y administración de redes: arquitecturas de red (ISO, SNA, DNA).}

\newcommand{\UNOxCINCOxDOS}{Transmisión de datos: media, técnicas de señalización, impedimento de transmisión, codificación, error.}

\newcommand{\UNOxCINCOxDOSxUNO}{Tecnologías de sistemas de comunicaciones: medios de transmisión, analógico-digital.}

\newcommand{\UNOxCINCOxTRES}{Configuración de línea: control de rror, control de flujo, multiplexado.}

\newcommand{\UNOxCINCOxCUATRO}{Redes de área local.}

\newcommand{\UNOxCINCOxCUATROxUNO}{Topologías, control de acceso al medio, multiplexado.}

\newcommand{\UNOxCINCOxCUATROxDOS}{Redes de área local y WANs: topología, {\it gateways}, usos (funciones y oficina).}

\newcommand{\UNOxCINCOxCUATROxTRES}{Determinación de requerimientos, monitoreo y control del rendimiento, aspectos económicos.}

\newcommand{\UNOxCINCOxCUATROxCUATRO}{Arquitectura de los sistemas distribuidos.}

\newcommand{\UNOxCINCOxCUATROxCINCO}{Aspectos de hardware de los sitemas distribuidos.}

\newcommand{\UNOxCINCOxCINCO}{Redes de área amplia: técnicas de {\it switching}, de {\it broadcast}, {\it routing}.}

\newcommand{\UNOxCINCOxSEIS}{Arquitecturas y protocolos de redes.}

\newcommand{\UNOxCINCOxSIETE}{{\it Internetworking}}

\newcommand{\UNOxCINCOxOCHO}{Configuración de redes, análisis y monitoreo de rendimiento.}

\newcommand{\UNOxCINCOxNUEVE}{Seguridad de redes: encriptación, firmas digitales, autenticación.}

\newcommand{\UNOxCINCOxUNOCERO}{Redes de alta velocidad: p.e. ISDN, SMDS, ATM, FDDI de banda ancha.}

\newcommand{\UNOxCINCOxUNOUNO}{Tecnologías emergentes: ATM, ISDN, redes de satélites, redes ópticas, etc., voz, datos y videos integrados.}

\newcommand{\UNOxCINCOxUNODOS}{Aplicación: p.e., cliente-servidor, EDI, EFT, redes de teléfonos, e-mail, multimedia, video.}

\newcommand{\UNOxCINCOxUNODOSxUNO}{Métodos y transmisión de información gráfica y de video.}

\newcommand{\UNOxCINCOxUNOTRES}{Protocolos de sistemas abiertos.}

\newcommand{\UNOxCINCOxUNOTRESxUNO}{Protocolos de transporte.}

\newcommand{\UNOxCINCOxUNOTRESxDOS}{Protocolos de soporte de aplicaciones: encriptación; compromiso, concurrencia.}

\newcommand{\UNOxCINCOxUNOCUATRO}{Distribución de información.}

\newcommand{\UNOxCINCOxUNOCUATROxUNO}{Structura de redes.}

\newcommand{\UNOxCINCOxUNOCUATROxDOS}{Tecnología cliente-servidor/cliente-servidor delgada.}

\newcommand{\UNOxCINCOxUNOCUATROxTRES}{Redes, {\it routing}, análisis de desempeño.}

\newcommand{\UNOxCINCOxUNOCUATROxCUATRO}{Sistemas de comunicaciones.}

\newcommand{\UNOxSEISxUNO}{DBMS: características, funciones y arquitectura.}

\newcommand{\UNOxSEISxUNOxUNO}{DBMS (características, funciones, arquitectura); componentes de un sistema de bases de datos.}

\newcommand{\UNOxSEISxUNOxDOS}{DBMS: vista general de álgebra relacional.}

\newcommand{\UNOxSEISxUNOxTRES}{Diseño lógico (diseño independiente de DBMS): ER, orientado a objetos.}

\newcommand{\UNOxSEISxDOS}{Modelos de datos: relacional, jerárquico, de red, de objetos, de objetos semánticos.}

\newcommand{\UNOxSEISxDOSxUNO}{Teminología de modelado de datos relacional.}

\newcommand{\UNOxSEISxDOSxDOS}{Modelado conceptual (p.e., entidad-relación, orientado a objtos).}

\newcommand{\UNOxSEISxDOSxTRES}{Interconversión entre tipos de modelos (p.e. jerárquico a relacional).}

\newcommand{\UNOxSEISxTRES}{Normalización}

\newcommand{\UNOxSEISxCUATRO}{Integridad (referencial, de item de datos, de intra-relaciones): representación de relaciones; de entidad y referencial.}

\newcommand{\UNOxSEISxCINCO}{Lenguajes de definición de datos (lenguajes de definición de esquemas, herramientas de desarrollo gráficas).}

\newcommand{\UNOxSEISxSEIS}{Interfaz de aplicaciones.}

\newcommand{\UNOxSEISxSEISxUNO}{Función soportada por sistemas de bases de datos típicos; métodos de acceso, seguridad.}

\newcommand{\UNOxSEISxSEISxDOS}{DML, consulta, QBE, SQL, etc.: lenguaje de consultas de bases de datos; definitción de datos, consulta.}

\newcommand{\UNOxSEISxSEISxTRES}{Interfaces de aplicación y de usuario (DML, consulta, QBE, SQL).}

\newcommand{\UNOxSEISxSEISxCUATRO}{Objetos de pantalla basados en eventos (botones, listas, etc.)}

\newcommand{\UNOxSEISxSEISxCINCO}{Procesamiento físico de transacciones; consideraciones cliente-servidor.}

\newcommand{\UNOxSEISxSEISxSEIS}{Distribución de consideraciones de procesamiento del cliente y del servidor.}

\newcommand{\UNOxSEISxSIETE}{Procesadores inteligentes de consultas y organización de consultas, herramientas OLAP.}

\newcommand{\UNOxSEISxOCHO}{Bases de datos distribuidas, repositorios y {\it warehouses}}

\newcommand{\UNOxSEISxNUEVE}{Productos DBMS: desarrollos recientes en sistemas de bases de datos (p.e., hipertexto, hipermedia, ópticos).}

\newcommand{\UNOxSEISxUNOCERO}{Máquinas y servidores de bases de datos.}

\newcommand{\UNOxSEISxUNOUNO}{Administración de datos y bases de datos.}

\newcommand{\UNOxSEISxUNOUNOxUNO}{Administración de datos.}

\newcommand{\UNOxSEISxUNOUNOxDOS}{Administración de bases de datos: impacto social de las bases de datos; seguridad y privacidad.}

\newcommand{\UNOxSEISxUNOUNOxTRES}{Propiedad y controles de acceso para sistema de datos y de aplicaciones.}

\newcommand{\UNOxSEISxUNOUNOxCUATRO}{Modelos de acceso basados en roles y capacidades.}

\newcommand{\UNOxSEISxUNOUNOxCINCO}{Replicación.}

\newcommand{\UNOxSEISxUNOUNOxSEIS}{Planeamiento de la capacidad del sistema.}

\newcommand{\UNOxSEISxUNOUNOxSIETE}{Planeamiento y administración de redundancia, seguridad y copias de seguridad.}

\newcommand{\UNOxSEISxUNODOS}{Diccionarios, enciclopedias y repositorios de datos.}

\newcommand{\UNOxSEISxUNOTRES}{Recuperación de información: p.e. herramientas de Internet, procesamiento de imágenes, hipermedia.}

\newcommand{\UNOxSIETExUNO}{Representación del conocimiento.}

\newcommand{\UNOxSIETExUNOxUNO}{Historia, contexto y límites de la inteligencia artificial; el test de Turing.}

\newcommand{\UNOxSIETExDOS}{Ingeniería del conocimiento.}

\newcommand{\UNOxSIETExTRES}{Proceso de inferencia.}

\newcommand{\UNOxSIETExTRESxUNO}{Estrategias de control básicas (p.e., por profundidad y por amplitud).}

\newcommand{\UNOxSIETExTRESxDOS}{Razonamiento hacia adelante y hacia atrás.}

\newcommand{\UNOxSIETExTRESxTRES}{Búsqueda heurística.}

\newcommand{\UNOxSIETExTRESxCUATRO}{Sistemas expertos.}

\newcommand{\UNOxSIETExCUATRO}{Otras técnicas: lógica difusa, razonamiento basado en casos, lenguaje natural y reconocimiento del habla.}

\newcommand{\UNOxSIETExCINCO}{Sistemas basados en conocimiento.}

\newcommand{\UNOxSIETExCINCOxUNO}{Lenguaje natural, habla y visión.}

\newcommand{\UNOxSIETExCINCOxDOS}{Reconocimiento de patrones.}

\newcommand{\UNOxSIETExCINCOxTRES}{Aprendizaje de máquina.}

\newcommand{\UNOxSIETExCINCOxCUATRO}{Robótica.}

\newcommand{\UNOxSIETExCINCOxCINCO}{Redes neuronales.}

\newcommand{\DOSxUNOxUNO}{Modelos organizacionales jerárquicos y de flujo.}

\newcommand{\DOSxUNOxDOS}{Grupos de trabajo organizacionales.}

\newcommand{\DOSxUNOxTRES}{Envergadura organizacional: usuario simple, grupo de trabajo, equipo, empresa, global.}

\newcommand{\DOSxUNOxCUATRO}{Rol de Sistemas de Información dentro de la empresa: estratégico, táctico y operativo.}

\newcommand{\DOSxUNOxCINCO}{Efecto de Sistemas de Información en la estructura organizacional; Sistemas de Información y mejora continua.}

\newcommand{\DOSxUNOxSEIS}{Estructura organizacional: centralizada, descentralizada, matriz.}

\newcommand{\DOSxUNOxSIETE}{Aspectos organizacionales para el uso de sistemas de software en organizaciones.}

\newcommand{\DOSxUNOxOCHO}{Procesos en la organización.}

\newcommand{\DOSxUNOxOCHOxUNO}{Vista estratégica de los procesos organizacionales; conceptos de eficiencia y efectividad organizacional.}

\newcommand{\DOSxUNOxOCHOxDOS}{Integración de las áreas funcionales de la organización.}

\newcommand{\DOSxUNOxOCHOxTRES}{Procesos relacionados a los objetivos financieros, usuario final y orientados al producto.}

\newcommand{\DOSxUNOxOCHOxCUATRO}{Innovación de procesos: análisis, modelado y simulación. }

\newcommand{\DOSxUNOxNUEVE}{Modelado y simulación de procesos de negocio.}

\newcommand{\DOSxUNOxNUEVExUNO}{Automatización del proceso de negocios.}

\newcommand{\DOSxUNOxNUEVExDOS}{Utilización de diagramas de actividad y de la Notación de Modelado de Procesos de Negocio (\emph{Business Process Modeling Notation} - BPMN - ) para el modelado del proceso de negocio.}

\newcommand{\DOSxUNOxNUEVExTRES}{Herramientas para el modelado del proceso de negocio.}

\newcommand{\DOSxUNOxNUEVExCUATRO}{Rediseño de tareas; impacto de la automatización en las prácticas de trabajo.}

\newcommand{\DOSxUNOxNUEVExCINCO}{Alcanzando seguridad y conformidad del proceso.}

\newcommand{\DOSxUNOxNUEVExSEIS}{Monitoreo y control de procesos.}

\newcommand{\DOSxUNOxNUEVExSIETE}{Administración de la cadena de abastecimiento (\emph{Supply Chain Management} - SCM). }

\newcommand{\DOSxUNOxNUEVExOCHO}{Administración de la relación con los clientes (\emph{Customer Relationship Management} - CRM).}

\newcommand{\DOSxUNOxNUEVExNUEVE}{Sistemas de gerenciamiento empresarial (\emph{Enterprise Management Systems} - ERP).}

\newcommand{\DOSxUNOxNUEVExUNOCERO}{El proceso continuo: de procesos estructurados a no estructurados.}

\newcommand{\DOSxUNOxUNOCERO}{Una vista integral de la firma y su relación con proveedores y clientes.}

\newcommand{\DOSxDOSxUNO}{Planeamiento de Sistemas de Información.}

\newcommand{\DOSxDOSxUNOxUNO}{Alineamiento del planeamiento de Sistemas de Información con el planeamiento empresarial.}

\newcommand{\DOSxDOSxUNOxDOS}{Planeamiento estratégico de Sistemas de Información.}

\newcommand{\DOSxDOSxUNOxTRES}{Paneamiento de corto alcance de Sistemas de Información.}

\newcommand{\DOSxDOSxUNOxCUATRO}{Reingeniería.}

\newcommand{\DOSxDOSxUNOxCINCO}{Mejora continua.}

\newcommand{\DOSxDOSxDOS}{Control de la función de Sistemas de Información: p.e., auditoría, {\it outsourcing}.}

\newcommand{\DOSxDOSxTRES}{Administración y contratación de recursos humanos.}

\newcommand{\DOSxDOSxTRESxUNO}{Planeamiento de habilidades.}

\newcommand{\DOSxDOSxTRESxDOS}{Administración del rendimiento del personal.}

\newcommand{\DOSxDOSxTRESxTRES}{Educación y entrenamiento.}

\newcommand{\DOSxDOSxTRESxCUATRO}{Estructuras de competencia, cooperación y premios.}

\newcommand{\DOSxDOSxCUATRO}{Estructuras funcionales de Sistemas de Información -- internas vs. {\it outsourcing}.}

\newcommand{\DOSxDOSxCINCO}{Determinación de las metas y objetivos de la organización de Sistemas de Información.}

\newcommand{\DOSxDOSxSEIS}{Administración de Sistema de Información como un negocio: p.e., definición del cliente, definición de la misión de Sistemas de Información, aspectos críticos del éxito de Sitemas de Información.}

\newcommand{\DOSxDOSxSIETE}{Oficial de Información en Jefe (\emph{Chief Information Officer} - CIO) y funciones del personal.}

\newcommand{\DOSxDOSxOCHO}{Sistemas de Información como una función de servicio: evaluación del desempeño -- externo e interno, {\it marketing} de servicios.}

\newcommand{\DOSxDOSxNUEVE}{Administración financiera de Sistemas de Información.}

\newcommand{\DOSxDOSxUNOCERO}{Uso estratégico de Sistemas de Información: p.e., ventajas competitivas y Sistemas de Información, proceso de reingeniería, Sistemas de Información y calidad.}

\newcommand{\DOSxDOSxUNOUNO}{Trabajo del conocimiento, computación de usuario final: soporte, roles, productividad y actividades.}

\newcommand{\DOSxDOSxUNODOS}{Política de Sistemas de Información y formulación y comunicación de procesos operativos.}

\newcommand{\DOSxDOSxUNOTRES}{Copias de seguridad, planeamiento y recuperación de desastres.}

\newcommand{\DOSxDOSxUNOCUATRO}{Administración de tecnologías emergentes.}

\newcommand{\DOSxDOSxUNOCINCO}{Administración de sub-funciones.}

\newcommand{\DOSxDOSxUNOCINCOxUNO}{Administración de telecomunicaciones.}

\newcommand{\DOSxDOSxUNOCINCOxDOS}{Administración de arquitecturas de computadores.}

\newcommand{\DOSxDOSxUNOCINCOxTRES}{Administración de sistemas de soporte a decisión de grupos.}

\newcommand{\DOSxDOSxUNOCINCOxCUATRO}{Administración de datos.}

\newcommand{\DOSxDOSxUNOCINCOxCINCO}{Sistemas de aplicación y propiedad de datos.}

\newcommand{\DOSxDOSxUNOCINCOxSEIS}{Optmización del ambiente para la creatividad-}

\newcommand{\DOSxDOSxUNOCINCOxSIETE}{Administración de la calidad: p.e., ingeniería de calidad; equipos de control de calidad.}

\newcommand{\DOSxDOSxUNOCINCOxOCHO}{Administración de las relaciones de consultoría, {\it outsourcing}.}

\newcommand{\DOSxDOSxUNOCINCOxNUEVE}{Administración para la contención de recursos.}

\newcommand{\DOSxDOSxUNOCINCOxUNOCERO}{Asustos operativos asociados con la instalación, operación, transición y mantenimiento de sistemas.}

\newcommand{\DOSxDOSxUNOCINCOxUNOUNO}{Actividades y disciplinas de controlque soportan la evolución del software.}

\newcommand{\DOS