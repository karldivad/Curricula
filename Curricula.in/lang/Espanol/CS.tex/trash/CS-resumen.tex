 \chapter*{Resumen ejecutivo}
\AbstractIntro

Todo el contenido del documento está basado en el estandar internacional denominado \textit{Computing Curricula}\footnote{http://www.sigcse.org/cc2001/} en el área específica de Ciencia de la Computación. Este documento es el resultado de un trabajo conjunto de la \textit{Association for Computing Machinery} (ACM) y la Sociedad de Computación de IEEE (IEEE-CS) y puede ser accesado a través de la dirección \textit{http://www.sigcse/cc2001} en internet.

Considerando que existen peculiaridades menores al aplicar esta propuesta internacional a nuestros paises, el modelo de \textit{Computing Curricula} fue utilizado para proponer el documento base de la presente malla. 

\noindent La computación hoy en día presenta 5 perfiles de formación profesional claramente definidos: 
\begin{itemize}
\item \textbf{Ciencia de la Computación} (\textit{Computer Science} -- CS),
\item Ingeniería de Computación (\textit{Computer Engineering} -- CE),
\item Ingeniería de Software (\textit{Software Engineering} -- SE),
\item Sistemas de Información (\textit{Information Systems} -- IS) y 
\item Tecnología de la Información (\textit{Information Technology} -- IT).
\end{itemize}

Los pilares fundamentales que consideramos en esta propuesta curricular son:
\begin{itemize}
\item Una sólida formación profesional en el área de Ciencia de la Computación,
\item Preparación para la generación de empresas de base tecnológica,
\item Una sólida formación ética y proyección a la sociedad
\end{itemize}

Estos pilares redundarán en la formación de profesionales que se puedan desempeñar en cualquier parte del mundo y que ayuden de forma clara al desarrollo de la Industria de Software de nuestro país. 

\OtherKeyStones

El resto de este documento está organizado de la siguiente forma: el Capítulo \ref{chap:intro}, define y explica el campo de acción de la Ciencia de la Computación (Informática), además se hace una muy breve explicación de las distintas carreras del área de computación propuestas por IEEE-CS y ACM.


\OnlySPC{
En el Capítulo \ref{chap:cs-market} se presenta el perfil profesional, un análisis de mercado que incluye un anàlisis de la oferta, demanda y tendencias del conocimiento.

Un anàlisis màs detallado del mercado junto con una encuesta a empresarios es presentada en el Capítulo \ref{chap:cs-estudio-de-mercado}. Este estudio arrojó datos interesantes con relaciòn a la percepciòn y la necesidad empresarial de este perfil profesional.
}

\OnlyUNSA{
En el Capítulo \ref{chap:cs-market} se presenta el perfil profesional, un análisis de mercado reconocidasque incluye un anàlisis de la oferta, demanda y tendencias del conocimiento.

Un anàlisis màs detallado del mercado junto con una encuesta a empresarios es presentada en el Capítulo \ref{chap:cs-estudio-de-mercado}. Este estudio arrojó datos interesantes con relaciòn a la percepciòn y la necesidad empresarial de este perfil profesional.

En el Capítulo \ref{chap:cs-resources} se presentan los recursos de plana docente, infraestructura de laboratorios y financieros necesarios para poder crear esta carrera profesional.
}

El Capítulo \ref{chap:cs-BOK}, muestra los 14 grupos que forman el centro del conocimiento de la Ciencia de la Computación, indicando los tópicos y objetivos cubiertos por cada uno de los temas, pertenecientes a estos grupos.

El Capítulo \ref{chap:cs-malla}, se detalla el contenido y objetivos de los cursos de esta propuesta; sus dependencias; 
número de horas dedicadas a teoría, practica, laboratorio y el creditaje asignado.
\newpage