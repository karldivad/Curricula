\section{Diagnóstico General}

\subsection{Análisis de Currí­culos}
Nuestra \SchoolFullName inicia sus actividades el 2010 con un currí­culo basado en la propuesta internacional de 
Ciencia de la Computación de IEEE y ACM en su versión 2008 \cite{ComputerScience2008}.

Esta propuesta ha permitido que nuestros alumnos sean formados de acuerdo a las recomendaciones internacionales para esta lí­nea
siendo su formación compatible con lo que se enseña a nivel mundial.

Además, el hecho de estar alineados a una propuesta internacional facilita enormemente la movilidad estudiantil y de docentes,
el acceso a estudios de postgrado en el extranjero, la acreditación internacional. Todos estos son factores muy importantes en el 
desarrollo académico y profesional de una Escuela Profesional tan dinámica como lo es la computación hoy en dia.

A nivel internacional la siguiente propuesta para el área de Ciencia de la Computación fue publicada en el documento \cite{CS2013} donde 
inclusive uno de nuestros profesores participó en la elaboración de dicho documento como miembro del comité internacional que la formuló.

Este último documento \cite{CS2013} y el Curriculo de inicial de 2010 sirvieron como insumos para el presente currí­culo 
por lo cual estamos seguros que está totalmente actualizado con relación a las tendencias internacionales del área.

\subsection{Propuestas curriculares de la Escuela en otras universidades nacionales y extranjeras}
\input{\InTexDir/basic-definitions-IEEE-ACM}

\subsection{Diagnóstico especí­fico para la carrera}

\subsubsection{Los grandes retos en el desarrollo de la carrera profesional}
Uno de los caminos que se espera del área de computación en nuestro paí­s pueda producir software a gran escala como las 
grandes empresas productoras de software a nivel mundial y convertirse en referente a nivel nacional, luego latinoamericano y 
finalmente a nivel mundial. En el ámbito de la computación, es común observar que los paí­ses cuentan con
Asociaciones de Productores de Software cuyas polí­ticas están orientadas a la exportación. Siendo así­, 
no tendrí­a sentido preparar a nuestros alumnos sólo para el mercado local o nacional. 
Siendo un reto que nuestros egresados deben estar preparados para desenvolverse en el mundo globalizado que nos ha tocado vivir.

Es necesario recordar que la mayor innovación de productos comerciales de versiones recientes utiliza tecnologí­a que se conocí­a 
en el mundo académico hace 20 años o más. Un ejemplo claro son las bases de datos que soportan datos y consultas espaciales 
desde hace muy pocos años. Sin embargo, utilizan estructuras de datos que ya existí­an hace algunas décadas. 
Es lógico pensar que la gente del área académica no se dedique a estudiar en profundidad la última versión de un 
determinado software cuando esa tecnologí­a ya la conocí­an hace mucho tiempo. Por esa misma razón es raro en el 
mundo observar que una universidad tenga convenios con una transnacional de software para dictar solamente esa 
tecnologí­a pues, nuestra función es generar esa tecnologí­a y no sólo saber usarla.

\subsubsection{Aporte al desarrollo de las personas, instituciones y sociedad}
Nuestros futuros profesionales deben estar orientados a crear nuevas empresas de base tecnológica que puedan 
incrementar las exportaciones de software peruano. Este nuevo perfil está orientado a generar industria innovadora. 
Si nosotros somos capaces de exportar software competitivo también estaremos en condiciones de atraer nuevas inversiones. 
Las nuevas inversiones generarí­an más puestos de empleo bien remunerados y con un costo bajo en relación a otros tipos de industria. 
Bajo esta perspectiva, podemos afirmar que esta carrera será un motor que impulsará al desarrollo del paí­s de forma decisiva 
con una inversión muy baja en relación a otros campos.

Tampoco debemos olvidar que los alumnos que ingresan hoy saldrán al mercado dentro de 5 años aproximadamente y, 
en un mundo que cambia tan rápido, no podemos ni debemos enseñarles tomando en cuenta solamente el mercado actual. 
Nuestros profesionales deben estar preparados para resolver los problemas que habrá dentro de 10 o 15 años y 
eso sólo es posible a través de la investigación en los diferentes problemas de nuestra región.


\subsubsection{Empleabilidad de egresados}
Nuestro egresado podrá prestar sus servicios profesionales en empresas e instituciones públicas y privadas que requieran 
sus capacidades en función del desarrollo que oferta, entre ellas:

\begin{itemize}
\item Empresas dedicadas a la producción de software con calidad internacional.
\item Empresas, instituciones y organizaciones que requieran software de calidad para mejorar sus actividades y/o servicios ofertados.
\end{itemize}


Nuestro egresado puede desempeñarse en el mercado laboral sin ningún problema ya que, en general, la exigencia del 
mercado y campo ocupacional está más orientada al uso de herramientas. Sin embargo, es poco común que los propios 
profesionales de esta carrera se pregunten: ?`qué tipo de formación deberí­a tener si yo quisiera crear esas herramientas además de saber usarlas?.  
Ambos perfiles (usuario y creador) son bastante diferentes pues no serí­a posible usar algo que todaví­a no fue creado. 
En otras palabras, los creadores de tecnologí­a son los que dan origen a nuevos puestos de trabajo y abren 
la posibilidad de que otros puedan usar esa tecnologí­a.

Debido a la formación basada en la investigación, nuestro profesional debe siempre ser un innovador donde trabaje. 
Esta misma formación permite que el egresado piense también en crear su propia empresa de desarrollo de software. 
Considerando que paí­ses como el nuestro tienen un costo de vida mucho menor que Norte América ó Europa, una 
posibilidad que se muestra interesante es la exportación de software pero eso requiere que la calidad del 
producto sea al mismo nivel de lo ofrecido a nivel internacional.

Este perfil profesional también posibilita que nuestros egresados se queden en nuestro paí­s; 
producir software en nuestro paí­s y venderlo fuera es más rentable que salir al extranjero y comercializarlo allá.

El campo ocupacional de un egresado es amplio y están en continua expansión y cambio. 
Prácticamente toda empresa u organización hace uso de servicios de computación de algún tipo, y la buena 
formación básica de nuestros egresados hace que puedan responder a los requerimientos de las mismas exitosamente. 
Este egresado, no sólo podrá dar soluciones a los problemas existentes sino que deberá proponer innovaciones 
tecnológicas que impulsen la empresa hacia un progreso constante.

A medida que la informatización básica de las empresas del paí­s avanza, la necesidad de personas capacitadas 
para resolver los problemas de mayor complejidad aumenta y el plan de estudios que hemos desarrollado tiene 
como objetivo satisfacer esta demanda considerándola a mediano y largo plazo. El campo para las tareas de 
investigación y desarrollo de problemas complejos en computación es también muy amplio y están creciendo dí­a a dí­a a nivel mundial.

Debido a la capacidad innovadora de nuestro egresado, existe una mayor la probabilidad de registrar 
patentes con un alto nivel inventivo lo cual es especialmente importante en nuestros paí­ses.

\subsubsection{Competencias requeridas para el mercado laboral}
En un área tan globalizada como lo es la computación el mercado laboral no podrí­a ser solamente local o nacional.
Hoy en dia, en Arequipa, es posible observar empresas de la India, de California, de Alemania, de Uruguay que operan desde nuestra ciudad.
Esto obliga a que la preparación de nuestros egresados tenga que ser competitiva a nivel mundial.

También existen aquellos defensores de que el Perú debe formar profesionales en esta área solo para problemas locales.
Sin embargo, consideramos que es una de las razones de nuestro atraso pues continuamos siendo un paí­s de usuarios de 
tecnologí­a pero no podemos producirla con calidad internacional aun debido al número muy reducido de profesional con perfil internacional. 

Un análisis muy reciente al respecto que nos ayuda a entender mejor como Perú es observado desde fuera puede ser visto en un 
artí­culo reciente publicado en Boston\footnote{http://bit.ly/2i5lzNM} por el Mag Eddy Wong.
En esta publicación de indica de forma clara que las universidades peruanas han venido formando solamente usuarios de tecnologí­a y 
que además existe un énfasis en mantener asociada la palabra Ingenierí­a a estas carreras.
Esta formación de usuarios crea una dependencia enorme de nuestro paí­s y dificulta enormemente el desarrollo de la industria nacional.
Es cierto que existen más de 100 carreras de esta lí­nea pero los profesionales unicamente son orientados a hacer uso de la tecnologí­a que es importada a nuestro paí­s.
De parte de las empresas del sector de software existe mucha dificultad para encontrar personal capacitado para crear software de calidad internacional.
Esto puede ser corroborado directamente con el Presidente de la Asociación Peruana de Productores de Software (APESOFT) 
quien con frecuencia reclama este problema al ámbito académico.

\subsubsection{Situación de egresados}
La primera promoción egreso a inicios de 2015 con un total de 6 egresados los cuales fueron aceptados en su totalidad 
para ir a estudiar maestrí­as con beca a Brasil en área de Ciencia de la Computación.
El hecho de sea el 100\% de egresados en esta situación es un reflejo claro del nivel que se ha podido alcanzar a 
pesar de las dificultades propias de una universidad nacional en nuestro paí­s.
En este momento estos 6 alumnos ya están culminando con éxito sus estudios de maestrí­a e iniciando los de doctorado.

La segunda promoción egresó a inicios de 2016 con un aproximado de 8 alumnos que han seguido un camino similar al primer 
grupo estudiando maestrí­as becados por concurso a tiempo completo en Arequipa, Holanda, y Brasil.

Estos niveles de competitividad no habrí­an sido posibles si la formación hubiese sido de usuarios o 
solamente orientada al mercado local o nacional. 

Las competencias que más destacan en nuestros egresados son:
\begin{itemize}
\item Habilidad aprender a aprender de forma autónoma,
\item Capacidad de adaptación rápida a grupos de investigación de nivel internacional,
\item Capacidad de adaptación rápida al trabajo con nuevas tecnologí­as,
\item Competencia para trabajo multidisciplinario,
\item Competencia para comunicarse de forma efectiva en inglés y en varios casos también en portugues.
\end{itemize}

Como varios de nuestros alumnos han salido a otros paí­ses, es nuestra responsabilidad que tengan el 
espacio adecuado en nuestro medio para poder integrarlos y mejorar la calidad académica existente.

\subsubsection{Oportunidades de empleo}
Ciencia de Datos está experimentando un crecimiento rápido y no planificado, impulsado por la proliferación de datos complejos y ricos en ciencia, industria y gobierno. Impulsado en parte por informes como el ampliamente citado informe McKinsey que pronosticó la necesidad de cientos de miles de empleos en Data Science en el próxima década.

Nuestros egresados cuentan con títulos de trabajo como \emph{analista de datos}, \emph{analista de negocios o estrategia} y \emph{consultor de análisis de datos}. Se pueden encontrar en empresas del sector público y privado como Telstra, IBM, EY, ANZ, Accenture, CSIRO y Microsoft.

Nuestro egresado puede desempeñarse en el mercado laboral sin ningún problema ya que, en general, la exigencia del mercado y campo ocupacional está mucho más orientada al uso de herramientas. Sin embargo, 
es poco común que los propios profesionales de esta carrera se pregunten: ?`qué tipo de formación debería tener si yo quisiera crear esas herramientas además de saber usarlas?. Ambos perfiles (usuario y creador) 
son bastante diferentes pues no sería posible usar algo que todavía no fue creado. En otras palabras, 
los creadores de tecnología son los que \underline{dan origen a nuevos puestos de trabajo} y abren la 
posibilidad de que otros puedan usar esa tecnología.

Debido a la formación basada en la investigación, nuestro profesional debe siempre ser un innovador 
donde trabaje. Esta misma formación permite que el egresado piense también en crear su propia empresa 
de desarrollo de software. Considerando que países como el nuestro tienen un costo de vida mucho menor 
que Norte América ó Europa, una posibilidad que se muestra interesante es la exportación de software 
pero eso requiere que la calidad del producto sea al mismo nivel de lo ofrecido a nivel internacional.

El campo ocupacional de un egresado es amplio y está en continua expansión y cambio. Prácticamente 
toda empresa u organización hace uso de servicios de computación de algún tipo, y la buena formación 
básica de nuestros egresados hace que puedan responder a los requerimientos de las mismas exitosamente. 
Este egresado, no sólo podrá dar soluciones a los problemas existentes sino que deberá proponer innovaciones 
tecnológicas que impulsen la empresa hacia un progreso constante.

A medida que la informatización básica de las empresas del país avanza, la necesidad de personas 
capacitadas para resolver los problemas de mayor complejidad aumenta y el plan de estudios que hemos 
desarrollado tiene como objetivo satisfacer esta demanda considerandola a mediano y largo plazo. El campo 
para las tareas de investigación y desarrollo de problemas complejos en computación es también muy amplio 
y está creciendo día a día a nivel mundial.

Debido a la capacidad innovadora de nuestro egresado, existe una mayor la probabilidad de registrar 
patentes con un alto nivel inventivo lo cual es especialmente importante en nuestros países.


% \subsubsection{Aporte al desarrollo del conocimiento}
% Las necesidades de los Cientí­ficos e Ingenieros con relación a computación han impulsado mucho la 
% investigación y la innovación en computación. A medida que las computadoras aumentan su poder en la 
% solución de problemas, la ciencia computacional ha crecido tanto en amplitud e importancia. 
% Es una disciplina por derecho propio y se considera que es una de las cinco con mayor crecimiento.
% Una increí­ble variedad de sub-campos han surgido bajo el paraguas de la Ciencia Computacional, 
% incluyendo la biologí­a computacional, quí­mica computacional, mecánica computacional, 
% arqueologí­a computacional, finanzas computacionales, sociologí­a computacional y forense.
