\begin{syllabus}

\course{MA101. Matemática II}{Obligatorio}{MA101}
% Source file: ../Curricula.in/lang/Espanol/cycle/2021-I/Syllabi/BasicSciences/MA101.tex

\begin{justification}
  El curso está enfocado en desarrollar capacidades en comprensión de problemas, entendimiento y aplicación de modelos matemáticos. Con este fin se desarrolla una metodología activa y participativa con uso racional de la tecnología y espacios de trabajo colaborativo. Las sesiones son teóricas y prácticas asociadas a situaciones contextualizadas que motivan al estudiante a involucrarse en su entendimiento y solución.
  EL curso tiene como finalidad abordar los siguientes temas principales el cual se monitoreará todas las semanas, estos temas son los siguientes: Vectores, Funciones de Varias Variables, Derivadas Parciales, Integrales dobles, Series y Ecuaciones diferenciales ordinarias de primer orden y de segundo o másn orden.
\end{justification}

\begin{goals}
  \item Capacidad de aplicar conocimientos de matemáticas.
  \item Capacidad de aplicar conocimientos de ingeniería.
  \item Capacidad de aplicar conocimientos de computación y de matemáticas.
\end{goals}

--COMMON-CONTENT--

\begin{unit}{Vectores}{}{StewartMVar,DennisZ}{24}{C1,C20}
  \begin{topics}      
  \item Componentes, canónicos, problemas de fuerza o velocidad.
  \item Ángulo entre dos vectores, calcular trabajo por una fuerza constante, momento de una fuerza, volumen.
  \item Ecuación de la recta y el plano, Dibujar planos, Distancia entre puntos, planos y rectas.
  \item Calcular trabajo por fuerza constante, momento de una fuerza, volumen.
  \item Dibujar funciones de dos y tres variables, curvas de nivel.
    \end{topics}

  \begin{learningoutcomes}
  \item Expresar un vector mediante sus componentes y usar operaciones vectoriales para interpretar los resultados geométricamente, utilizando las combinaciones lineales de vectores unitarios estándar o canónicos.
  \item Entender el sistema de coordenadas rectangulares tridimensional y analizar vectores en el espacio; hallando el ángulo entre dos vectores y el vector perpendicular entre dos vectores.
  \item Aplicar conocimientos sobre las propiedades vectoriales en propiedades físicas y químicas.
  \item Dar un conjunto de ecuaciones paramétricas para una recta en el espacio.
  \item Dar una ecuación lineal para representar un plano en el espacio, utilizándolo para dibujar el plano dado por la ecuación lineal.
  \item Hallar las distancias entre puntos, planos y rectas en el espacio.
  \end{learningoutcomes}
\end{unit}

\begin{unit}{Derivadas e Integrales}{}{StewartMVar,DennisZ}{24}{C1,C20}
  \begin{topics}
    \item Interpretar las derivadas direccionales, Análisis de errores, regla de la cadena.
    \item Derivada direccional, gradiente de una función de dos variables, aplicación.
    \item Extremos absolutos y extremos relativos / criterio de las segundas derivadas parciales.
    \item Áreas, volúmenes y valores promedios.
    \item Integrales dobles usando coordenadas polares.
\end{topics}  

  \begin{learningoutcomes}
   \item Entender la notación para una función de varias variables, ayudándolo a dibujar la gráfica en el espacio. Realizar las gráficas de curvas de nivel de una función de dos variables.
   \item Hallar y utilizar las derivadas parciales de una función de dos o más variables, para entender los conceptos de incrementos y diferenciales.
   \item Utilizar una diferencial como aproximación y utilizar la regla de la cadena para funciones de varias variables.
   \item Hallar y usar las derivadas direccionales de una función de dos variables, utilizándolo para encontrar la gradiente de una función de dos o más variables.
   \item Hallar extremos absolutos y relativos de una función de dos variables, utilizando el criterio de las segundas derivadas parciales.
   \item Resolver problemas de optimización con funciones de varias variables sin y con restricciones, utilizando el método de los multiplicadores de Lagrange.
   \item Evaluar y utilizar una integral iterada para hallar el área de una región plana en coordenadas cartesianas.
  \end{learningoutcomes}
\end{unit}

\begin{unit}{Series y Sucesiones}{}{StewartMVar,DennisZ}{12}{C1,C20}
  \begin{topics}
    \item Sucesiones - límite de una sucesión-reconocimiento de patrones de una sucesión.
    \item Series infinitas series geométricas-Criterio de la integral y series P.
    \item Criterio del cociente / Polinomios de Taylor y de Maclaurin.
    \item Series de Taylor / Maclaurin.
   \end{topics}
  
  \begin{learningoutcomes}
    \item Hallar la masa, el centro de masa y los momentos de inercia de una lámina plana utilizando una integral doble.
    \item Determinar si una sucesión converge o diverge, utilizando límites y regla de L'Hospital.
    \item Entender la definición de una serie infinita usando propiedades para encontrar si son convergentes o divergentes.
    \item Emplear criterios y propiedades de las series infinita para determinar si es convergente o divergente. Encontrar aproximaciones polinomiales de las funciones mediante polinomios de Taylor y Maclaurin a funciones elementales.
    \item Comprender la definición de una serie de potencia para calcular el radio y el intervalo de convergencia. Hallar una serie de Taylor o de Maclaurin para una función.
    \end{learningoutcomes}
\end{unit}

\begin{unit}{Ecuaciones Diferenciales}{}{StewartMVar,DennisZ}{30}{C1,C20}
  \begin{topics}
    \item Definiciones y terminologías / Problemas con valores iniciales.
    \item Variable separable - Ecuaciones Lineales.
    \item Modelos Lineales de Crecimiento (Poblacional), Decaimiento (Bacterias - Vida Media - Mezclas - Ley de Newton.)
    \item Ecuaciones Exactas - Soluciones por sustitución.
    \item Modelos No lineales (Cadena cayendo - Crecimiento población logística - Tanque cilíndrico con gotera - cónico invertido, Colector solar, Modelo de inmigración.
    \item Series radiactivas - Mezclas - Mallas.
    \item Concentración de nutrientes - Ley de Newton.
    \item Problemas con valores iniciales - homogénea y no homogénea.
    \item Método del anulador - Ecuación de Cauchy Euler.
   \end{topics}
  
  \begin{learningoutcomes}
    \item Entender las definiciones y terminología de ecuaciones diferenciales con y sin valores iniciales.
    \item Explicar los modelos de ecuaciones diferenciales de 1er y 2do orden.
    \item Resolver las ecuaciones diferenciales de primer orden por el método de variables separables.
    \item Resolver las ecuaciones lineales diferenciales de primer orden homogéneas y no homogéneas usando el factor integrante.
    \item Resolver ecuaciones diferenciales de primer orden exactas con y sin valores iniciales, usando factor de integración.
    \item Obtener la solución general de una ecuación lineal homogénea de segundo orden con coeficientes constantes.
    \item Resolver la ecuación de Euler de segundo orden, aplicando para analizar aplicaciones en vibraciones mecánicas y oscilaciones en circuitos eléctricos.
    \end{learningoutcomes}
\end{unit}

\begin{coursebibliography}
\bibfile{BasicSciences/MA101}
\end{coursebibliography}

\end{syllabus}
