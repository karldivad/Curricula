\begin{syllabus}

\course{MA100. Matemática I}{Obligatorio}{MA100}
% Source file: ../Curricula.in/lang/Espanol/cycle/2021-I/Syllabi/BasicSciences/MA100.tex

\begin{justification}
El curso tiene como objetivo desarrollar en los estudiantes la capacidad de analizar modelos en ciencia e ingeniería mediante herramientas de cálculo
diferencial e integral, con funciones reales de variable real.
En el curso se estudian y aplican conceptos relacionados con funciones, derivadas e integrales de funciones reales de una variable, las cuáles se utilizarán como base y apoyo para el estudio de nuevos contenidos y materias.
También busca lograr capacidades heurísticas, de razonamiento y comunicación para abordar problemas del mundo real mediante los conceptos y procedimientos aprendidos.
\end{justification}

\begin{goals}
\item Aplicar conocimientos de matemáticas.
\end{goals}

--COMMON-CONTENT--

\begin{unit}{Vectores y números complejos}{}{StewartCOVar,Larson}{20}{C1}
   \begin{topics}
      \item Operaciones con números complejos
      \item Teorema de Moivre
   \end{topics}

   \begin{learningoutcomes}
      \item Definir y operar con números complejos, calculando su forma polar y exponencial.
      \item Utilizar el teorema de Moivre para simplificar los cálculos de complejos.
      \item Operar con vectores caracterizándolo por su dirección y magnitud.  Representar una función a partir de la relación de conjuntos,  dados verbal, gráfica y algebraicamente,  en un diagrama de Venn  y/o en el plano cartesiano proporcionando, si es posible, su regla de correspondencia  y sus principales características.
   \end{learningoutcomes}
\end{unit}

\begin{unit}{Funciones de una variable}{}{StewartCOVar,Larson}{10}{C20}
   \begin{topics}
      \item Definición, características y representación gráfica.
      \item Álgebra de funciones.
      \item Funciones lineales, polinomiales, sinusoidales, exponenciales y logarítmicas.
      \item Modelamiento de situaciones cercanas a la realidad usando funciones.
   \end{topics}

   \begin{learningoutcomes}
      \item Modelar situaciones reales del entorno cercano usando funciones constantes, lineales, cuadráticas y polinómicas, y otras resultante de las operaciones ( f+-*/g, fog  , af(bx -c)+d) entre funciones elementales, con énfasis en el cálculo, la gráfica y la interpretación de la pendiente y concavidad en un contexto aplicado. 
      \item Modelar situaciones reales del entorno cercano usando funciones sinusoidales.
      \item Usar las funciones exponenciales, logarítmica y logística para modelar situaciones reales del entorno cercano que se ajustan a sus comportamientos, reconociendo sus características (crecimiento, decrecimiento, comportamiento asintótico).
      \item Reconoce y construye funciones trigonométricas.
      \item Aplicar reglas para transformar funciones.
   \end{learningoutcomes}
\end{unit}

\begin{unit}{Derivadas de funciones}{}{StewartCOVar,Larson}{20}{C1}
   \begin{topics}
      \item Definición de derivada como razón de cambio y como pendiente de la tangente a la curva en un punto.
      \item Reglas de derivación.
      \item Aplicaciones de las derivadas en problemas de velocidades relacionadas.
      \item Aplicaciones de las derivadas en problemas de optimización de funciones.
   \end{topics}

   \begin{learningoutcomes}
      \item Resolver problemas usando la derivada de una función como una razón de cambio entre sus dos variables o como la pendiente de la recta tangente en un punto, aplicando las reglas de derivación a funciones simples. 
      \item Aproximar funciones usando los diferenciales. $df=f'(x)dx$, aplicando las reglas de la derivación para calcular derivada de funciones compuestas e implícitas con la notación de Leibniz.
      \item Resolver problemas de contexto real del entorno cercano que involucran el cálculo de velocidades relacionadas derivando funciones simples, compuestas e implícitamente teniendo presente el uso de los diferenciales.
      \item Resolver problemas de optimización analizando el comportamiento de una función mediante su primera y segunda derivada (crecimiento, decrecimiento, concavidad, extremos).
   \end{learningoutcomes}
\end{unit}

\begin{unit}{Integrales}{}{StewartCOVar,Larson}{22}{C20}
   \begin{topics}
      \item Integral indefinida y métodos de integración (sustitución, integración por partes, sustituciones trigonométricas y descomposición por fracciones parciales).
      \item Suma de Riemann para estimar áreas.
      \item Teoremas del cálculo (TFC1, TFC2, TCN).
      \item Cálculo de área entre curvas y valor promedio.
      \item Ecuaciones diferenciales que se resuelven por variables separables.
   \end{topics}

   \begin{learningoutcomes}
      \item Resolver integrales indefinidas mediante diversos métodos (sustitución, integración por partes, sustitución trigonométrica, descomposición en fracciones parciales).
      \item Estimar el área bajo una curva mediante la división en rectángulos y sumas de Riemann, con interpretaciones en contextos de física y otros cotidianos.  
      \item Aplicar los teoremas del cálculo (TFC1, TFC2, TCN) para resolver integrales indefinidas usando diferentes métodos de integración.
      \item Resolver problemas de área y valor promedio de una función, con las correspondientes interpretaciones físicas de la integral en cinemática. 
      \item Modelar situaciones reales usando ecuaciones diferenciales y resolverlas usando método de separación de variables. (Ley de enfriamiento de Newton, Dinámica poblacional (Logística, curva de aprendizaje), etc.).
      \item Define un número complejo y lo representa en diversas formas. Usa la fórmula de Moivre al cálculo de operaciones con complejos.
   \end{learningoutcomes}
\end{unit}

\begin{coursebibliography}
\bibfile{BasicSciences/MA100}
\end{coursebibliography}

\end{syllabus}
   
