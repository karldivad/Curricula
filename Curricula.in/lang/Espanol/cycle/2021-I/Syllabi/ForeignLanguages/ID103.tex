\begin{syllabus}

\course{ID201. Inglés III}{Obligatorio}{ID201}
% Source file: ../Curricula.in/lang/Espanol/cycle/2021-I/Syllabi/ForeignLanguages/ID103.tex

\begin{justification}
El aprendizaje del idioma ingles en los estudiantes de nivel superior se ha 
convertido en elemento necesario para su desarrollo académico, 
personal y profesional. La mayor cantidad de literatura académica de los 
diversos campos del saber es redactada en inglés, lo cual garantiza que las 
personas que cuenten con el dominio del idioma puedan siempre estar actualizados. 

Asimismo, el conocimiento de este idioma permite tener perspectivas 
sociales y culturales más amplias. En ese sentido, para un efectivo aprendizaje 
de la lengua es necesario el desarrollo de las cuatro (4) habilidades: 
escuchar, hablar, leer y escribir considerando los lineamientos del 
Marco Común Europeo de Referencia de Lengua - MCERL.
\end{justification}

\begin{goals}
\item Comunicarse e intercambiar información de manera limitada en situaciones predecibles de forma general.
\item Alcanzar el nivel B1 según el MCERL.
\item Manejar terminología propia del área de estudios.
\end{goals}

--COMMON-CONTENT--

\begin{unit}{Be polite}{}{de2002marco}{12}{C25}
   \begin{topics}
      \item Imperatives (+/-).
      \item Modals – can/could – have to – should.
      \item Connector (effect).
   \end{topics}

   \begin{learningoutcomes}
      \item Make polite requests.
   \end{learningoutcomes}
\end{unit}

\begin{unit}{My neighborhood}{}{de2002marco}{12}{C25}
   \begin{topics}
      \item Adverbial phrases (place – word order).
      \item Prepositional phrases (place, time and movement).
      \item Connector (effect).
   \end{topics}

   \begin{learningoutcomes}
      \item Give directions.
      \item Describe places and location.
   \end{learningoutcomes}
\end{unit}

\begin{unit}{The best in town!}{}{de2002marco}{12}{C25}
   \begin{topics}
      \item Comparative.
      \item Definite article.
      \item Intensifiers (too + enough).
      \item Superlative.
      \item Connector (effect).
   \end{topics}

   \begin{learningoutcomes}
      \item Compare places.
   \end{learningoutcomes}
\end{unit}

\begin{unit}{What can you do?}{}{de2002marco}{12}{C25}
   \begin{topics}
      \item Modal verb “can”.
      \item Action verbs.
      \item Phrasal verbs – common.
      \item Connector (effect).
   \end{topics}

   \begin{learningoutcomes}
      \item Talk about abilities
   \end{learningoutcomes}
\end{unit}

\begin{unit}{What a day!}{}{de2002marco}{12}{C25}
   \begin{topics}
      \item Present simple – action verbs.
      \item Phrasal verbs – common.
      \item Adverbial and prepositional phrases (time, place and frequency).
      \item Connector (effect).
   \end{topics}

   \begin{learningoutcomes}
      \item Discuss one’s daily routine and habits.
      \item Express frequency on everyday activities.
   \end{learningoutcomes}
\end{unit}

\begin{unit}{Can you speak English?}{}{de2002marco}{15}{C25}
   \begin{topics}
      \item Design by the teacher.
      \item Connector (effect).
   \end{topics}

   \begin{learningoutcomes}
      \item Get familiar with technical expressions of the field.
   \end{learningoutcomes}
\end{unit}

\begin{coursebibliography}
\bibfile{ForeignLanguages/ID101}
\end{coursebibliography}

\end{syllabus}
%\end{document}
