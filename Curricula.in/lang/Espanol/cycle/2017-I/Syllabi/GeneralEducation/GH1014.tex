\begin{syllabus}

\course{GH1014. Culturas de gobernanza y distribución de poder}{Obligatorio}{GH1014} % Common.pm

\begin{justification}
El objetivo del curso es que el estudiante comprenda la interrelación que existe entre los sistemas políticos y económicos de un país o región. El hilo conductor de este curso será el libro "Why Nations Fail: The Origins of Power, Prosperity, and Poverty" de Acemoglu-Robinson. El aprendizaje del curso debe ser una interpretación informada de distintas dinámicas sociales en las que se organiza y reparte poder, sea de carácter simbólico, económico y/o político.  
Este curso debe trabajar la capacidad del estudiante de utilizar conceptos más complejos y desarrollar interpretaciones más elaboradas de la realidad.
\end{justification}

\begin{goals}
\item Capacidad de interpretar información.
\item Capacidad para formular alternativas de solución.
\item Capacidad de comprender textos
\end{goals}

\begin{outcomes}
    \item \ShowOutcome{d}{2}
    \item \ShowOutcome{e}{2}
    \item \ShowOutcome{n}{2}
    
\end{outcomes}

\begin{competences}
    \item \ShowCompetence{C10}{d,n}
    \item \ShowCompetence{C17}{d}
    \item \ShowCompetence{C18}{n}
    \item \ShowCompetence{C21}{e}
\end{competences}

\begin{unit}{Culturas de Gobernanza y Distribución de Poder}{}{Lessig15}{12}{4}
   \begin{topics}
      \item ?`Cómo se relaciona la economía con la política?.
      \item El rol de las Instituciones.
      \item Análisis de casos.
   \end{topics}
   \begin{learningoutcomes}
      \item Desarrollo del innterés por conocer sobre temas actuales en la sociedad peruana y el mundo.
   \end{learningoutcomes}
\end{unit}



\begin{coursebibliography}
\bibfile{GeneralEducation/GH1014}
\end{coursebibliography}

\end{syllabus}
