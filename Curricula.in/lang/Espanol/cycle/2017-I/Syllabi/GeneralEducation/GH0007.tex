\begin{syllabus}

\course{CB101. �lgebra y Geometr�a}{Obligatorio}{CB101}

\begin{justification}
Este curso tiene como objetivo proporcionar a los estudiantes una experiencia pr�ctica de la vida  real en los  primeros pasos dentro de un ciclo de vida de negocios, a trav�s del cual una idea se transforma en un modelo de negocio formal.
Es el primero de un conjunto de tres cursos dise�ados para acompa�ar a los estudiantes a medida que transforman una idea en un negocio o negocio prospectivo, desde la idea  hasta la revisi�n de la estrategia empresarial actual.
\end{justification}

\begin{goals}
  \item Capacidad n�lisis de la informaci�n.
� \item Interpretaci�n de informaci�n y resultados.
  \item Capacidad de Trabajo en equipo.
  \item �tica.
  \item Comunicaci�n oral.
  \item Comunicaci�n escrita.
  \item Comunicaci�n gr�fica.
  \item Entiender la necesidad de aprender de forma continua.
\end{goals}

\begin{outcomes}
\ExpandOutcome{a}{3}
\ExpandOutcome{i}{2}
\ExpandOutcome{j}{4}
\end{outcomes}

\begin{unit}{El ciclo de vida empresarial: desde la idea hasta la revisi�n de su estrategia.}{Lehmann05}{12}{4}
   \begin{topics}
      \item 
      \item 
   \end{topics}
   \begin{unitgoals}
      \item 
   \end{unitgoals}
\end{unit}

\begin{unit}{El proceso de ideaci�n y la visi�n del cliente}{Lehmann05}{24}{3}
   \begin{topics}
      \item 
      \item 
   \end{topics}

   \begin{unitgoals}
      \item 
      \item
      \item 
      \end{unitgoals}
\end{unit}

\begin{unit}{�C�mo construir y mantener equipos eficaces?}{Strang03,Grossman96}{24}{3}
   \begin{topics}
      \item 
      \item 
      \item 
      \end{topics}

   \begin{unitgoals}
      \item 
      \item 
      \item 
     
   \end{unitgoals}
\end{unit}

\begin{unit}{Ejecuci�n de LEAN: lo b�sico}{Grossman96}{30}{3}
   \begin{topics}
      \item 
      \item 
   \end{topics}

   \begin{unitgoals}
      \item 
      \item 
      \item 
   \end{unitgoals}
\end{unit}

\begin{unit}{Dise�o de un modelo de negocio: herramientas de dise�o y lienzo.}{Grossman96}{30}{3}
   \begin{topics}
      \item 
      \item 
   \end{topics}

   \begin{unitgoals}
      \item 
      \item 
      \item 
   \end{unitgoals}
\end{unit}

\begin{unit}{Generaci�n de Modelos de Negocio: la Lona Modelo de Negocio (Osterwalder).}{Grossman96}{30}{3}
   \begin{topics}
      \item 
      \item 
   \end{topics}

   \begin{unitgoals}
      \item 
      \item 
      \item 
   \end{unitgoals}
\end{unit}

\begin{unit}{Ingenier�a de Venture: usar las habilidades de la inform�tica para construir un modelo de negocio efectivo.}{Grossman96}{30}{3}
   \begin{topics}
      \item 
      \item 
   \end{topics}

   \begin{unitgoals}
      \item 
      \item 
      \item 
   \end{unitgoals}
\end{unit}

\begin{unit}{Herramientas de investigaci�n de mercado primario y nichos de mercado.}{Grossman96}{30}{3}
   \begin{topics}
      \item 
      \item 
   \end{topics}

   \begin{unitgoals}
      \item 
      \item 
      \item 
   \end{unitgoals}
\end{unit}

\begin{unit}{La Importancia del Capital: Humano, Financiero e Intelectual.}{Grossman96}{30}{3}
   \begin{topics}
      \item 
      \item 
   \end{topics}

   \begin{unitgoals}
      \item 
      \item 
      \item 
   \end{unitgoals}
\end{unit}

\begin{unit}{T�cnicas de monetizaci�n y financiamiento.}{Grossman96}{30}{3}
   \begin{topics}
      \item 
      \item 
   \end{topics}

   \begin{unitgoals}
      \item 
      \item 
      \item 
   \end{unitgoals}
\end{unit}


\begin{unit}{Comunicaci�n eficaz: crear una presentaci�n de un modelo de negocio de impacto.}{Grossman96}{30}{3}
   \begin{topics}
      \item 
      \item 
   \end{topics}

   \begin{unitgoals}
      \item 
      \item 
      \item 
   \end{unitgoals}
\end{unit}
\begin{coursebibliography}
\bibfile{GeneralEducation/FG170}
\end{coursebibliography}

\end{syllabus}
