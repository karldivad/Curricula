\begin{syllabus}

\course{CS1701107. Estructuras Discretas I}{Obligatorio}{CS1701107} % Common.pm

\begin{justification}
Las estructuras discretas proporcionan los fundamentos teóricos necesarios para la computación. Estos fundamentos no sólo son útiles para desarrollar la computación desde un punto de vista teórico como sucede
En el curso de la teoría computacional, pero también es útil para la práctica de la informática; En particular en aplicaciones tales como verificación,
Criptografía, métodos formales, etc.
\end{justification}

\begin{goals}
\item Aplicar Correctamente conceptos de matemáticas finitas (conjuntos, relaciones, funciones) para representar datos de problemas reales.
\item Modelar situaciones reales descritas en lenguaje natural, usando lógica proposicional y lógica predicada.
\item Determinar las propiedades abstractas de las relaciones binarias.
\item Elegir el método de demostración más apropiado para determinar la veracidad de una propuesta y construir argumentos matemáticos correctos.
\item Interpretar soluciones matemáticas a un problema y determinar su fiabilidad, ventajas y desventajas.
\item Expresar el funcionamiento de un circuito electrónico simple usando álgebra booleana.
\end{goals}

\begin{outcomes}{V1}
    \item \ShowOutcome{a}{2}
    \item \ShowOutcome{j}{2}
\end{outcomes}

\begin{outcomes}{V2}
    \item \ShowOutcome{1}{2}
    \item \ShowOutcome{6}{2}
\end{outcomes}

\begin{competences}{V1}
    \item \ShowCompetence{C1}{a}
    \item \ShowCompetence{C20}{j}
\end{competences}

\begin{competences}{V2}
    \item \ShowCompetence{C1}{1}
    \item \ShowCompetence{C20}{6}
\end{competences}

\begin{unit}{\DSSetsRelationsandFunctions}{}{Grimaldi03,Rosen2007}{22}{C1,C20}
   \begin{topics}
        \item \DSSetsRelationsandFunctionsTopicSets
        %\item \DSSetsRelationsandFunctionsTopicRelations
        \item Relaciones:
        \begin{subtopics}
            \item Reflexividad, simetria, transitividad
            \item Relaciones de equivalencia
            \item Relación de orden parcial y conjuntos parcialmente ordenados
            \item Elementos extremos de un conjunto parcialmente ordenado
        \end{subtopics}
        \item \DSSetsRelationsandFunctionsTopicFunctions
   \end{topics}
   \begin{learningoutcomes}
	\item \DSSetsRelationsandFunctionsLOExplainWith [\Assessment]
	\item \DSSetsRelationsandFunctionsLOPerformThe [\Assessment]
	\item \DSSetsRelationsandFunctionsLORelate [\Assessment]
   \end{learningoutcomes}
 \end{unit}

 \begin{unit}{\DSBasicLogic}{}{Rosen2007,Grimaldi03}{14}{C1,C20}
   \begin{topics}
        \item \DSBasicLogicTopicPropositional%
        \item \DSBasicLogicTopicLogical%
        \item \DSBasicLogicTopicTruth%
        \item \DSBasicLogicTopicNormal%
        \item \DSBasicLogicTopicValidity%
        \item \DSBasicLogicTopicPropositionalInference%
        \item \DSBasicLogicTopicPredicate%
        \item \DSBasicLogicTopicLimitations%
   \end{topics}
   \begin{learningoutcomes}
	\item \DSBasicLogicLOConvertLogical [\Usage ]
	\item \DSBasicLogicLOApplyFormal [\Usage ]
	\item \DSBasicLogicLOUseThe [\Usage]
	\item \DSBasicLogicLODescribeHowCan [\Familiarity]
	\item \DSBasicLogicLOApplyFormalAnd [\Usage ]
	\item \DSBasicLogicLODescribeTheLimitationsAnd [\Usage]
   \end{learningoutcomes}
 \end{unit}

\begin{unit}{\DSProofTechniques}{}{Rosen2007, Epp10, Scheinerman12}{14}{C1,C20}
\begin{topics}
        \item \DSProofTechniquesTopicNotions%
        \item \DSProofTechniquesTopicThe%
        \item \DSProofTechniquesTopicDirect%
        \item \DSProofTechniquesTopicDisproving%
        \item \DSProofTechniquesTopicProof%s
        \item \DSProofTechniquesTopicInduction%
        \item \DSProofTechniquesTopicStructural%
        \item \DSProofTechniquesTopicWeak%
        \item \DSProofTechniquesTopicRecursive%
        \item \DSProofTechniquesTopicWell%
\end{topics}

\begin{learningoutcomes}
    %% itemizar cada learning outcomes [nivel segun el curso]
	\item \DSProofTechniquesLOIdentifyTheUsed [\Assessment]
	\item \DSProofTechniquesLOOutline [\Usage ]
	\item \DSProofTechniquesLOApplyEach [\Usage ]
	\item \DSProofTechniquesLODetermineWhich [\Assessment]
	\item \DSProofTechniquesLOExplainTheIdeas [\Familiarity ]
	\item \DSProofTechniquesLOExplainTheWeak [\Assessment]
	\item \DSProofTechniquesLOStateThe [\Familiarity]
\end{learningoutcomes}
\end{unit}

\begin{unit}{Representación de Datos}{}{Rosen2007,Grimaldi03}{10}{C1,C20}
\begin{topics}
    \item Representaciones numéricas: signo magnitud, punto flotante.
    \item Representaciones de otros objetos: conjuntos, relaciones, funciones
\end{topics}

\begin{learningoutcomes}
    \item Conocer las formas de representación numérica como signo magnitud y punto flotante. [\Assessment].
    \item Llevar a cabo operaciones aritméticas utilizando las distintas formas de representación. [\Assessment].
    \item Conocer el estándar de punto flotante IEEE-754 [\Familiarity].
\end{learningoutcomes}
 \end{unit}

\begin{coursebibliography}
\bibfile{Computing/CS/CS1D1}
\end{coursebibliography}

\end{syllabus}

%\end{document}
