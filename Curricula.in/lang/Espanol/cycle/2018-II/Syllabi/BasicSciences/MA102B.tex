\begin{syllabus}

\course{MA102B. Cálculo Integral}{Obligatorio}{MA102B} % Common.pm

\begin{justification}
El curso de Cálculo Integral contribuirá a la adquisición de la competencia Aprender a aprender, en relación con el perfil de egresado de Estudios Generales Ciencias y con el objetivo (a) de ABET ``Capacidad de aplicar conocimientos de matemáticas, ciencias e ingeniería'' en la medida en que al finalizar la asignatura el estudiante será capaz de:
\end{justification}

\begin{goals}
\item Describir la integral definida como un límite de sumas de Riemann, e interpretarla como área bajo una curva con el fin de hallar su valor sin recurrir a métodos de integración.
\item Aplicar el primer y segundo teoremas fundamentales del cálculo y propiedades sobre integrales definidas de manera natural y eficiente, con el fin de aprender la manipulación algebraica correcta de las integrales definidas.
\item Resolver ecuaciones diferenciales lineales ordinarias de orden arbitrario con coeficientes constantes en contextos intra y extramatema?ticos, con el fin de apreciar la utilidad del proceso de integración en la resolución de problemas de modelado matemático en las Ciencias e Ingeniería.
\item Construir integrales definidas para el cálculo de áreas y centroides de regiones planas, así como en el cálculo de áreas y volúmenes de superficies y sólidos de revolución, respectivamente, representando gráficamente dichas regiones y sólidos, para su uso en la resolución de problemas geométricos y físicos.
\item Aproximar valores funcionales mediante el uso de polinomios de Taylor, obteniendo estimaciones razonables por medio del cálculo de la fórmula del error, y comprobar la validez de dicha aproximación vía calculadora, con el fin de apreciar que muchas veces el cálculo exacto puede ser reemplazado por su aproximación.
\item Aplicar los conceptos y procedimientos propios de las coordenadas polares, y gráfica de ecuaciones en coordenadas polares, como un método alternativo a las coordenadas cartesianas para resolver problemas de medición de longitudes y cálculo de áreas.

También se favorecerá el desarrollo de la competencia desempeño personal y académico, en particular, en lo que se refiere al desarrollo del pensamiento crítico cuando el estudiante sea capaz de:
\item Analizar la validez de afirmaciones sobre integrales definidas en base a su definición y propiedades dadas en clase, de modo que presente por escrito una demostración formal en caso sean verdaderas y un contraejemplo en caso contrario, para fortalecer el desarrollo del pensamiento lógico.
Además, se contribuirá con el desarrollo de la competencia de comunicación lo que se refiere al uso del lenguaje científico, en relación con el perfil de egresado de Estudios Generales Ciencias y con el objetivo g) de ABET ``Capacidad para comunicarse eficazmente'' ya que el estudiante será capaz de:
\item Analizar la convergencia de las integrales impropias, justificando sus conclusiones a partir de los conceptos estudiados y mediante el uso correcto del lenguaje simbólico, con la finalidad de desarrollar su capacidad de análisis y el uso correcto de este tipo de integrales en cursos posteriores.
\end{goals}

\begin{outcomes}
\item \ShowOutcome{d}{2}
\item \ShowOutcome{h}{2}
\end{outcomes}

\begin{competences}
    \item \ShowCompetence{C20}{d,h}
\end{competences}

\begin{unit}{Capítulo 1: La integral definida (10 horas)}{}{}{4}{C20}

% Descripción general de la unidad.
% 
% El estudio del concepto y las propiedades de la integral definida es fundamental, para poder luego utilizarlo en las aplicaciones a las ciencias e ingeniería. Se desarrolla, asimismo, mediante los teoremas fundamentales del cálculo, las relaciones de este nuevo concepto con los de derivada e integral indefinida.
% Contenidos

\begin{topics}
      \item Notación Sigma (Sumas). 
      \item Fórmulas de algunas sumas especiales.
      \item Idea intuitiva del cálculo del área de una región plana limitada por la gráfica de una función no negativa en un intervalo acotado.
      \item Partición regular.
      \item Sumas de Riemann.
      \item La integral definida como límite de una suma de Riemann. 
      \item Propiedades de la integral definida.
      \item Definiciones de función acotada y continua por tramos (seccionalmente continua).
      \item El teorema del valor medio.
      \item Interpretación geométrica y demostración. 
      \item Definición de valor promedio de una función en un intervalo cerrado.
      \item Teoremas fundamentales del cálculo.      
   \end{topics}

   \begin{learningoutcomes}
      \item .
   \end{learningoutcomes}
\end{unit}

\begin{unit}{Capítulo 2: Métodos de integración (10 horas)}{}{}{4}{C20}

% Descripción general de la unidad
% Las aplicaciones de la integral, tales como el cálculo de áreas, volúmenes, centroides, etc. requieren la construcción de antiderivadas. En esta unidad se trabajan diversas técnicas para lograr dicha finalidad.

\begin{topics}
      \item Integración por partes para integrales definidas.
      \item Demostración de integrales definidas por recurrencia. 
      \item Integrales trigonométricas: potencias de seno por coseno, tangente por secante y productos de seno y coseno con ángulos diferentes.
      \item Integración por sustitución trigonométrica. 
      \item Aplicación a integrales con potencias fraccionarias.
      \item Integración de funciones racionales con el método de fracciones parciales
\end{topics}
   
\begin{learningoutcomes}
      \item .
\end{learningoutcomes}
\end{unit}

\begin{unit}{Capítulo 3: Aplicaciones de la integral (22 horas)}{}{}{4}{C20}

% Descripción general de la unidad
% Se desarrollan los métodos para calcular áreas, volúmenes, longitudes de curvas y centroides, que se necesitan para resolver problemas de física e ingeniería. Se aborda también el estudio de las coordenadas polares, y los polinomios de Taylor como herramienta para la obtención de valores aproximados de una integral definida.

\begin{topics}
      \item Área entre dos curvas en coordenadas cartesianas. 
      \item Cálculo de volúmenes de sólidos mediante secciones transversales. 
      \item Cálculo de volúmenes de sólidos de revolución: método del disco y del anillo.
      \item Método de las cortezas cilíndricas.
      \item Longitud de arco de una curva. 
      \item Área de la superficie de revolución generada por una curva. 
      \item Momentos y centro de masa. 
      \item Centroide de una región plana. 
      \item Teorema de Pappus para hallar el volumen de un sólido de revolución.
      \item Coordenadas polares. 
      \item Relaciones entre coordenadas cartesianas y polares. 
      \item Ecuaciones polares y ecuaciones equivalentes. 
      \item Gráfica de ecuaciones en coordenadas polares.
      \item Área de regiones plana y longitud de arco en coordenadas polares. 
      \item La fórmula de Taylor con resto de Lagrange. 
      \item Aproximación de cfunciones y Cálculo aproximado de integrales mediante polinomios de 				Taylor.
\end{topics}

   \begin{learningoutcomes}
      \item .
   \end{learningoutcomes}
\end{unit}

\begin{unit}{Capítulo 4: Ecuaciones diferenciales (8 horas)}{}{}{6}{C20}

% Descripción general de la unidad
% Se continúa y profundiza el tema de ecuaciones diferenciales, iniciado en el curso anterior, con el estudio de las ecuaciones diferenciales lineales de orden n; motivado este por sus aplicaciones en mecánica y electricidad.

\begin{topics}
      \item Definición de EDO lineal de orden n. 
      \item Caso particular: EDO lineal de orden 2.
      \item Solución de ecuaciones lineales no homogéneas lineales con coeficientes constantes.
      \item Teorema de existencia y unicidad para EDOs de orden $n$.
      \item El método de reducción de orden. Independencia lineal y wronskiano.
      \item Obtención de una solución particular de la ecuación no homogénea.
      \item El método de los coeficientes indeterminados y el de variación de parámetros. 
      \item Aplicaciones: Movimientos vibratorios de sistemas mecánicos. Movimiento armónico 				simple. 
      \item  Movimiento vibratorio con amortiguamiento.
      \item El fenómeno de resonancia. Problemas de circuitos eléctricos y el péndulo simple.
   \end{topics}

   \begin{learningoutcomes}
      \item .
   \end{learningoutcomes}
\end{unit}

\begin{unit}{Capítulo 5: Integrales impropias (6 horas)}{}{}{4}{C20}

% Descripción general de la unidad
% Se continúa y profundiza el tema de ecuaciones diferenciales, iniciado en el curso anterior, con el estudio de las ecuaciones diferenciales lineales de orden n; motivado este por sus aplicaciones en mecánica y electricidad.

\begin{topics}
	\item Definición de integrales impropias (en intervalos no acotados y con asíntotas verticales). 
    \item Criterios de convergencia: comparación, paso al límite y convergencia absoluta. 
    \item Ejemplos de convergencia de integrales impropias que dependan de un parámetro
   \end{topics}

   \begin{learningoutcomes}
      \item .
      \item .
      \item .
   \end{learningoutcomes}

\end{unit}



\begin{coursebibliography}
\bibfile{BasicSciences/MA102B}
\end{coursebibliography}

% APOSTOL, Tom
% 1973	Calculus. Cálculo con funciones de una variable. Barcelona: Editorial Reverté.
% Enlace permanente al catálogo de biblioteca
% https://pucp.ent.sirsi.net/client/es_ES/campus/search/detailnonmodal/ent:$002f$002fSD_ILS$002f0$002fSD_ILS:11626/one
?
% CENGEL, Yunus. A. y William J. PALM
% 2014	Ecuaciones diferenciales para Ingeniería y Ciencias. México: McGraw-Hill Education.
% Enlace permanente al catálogo de biblioteca
% https://pucp.ent.sirsi.net/client/es_ES/campus/search/detailnonmodal/ent:$002f$002fSD_ILS$002f0$002fSD_ILS:590704/one
% EDWARDS, Ch. y D. E. PENNEY
% 2008	Ecuaciones diferenciales y problemas con valores en la frontera: cómputo y modelado. Cuarta edición. Naucalpan de Juárez: Pearson.
% http://www.ingebook.com.ezproxybib.pucp.edu.pe:2048/ib/NPcd/IB_BooksVis?cod_primaria=1000187&codigo_libro=1281
% Enlace permanente al catálogo de biblioteca
% https://pucp.ent.sirsi.net/client/es_ES/campus/search/detailnonmodal/ent:$002f$002fSD_ILS$002f0$002fSD_ILS:552189/one
% KONG, Maynard
% 2004	Cálculo Integral. Lima: Pontificia Universidad Católica del Perú. Fondo Editorial.
% http://repositorio.pucp.edu.pe/index/handle/123456789/54974
% Enlace permanente al catálogo de biblioteca
% https://pucp.ent.sirsi.net/client/es_ES/campus/search/detailnonmodal/ent:$002f$002fSD_ILS$002f0$002fSD_ILS:371291/one
% LEITHOLD, Louis.
% 1998	El Cálculo. Sétima edición. México: Oxford University Press.
% Enlace permanente al catálogo de biblioteca
% https://pucp.ent.sirsi.net/client/es_ES/campus/search/detailnonmodal/ent:$002f$002fSD_ILS$002f0$002fSD_ILS:264155/one
% STEWART, James, María del Carmen Rodríguez Pedroza y Ernesto Filio Lo?pez
% 2012	Cálculo de una Variable: trascendentes tempranas. Sétima edición. México: Cengage Learning.
% http://ezproxybib.pucp.edu.pe:2048/login?url=http://www.ebooks7-24.com/?il=787
% Enlace permanente al catálogo de biblioteca
% https://pucp.ent.sirsi.net/client/es_ES/campus/search/detailnonmodal/ent:$002f$002fSD_ILS$002f0$002fSD_ILS:577768/one
% ZILL, D. y Warren S. WRIGHT
% 2011	Cálculo: trascendentes tempranas. Cuarta edición. México: McGraw-Hill Interamericana.
% http://ezproxybib.pucp.edu.pe:2048/login?url=http://www.ebooks7-24.com/?il=626
% Enlace permanente al catálogo de biblioteca
% https://pucp.ent.sirsi.net/client/es_ES/campus/search/detailnonmodal/ent:$002f$002fSD_ILS$002f0$002fSD_ILS:576506/one

\end{syllabus}
