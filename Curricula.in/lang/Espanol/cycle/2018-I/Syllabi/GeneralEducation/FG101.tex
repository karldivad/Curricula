\begin{syllabus}

\course{. }{}{} % Common.pm

\begin{justification}
Para lograr una eficaz comunicación en el ámbito personal y profesional, es prioritario el manejo adecuado de la Lengua en forma oral y escrita. Se justifica, por lo tanto, que los alumnos de la Universidad Católica San Pablo conozcan, comprendan y apliquen los aspectos conceptuales y operativos de su idioma, para el desarrollo de sus habilidades comunicativas fundamentales: Escuchar, hablar, leer y escribir.
En consecuencia el ejercicio permanente y el aporte de los fundamentos contribuyen grandemente en la formación académica y, en el futuro, en el desempeño de su profesión
\end{justification}

\begin{goals}
\item Desarrollar capacidades comunicativas a través de la teoría y práctica del lenguaje que ayuden al estudiante a superar las exigencias académicas del pregrado y contribuyan a su formación humanística y como persona humana.
\end{goals}

\begin{outcomes}
   \item \ShowOutcome{f}{2}
   \item \ShowOutcome{h}{2}
   \item \ShowOutcome{n}{2}
\end{outcomes}

\begin{competences}
    \item \ShowCompetence{C17}{f,h,n}
    \item \ShowCompetence{C20}{f,n}
    \item \ShowCompetence{C24}{f,h}
\end{competences}

\begin{unit}{Primera Unidad}{}{Real}{16}{C17,C20}
\begin{topics}
      \item La comunicación, definición, relevancia. Elementos. Proceso. Funciones. Clasificación.Comunicación oral y escrita.
      \item El lenguaje: definición. Características y funciones. Lengua: niveles. Sistema. Norma. Habla. El signo lingüístico: definición, características.
      \item Multilingüismo en el Perú. Variaciones dialectales en el Perú.
      \item La palabra: definición, clases y estructura. Los monemas: lexema y morfema. El morfema: clases. La etimología.
      \item El Artículo académico: Definición, estructura, elección del tema, delimitación del tema.
\end{topics}

\begin{learningoutcomes}
   \item Reconocer y valorar la comunicación como un proceso de comprensión e intercambio de mensajes, diferenciando sus elementos, funciones y clasificación [\Usage].
   \item Analizar las características, funciones y elementos del lenguaje y de la lengua [\Usage].
   \item Identificar las características del multilingüismo en el Perú, valorando su riqueza idiomática [\Usage].
   \item Identificar las cualidades de la palabra y sus clases [\Usage].
\end{learningoutcomes}
\end{unit}



\begin{coursebibliography}
\bibfile{GeneralEducation/FG101}
\end{coursebibliography}

\end{syllabus}
