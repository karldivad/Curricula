\begin{syllabus}

\course{GH0012. Economías en Desarrollo}{Obligatorio}{GH0012} % Common.pm

\begin{justification}
Este curso busca introducir al estudiante a los conceptos generales de microeconomía y macroeconomía. El objetivo es que los estudiantes puedan explicar procesos de la realidad desde la lógica de la economía. Después de haber llevado los módulos de micro y macroeconomía, los estudiantes deben elegir uno de los dos tracks electivos propuestos.
Los tracks electivos son i) Casos de economías de rápido crecimiento y de dramáticas recesiones y ii) políticas públicas para la reducción de la pobreza en América Latina.
\end{justification}

\begin{goals}
\item Capacidad de interpretar información.
\item Capacidad para formular alternativas de solución.
\item Capacidad de comprender textos.
\end{goals}

\begin{outcomes}{V1}
    \item \ShowOutcome{d}{2} % Multidiscip teams
    \item \ShowOutcome{e}{2} % ethical, legal, security and social implications
    \item \ShowOutcome{f}{2} % communicate effectively
    \item \ShowOutcome{n}{2} % Apply knowledge of the humanities

\end{outcomes}

\begin{competences}{V1}
    \item \ShowCompetence{C10}{d,n}
    \item \ShowCompetence{C17}{f}
    \item \ShowCompetence{C18}{f}
    \item \ShowCompetence{C21}{e}
\end{competences}

\begin{unit}{Economías en Desarrollo}{}{Gregory02}{12}{4}
   \begin{topics}
      \item Microeconomía.
      \item Macroeconomía.
      \item Casos de Economías de rápido crecimiento de dramáticas recesiones.
      \item Políticas públicas para la reducción de la pobreza en América Latina.
   \end{topics}
   \begin{learningoutcomes}
      \item Desarrollo del innterés por conocer sobre temas actuales en la sociedad peruana y el mundo
   \end{learningoutcomes}
\end{unit}

\begin{coursebibliography}
\bibfile{GeneralEducation/GH2012}
\end{coursebibliography}

\end{syllabus}
