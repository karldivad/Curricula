\begin{syllabus}

\course{GH0005. Laboratorio de Comunicación I}{Obligatorio}{GH0005} % Common.pm

\begin{justification}
A través de este curso, el alumno mejorará y fortalecerá sus capacidades para comunicarse tanto a nivel oral como escrito en un contexto académico. Para ello, el alumno se ejercitará en la composición de textos, tomando en cuenta las exigencias propias de un lenguaje formal académico: características de la redacción académica (reglas de puntuación, ortografía, competencia léxico gramatical, normativa) y empleo correcto de la información. A su vez, el curso promueve una lectura comprensiva que no se limita al nivel descriptivo, sino que abarca también lo conceptual y metafórico, pues solo de ese modo el estudiante desarrollará su capacidad crítica y analítica. El estudiante afrontará lecturas académicas y de divulgación científica que le permitirán distinguir los objetivos planteados en los distintos tipos de textos, y reconocer al texto oral y escrito como una unidad coherente y cohesionada en cuanto a forma y contenido. Alcanzados estos objetivos, el estudiante comprenderá que las habilidades comunicativas orales y escritas son competencias centrales de su vida universitaria y, posteriormente, de su vida profesional. 
\end{justification}

\begin{goals}
\item Con este curso el estudiante desarrolla y fortalece sus habilidades comunicativas orales y escritas en el marco de un contexto académico. Además, comprende conceptual y metafóricamente textos expositivos, e identifica los objetivos, jerarquía de las ideas y estructura de dichos textos. Al finalizar el curso, el estudiante es capaz de producir textos expositivos descriptivos e informativos. Así mismo, desarrolla su capacidad de apertura y tolerancia hacia la diversidad de puntos de vista gracias al continuo trabajo grupal, autoevaluaciones y evaluaciones de pares que enfrentará a lo largo del ciclo en el curso. 
\end{goals}

\begin{outcomes}{V1}
     \item \ShowOutcome{e}{2}
     \item \ShowOutcome{f}{2}
     \item \ShowOutcome{i}{2}
     \item \ShowOutcome{n}{2}
\end{outcomes}

\begin{competences}{V1}
    \item \ShowCompetence{C17}{f,h,n}
    \item \ShowCompetence{C20}{f,n}
    \item \ShowCompetence{C24}{f,h}
\end{competences}

\begin{unit}{Laboratorio de Comunicación I}{}{Cassany93}{12}{4}
   \begin{topics}
      \item Características de Escritura Académica.
      \item Estrategias de Lectura.
      \item Estructura del texto.
      \item Estructura de párrafos.
      \item Características del párrafo.
      \item Texto argumentativo Vs. Texto expositivo.
      \item Proceso de Redacción.
      \item Citas:función y tipos -Bibiliografía.
      \item Aproximación a características de la exposición oral.
      \item Conferencia :caracterpisticas exposición formal.
      \item Redacción de texto completo con citas.  
   \end{topics}
   \begin{learningoutcomes}
      \item Auto-evaluación: el estudiante es capaz de reconocer sus propias fortalezas y deficiencias al formular críticas constructivas sobre su propio trabajo.
   \end{learningoutcomes}
\end{unit}

\begin{coursebibliography}
\bibfile{GeneralEducation/GH1005}
\end{coursebibliography}

\end{syllabus}
