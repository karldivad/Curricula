\begin{syllabus}

\course{ET302. Formación de Empresas de Base Tecnológica III}{Obligatorio}{ET302}

\begin{justification}
Este curso dentro del área formación de empresas de base tecnológica, 
pretende abordar todos los procesos y buenas prácticas en la 
gestión de proyectos recomendadas por el \textit{Project Management Institute} (PMI) 
contenidas en el \textit{Project Management Body of Knowledge 2012} (PMBOK)  
aplicado en particular a proyectos de base tecnológica como pueden ser la 
construcción, desarrollo, integración e implementación de soluciones de 
software de aplicación.

El futuro profesional que pretenda incursionar con una empresa de 
software en el competitivo mercado globalizado, debe necesariamente 
conocer las habilidades duras y practicar las habilidades blandas que se 
consideran en el PMBOK. Todos los contratos de suministro de bienes 
tangibles (Hardware) o intangibles (Software) así como los servicios de 
consultoría deben ser manejados como pequeños proyectos.

Creemos de suma importancia impartir los fundamentos y experiencias 
asociadas a la dirección de proyectos a los futuros profesionales, 
debemos considerar que en la actualidad las empresas cliente 
(nacionales o internacionales) que demandan soluciones exigen a 
las empresas de consultoría se lleve a cabo los proyectos de sistemas 
de información y tecnología de información con los estándares del PMI, 
cada vez mas resulta ser una condición de exigibilidad para poder ganar 
licitaciones y firmar contratos de suministro de soluciones de tecnología, 
asimismo se exige que el jefe del proyecto, adicionalmente a su formación y 
experiencia para llevar a buen puerto el proyecto sea un PMP.
\end{justification}

\begin{goals}
\item Que el alumno domine los conceptos relacionados a la gestión de proyectos informáticos.
\item Proporcionar al alumno las técnicas y herramientas que le permitan gestionar con éxito proyectos de diversas magnitudes.
\item Que el alumno construya su plan de negocios orientado a conseguir un inversionista internacional que pueda impulsar y proyectar a la empresa a un ámbito internacional.
\end{goals}

%% (1) familiar  (2)usar (3)evaluar
\begin{outcomes}
    \item \ShowOutcome{d}{2}
    \item \ShowOutcome{f}{2}
    \item \ShowOutcome{m}{3}
\end{outcomes}

\begin{competences}
    \item \ShowCompetence{C17}{f} 
    \item \ShowCompetence{C18}{d}
    \item \ShowCompetence{C19}{m}
    \item \ShowCompetence{C20}{m}
    \item \ShowCompetence{C21}{m}
    \item \ShowCompetence{C22}{m}
    \item \ShowCompetence{C23}{m}
    \item \ShowCompetence{C24}{m}
\end{competences}

%% Nivel = 1(Familiarity),  2(Usage),  3(Assessment) 
\begin{unit}{Marco Conceptual de la Dirección de Proyectos}{}{pmbo08,pmep09}{15}{C19}
\begin{topics}
      \item Introducción
       \item Finalidad de la guía del PMBOK, ?`Qué es un proyecto?, ?`Qué es la dirección de proyectos?, La estructura de la guía del PMBOK, Áreas de experiencia, contexto de la dirección de proyectos
      \item Ciclo de Vida del Proyecto y Organización
       \item Ciclo de vida del proyecto, interesados en el proyecto, influencias de la organización
   \end{topics}

   \begin{learningoutcomes}
      \item Conocer el marco conceptual en el que se desarrollan los proyectos. [\Usage]
   \end{learningoutcomes}
\end{unit}

\begin{unit}{Norma para la dirección de un proyecto}{}{pmbo08,pmep09}{15}{C20}
\begin{topics}
      \item Procesos de Dirección de Proyectos para un Proyecto
       \item Procesos de dirección de proyectos, grupos de procesos de dirección de proyectos, grupos de procesos de dirección de proyectos, interacciones entre procesos, correspondencia de los procesos de dirección de proyectos
   \end{topics}

   \begin{learningoutcomes}
      \item Conocer los estándares de gestión de proyectos aplicado a proyectos. [\Usage]
   \end{learningoutcomes}
\end{unit}

\begin{unit}{Áreas de conocimiento de la dirección de proyectos}{}{pmbo08,pmep09}{60}{C23}
\begin{topics}
      \item Introducción
      \item Gestión de la Integración del Proyecto
      \item Gestión del Alcance del Proyecto
      \item Gestión del Tiempo del Proyecto
      \item Gestión de los Costes del Proyecto
      \item Gestión de la Calidad del Proyecto
      \item Gestión de los Recursos Humanos del Proyecto
      \item Gestión de las Comunicaciones del Proyecto
      \item Gestión de los Riesgos del Proyecto
      \item Gestión de las Adquisiciones del Proyecto
   \end{topics}

   \begin{learningoutcomes}
      \item Entender la naturaleza de la gerencia de proyectos y su importancia para lograr el éxito en los proyectos. [\Assessment]
      \item Adquirir el conocimiento necesario para gestionar proyectos de manera exitosa en terminos de: Tiempo, Costos, Alcance, Riesgos, Calidad, RRHH, Procura, Comunicaciones e Integración. [\Usage]
       \item Valorar la importancia de una buena Gerencia de Proyectos. [\Assessment]
      \item Demostrar competencias para la realización de presentaciones efectivas. [\Usage]
      \item Desarrollar habilidades para gestionar equipos de trabajo multidisciplinarios. [\Usage]
   \end{learningoutcomes}
\end{unit}




\begin{coursebibliography}
\bibfile{Enterpreneurship/ET302}
\end{coursebibliography}

\end{syllabus}
