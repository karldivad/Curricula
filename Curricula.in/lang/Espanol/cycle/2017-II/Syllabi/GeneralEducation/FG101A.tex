\begin{syllabus}

\course{FG101A. Comunicación Oral y Escrita I}{Obligatorio}{FG101A} % Common.pm

\begin{justification}
Para lograr una eficaz comunicación en el ámbito personal y profesional, es prioritario el manejo adecuado de la Lengua en forma oral y escrita. Se justifica, por lo tanto, que los alumnos  conozcan, comprendan y apliquen los aspectos conceptuales y operativos de su idioma, para el desarrollo de sus habilidades comunicativas fundamentales: Escuchar, hablar, leer y escribir.
En consecuencia el ejercicio permanente y el aporte de los fundamentos contribuyen grandemente en la formación académica y, en el futuro, en el desempeño de su profesión
\end{justification}

\begin{goals}
\item Desarrollar capacidades comunicativas a través de la teoría y práctica del lenguaje que ayuden al estudiante a superar las exigencias académicas del pregrado y contribuyan a su formación humanística y como persona humana.
\end{goals}

\begin{outcomes}
   \item \ShowOutcome{f}{2}
   \item \ShowOutcome{h}{2}
   \item \ShowOutcome{n}{2}
\end{outcomes}

\begin{competences}
    \item \ShowCompetence{C17}{f,h,n}
    \item \ShowCompetence{C20}{f,n}
    \item \ShowCompetence{C24}{f,h}
\end{competences}

\begin{unit}{Presentación del Curso}{}{Real}{16}{C17,C20}
  \begin{topics}
      \item Comunicación Oral y Escrita I
  \end{topics}

  \begin{learningoutcomes}
   \item Que el alummno tenga una Aproximación a algunas características de la escritura formal.
  \end{learningoutcomes}
\end{unit}

\begin{unit}{Características de la Escritura Académica}{}{Real}{16}{C17,C20}
  \begin{topics}
      \item Características de la Escritura Académica
  \end{topics}

  \begin{learningoutcomes}
   \item El estudiante podra diferenciar entre un texto Formal o Informal.
  \end{learningoutcomes}
\end{unit}

\begin{unit}{Estrategias de Lectura}{}{Real}{16}{C17,C20}
  \begin{topics}
      \item Características de las estrategias de lectura.
  \end{topics}

  \begin{learningoutcomes}
   \item El estudiante podra desarrollar  un esquema de lectura y resumen.
  \end{learningoutcomes}
\end{unit}

\begin{unit}{Estructura del Texto}{}{Real}{16}{C17,C20}
  \begin{topics}
      \item Partes del texto.
      \item Identificación de estructura en textos.
  \end{topics}

  \begin{learningoutcomes}
   \item .%ToDo
  \end{learningoutcomes}
\end{unit}

\begin{unit}{Estructura de Párrafos}{}{Real}{16}{C17,C20}
  \begin{topics}
      \item Partes del párrafo.
      \item Esquema de párrafos.
  \end{topics}

  \begin{learningoutcomes}
   \item .%ToDo
  \end{learningoutcomes}
\end{unit}

\begin{unit}{Características del párrafo}{}{Real}{16}{C17,C20}
  \begin{topics}
      \item Ejercicios de redacción de párrafos.
      \item Indicaciones generales sobre Trabajo Final.
      \item Co-evaluación sobre párrafo en FORO .
  \end{topics}

  \begin{learningoutcomes}
   \item .%ToDo
  \end{learningoutcomes}
\end{unit}

\begin{unit}{Esquema de síntesis}{}{Real}{16}{C17,C20}
  \begin{topics}
      \item Esquema y Resumen.
  \end{topics}

  \begin{learningoutcomes}
   \item .%ToDo
  \end{learningoutcomes}
\end{unit}

\begin{unit}{Texto argumentativo vs. expositivo}{}{Real}{16}{C17,C20}
  \begin{topics}
      \item Características y partes del texto expositivo.
      \item Proceso de redacción.
  \end{topics}

  \begin{learningoutcomes}
   \item .%ToDo
  \end{learningoutcomes}
\end{unit}

\begin{unit}{Proceso de Redacción}{}{Real}{16}{C17,C20}
  \begin{topics}
      \item Delimitación de tema y esquema de producción.
      \item Asesoría de preparación
      \item Tema, esquema ,producción (partes y subpartes) .
  \end{topics}

  \begin{learningoutcomes}
   \item .%ToDo
  \end{learningoutcomes}
\end{unit}

\begin{unit}{Entrega Virtual}{}{Real}{16}{C17,C20}
  \begin{topics}
      \item Tema delimitado.
      \item Justificación.
      \item Esquema.
      \item Reporte de Fuentes.
  \end{topics}

  \begin{learningoutcomes}
   \item .%ToDo
  \end{learningoutcomes}
\end{unit}

\begin{unit}{Proceso de Redacción}{}{Real}{16}{C17,C20}
  \begin{topics}
      \item Función y tipos (APA 6ta edición).
  \end{topics}

  \begin{learningoutcomes}
   \item .%ToDo
  \end{learningoutcomes}
\end{unit}

\begin{unit}{Citas}{}{Real}{16}{C17,C20}
  \begin{topics}
      \item Función y tipos
      \item Bibliografía
  \end{topics}

  \begin{learningoutcomes}
   \item .%ToDo
  \end{learningoutcomes}
\end{unit}

\begin{unit}{Tipos de párrafos}{}{Real}{16}{C17,C20}
  \begin{topics}
      \item Tipos de párrafos.
      \item Trabajo grupal en clase.
  \end{topics}

  \begin{learningoutcomes}
   \item .%ToDo
  \end{learningoutcomes}
\end{unit}

\begin{unit}{Aproximación a características de la exposición oral}{}{Real}{16}{C17,C20}
  \begin{topics}
      \item Aproximación a características de la exposición oral.
      \item Ejercicios de escritura
  \end{topics}

  \begin{learningoutcomes}
   \item .%ToDo
  \end{learningoutcomes}
\end{unit}

\begin{unit}{Párrafo Comparativo}{}{Real}{16}{C17,C20}
  \begin{topics}
      \item Establecimiento de Criterios.
      \item Asesoría Avance 2 
  \end{topics}

  \begin{learningoutcomes}
   \item .%ToDo
  \end{learningoutcomes}
\end{unit}

\begin{unit}{Características de la exposición oral y tipos de párrafo}{}{Real}{16}{C17,C20}
  \begin{topics}
      \item Coevaluaciones: párrafos enumerativo y comparativo.
      \item FORO: características de oralidad en un contexto académico.
  \end{topics}

  \begin{learningoutcomes}
   \item .%ToDo
  \end{learningoutcomes}
\end{unit}

\begin{unit}{Redacción de texto completo}{}{Real}{16}{C17,C20}
  \begin{topics}
      \item Redacción de texto completo con citas.
  \end{topics}

  \begin{learningoutcomes}
   \item .%ToDo
  \end{learningoutcomes}
\end{unit}

\begin{unit}{Teoría: redacción de textos completos}{}{Real}{16}{C17,C20}
  \begin{topics}
      \item Asesoría
      \item Indicaciones para el tercer avance.
  \end{topics}

  \begin{learningoutcomes}
   \item .%ToDo
  \end{learningoutcomes}
\end{unit}

\begin{unit}{Exposiciones}{}{Real}{16}{C17,C20}
  \begin{topics}
      \item Retroalimentación de Exposiciones.
      \item Coevaluaciones.
  \end{topics}

  \begin{learningoutcomes}
   \item .%ToDo
  \end{learningoutcomes}
\end{unit}



\begin{coursebibliography}
\bibfile{GeneralEducation/FG250}
\end{coursebibliography}

\end{syllabus}
