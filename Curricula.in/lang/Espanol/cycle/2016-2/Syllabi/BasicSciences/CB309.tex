\begin{syllabus}

\course{CB309. Computación Molecular Biológica}{Electivos}{CB309}

\begin{justification}
El uso de métodos computacionales en las ciencias biológicas se ha convertido en una de las herramientas claves para el campo de la biología molecular, siendo parte fundamental en las investigaciones de esta área. 
\\
En Biología Molecular, existen diversas aplicaciones que involucran tanto al ADN, al análisis de proteínas o al secuenciamiento del genoma humano, que dependen de métodos computacionales. Muchos de estos problemas son realmente complejos y tratan con grandes conjuntos de datos. 
\\
Este curso puede ser aprovechado para ver casos de uso concretos de varias áreas de conocimiento de Ciencia de la Computacion como: Lenguajes de Programación (PL), Algoritmos y Complejidad (AL), Probabilidades y Estadística, Manejo de Información (IM), Sistemas Inteligentes (IS).
\end{justification}

\begin{goals}
\item Que el alumno tenga un conocimiento sólido de los problemas biológicos moleculares que desafían a la computación.
\item Que el alumno sea capaz de abstraer la esencia de los diversos problemas biológicos para plantear soluciones usando sus conocimientos de Ciencia de la Computación
\end{goals}

\begin{outcomes}
    \item \ShowOutcome{a}{2}
    \item \ShowOutcome{b}{3}
    \item \ShowOutcome{l}{1}
\end{outcomes}

\begin{competences}
    \item \ShowCompetence{C1}{a,b} 
    \item \ShowCompetence{C3}{b,l}
    \item \ShowCompetence{C5}{a,b}
\end{competences}

\begin{unit}{Introducción a la Biología Molecular}{}{Clote2000,Setubal1997}{4}{CS1}
\begin{topics}
\item Revisión de la química orgánica: moléculas y macromoléculas, azúcares, acidos nucleicos, nuclótidos, ARN, ADN, proteínas, aminoácidos y nivels de estructura en las proteinas. 
\item El dogma de la vida: del ADN a las proteinas, transcripción, traducción, síntesis de proteinas
\item Estudio del genoma: Mapas y secuencias, técnicas específicas
\end{topics}
   \begin{learningoutcomes}
      \item  Lograr un conocimiento general de los tópicos más importantes en Biología Molecular. [\Familiarity]
	  \item Entender que los problemas biológicos son un desafío al mundo computacional. [\Assessment]
   \end{learningoutcomes}
\end{unit}

\begin{unit}{Comparación de Secuencias}{}{Clote2000,Setubal1997,Pevzner2000}{4}{CS2}
\begin{topics}
\item Secuencias de nucléotidos y secuencias de aminoácidos.
\item Alineamiento de secuencias, el problema de alineamiento por pares, búsqueda exhaustiva, Programación dinámica, alineamiento global, alineamiento local, penalización por gaps
\item Comparación de múltiples secuencias: suma de pares, análisis de complejidad por programación dinámica, heurísticas de alineamiento, algoritmo estrella, algoritmos de alineamiento progresivo.
\end{topics}
\begin{learningoutcomes}
\item  Entender y solucionar el problema de alineamiento de un par de secuencias. [\Usage]
\item  Comprender y solucionar el problema de alineamiento de múltiples secuencias. [\Usage]
\item Conocer los diversos algoritmos de alineamiento de secuencias existentes en la literatura. [\Familiarity]
\end{learningoutcomes}
\end{unit}

\begin{unit}{Árboles Filogenéticos}{}{Clote2000,Setubal1997,Pevzner2000}{4}{CS2}
\begin{topics}
\item Filogenia: Introducción y relaciones filogenéticas.
\item Arboles Filogenéticos: definición, tipo de árboles, problema de búsqueda y reconstrucción de árboles
\item Métodos de Reconstrucción: métodos por parsimonia, métodos por distancia, métodos por máxima verosimilitud, confianza de los árboles reconstruidos
\end{topics}

\begin{learningoutcomes}
\item  Comprender el concepto de filogenia, árboles filogenéticos y la diferencia metodológica entre biología y biología molecular. [\Familiarity]
\item Comprender el problema de reconstrucción de árboles filogenéticos, conocer y aplicar los principales algoritmos para reconstrucción de árboles filogenéticos. [\Assessment]
\end{learningoutcomes}
\end{unit}

\begin{unit}{Ensamblaje de Secuencias de ADN}{}{Setubal1997,Aluru2006}{4}{CS2}
\begin{topics}
\item Fundamento biológico: caso ideal, dificultades, métodos alternativos para secuenciamiento de ADN
\item Modelos formales de ensamblaje: \textit{Shortest Common Superstring}, \textit{Reconstruction}, \textit{Multicontig}
\item Algoritmos para ensamblaje de secuencias: representación de overlaps, caminos para crear \textit{superstrings}, algoritmo voraz, grafos acíclicos.
\item Heurísticas para ensamblaje: búsqueda de sobreposiciones, ordenación de fragmentos, alineamientos y consenso.
\end{topics}

\begin{learningoutcomes}
\item Comprender el desafío computacional que ofrece el problema de Ensamblaje de Secuencias. [\Familiarity]
\item Entender el principio de modelo formal para ensamblaje. [\Assessment]
\item Conocer las principales heurísticas para el problema de ensambjale de secuencias ADN [\Usage]
\end{learningoutcomes}
\end{unit}

\begin{unit}{Estructuras secundarias y terciarias}{}{Setubal1997,Clote2000,Aluru2006}{4}{CS2}
   \begin{topics}
    \item Estructuras moleculares: primaria, secundaria, terciaria, cuaternaria.
    \item Predicción de estructuras secundarias de ARN: modelo formal, energia de pares, estructuras con bases independientes, solución con Programación Dinámica, estructuras con bucles.
    \item {\it Protein folding}: Estructuras en proteinas, problema de \textit{protein folding}.
    \item {\it Protein Threading}: Definiciones, Algoritmo \textit{Branch \& Bound}, \textit{Branch \& Bound} para \textit{protein threading}.
    \item {\it Structural Alignment}: definiciones, algoritmo DALI
   \end{topics}
   \begin{learningoutcomes}
     \item Conocer las estructuras protéicas y la necesidad de métodos computacionales para la predicción de la geometría. [\Familiarity]
	   \item Cnocer ls algoritmos de solución de problemas de predicción de estructuras secundarias ARN, y de estructuras en proteínas. [\Assessment]
   \end{learningoutcomes}
\end{unit}

\begin{unit}{Modelos Probabilísticos en Biología Molecular}{}{Durbin1998,Clote2000,Aluru2006,Krogh1994}{4}{CS2}
   \begin{topics}
    \item Probabilidad: Variables aleatorias, Cadenas de Markov, Algoritmo de Metropoli-Hasting, Campos Aleatorios de Markov y Muestreador de Gibbs, Máxima Verosimilitud.
    \item Modelos Ocultos de Markov (HMM), estimación de parámetros, algoritmo de Viterbi y método Baul-Welch, Aplicación en alineamientos de pares y múltiples, en detección de Motifs en proteínas, en ADN eucariótico, en familias de secuencias.
		\item Filogenia Probabilística: Modelos probabilísticos de evolución, verosimilitud de alineamientos, verosimilitud para inferencia, comparación de métodos probailísticos y no probabilísticos
   \end{topics}
   \begin{learningoutcomes}
      \item  Revisar conceptos de Modelos Probabilísticos y comprender su importancia en Biología Molecular Computacional. [\Assessment]
	  \item Conocer y aplicar Modelos Ocultos de Markov para varios análisis en Biología Molecular. [\Usage]
		\item Conocer la aplicación de modelos probabilísticos en Filogenia y comparalos con modelos no probabilísticos [\Assessment]
   \end{learningoutcomes}
\end{unit}



\begin{coursebibliography}
\bibfile{BasicSciences/CB309}
\end{coursebibliography}

\end{syllabus}
