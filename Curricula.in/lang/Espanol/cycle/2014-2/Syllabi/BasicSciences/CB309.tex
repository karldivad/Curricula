\begin{syllabus}

\course{CB309. Computación Molecular Biológica}{Electivo}{CB309}

\begin{justification}
El uso de métodos computacionales en las ciencias biológicas
se ha convertido en una de las herramientas claves para el
campo de la biología molecular, y éstas actualmente son usadas
como parte crítica en sus investigaciones. Existen diversas
aplicaciones en biología molecular relativas tanto al ADN como
al análisis de proteínas. La construcción del genoma humano,
por ejemplo, depende fundamentalmente de la biología molecular
computacional. Muchos de los problemas de ésta área son realmente
complejos y con conjuntos enormes de datos. Este curso además
puede servir para ejemplificar algunos tópicos de Fundamentos
de Programación (PF) y Algoritmos y Complejidad (AL) de
acuerdo al Computing Curricula 2001.
\end{justification}

\begin{goals}
\item Interpretar problemas biológicos haciendo uso de técnicas computacionales.
\item Analizar e implementar algorítmos y estructuras aplicables al campo de la biología.
\end{goals}

\begin{outcomes}
\ExpandOutcome{a}{3}
\ExpandOutcome{g}{3}
\ExpandOutcome{h}{3}
\ExpandOutcome{i}{3}
\ExpandOutcome{j}{4}
\end{outcomes}

\begin{unit}{Conceptos Introductorios}{dav01,pav04,pete00}{0}{3}
\begin{topics}
        \item Introdución a la Historia de la Genética
        \item Conceptos Básicos de Biología Molecular
        \item Problemas clásicos en Bioinformática
        \item Herramientas de recolección y almacenamiento de secuencias en laboratorio
        \item Recursos de Software, introducción a BLAST, CLUSTALW
        \item Cadenas, Grafos y Algoritmos
    \end{topics}
    \begin{unitgoals}
        \item Identificación de los conceptos básicos en Biología Molecular
        \item Reconocimiento de problemas clásicos en Biología Molecular y su representación en el campo computacional
        \item Aprendizaje de las herramientas de software e Internet clásicas para el campo de Bioinformática
        \item Introducción a los conceptos necesarios en manejo de Cadenas, Grafos y su representación algorítmica a fin de transformar problemas biológicos al tipo computacional
    \end{unitgoals}
\end{unit}

\begin{unit}{Alineamiento de Secuencias}{dav01,pav04,pete00}{0}{4}
\begin{topics}
        \item Introducción al alineamiento de secuencias
        \item Comparación de pares de secuencias
        \item Alineamiento de Secuencias Global
        \item Alineamiento de Secuencias Múltiples
        \item Cadenas ocultas de Markov
        \item Métodos exactos, aproximados y heurísticos del alineamiento de secuencias
        \item Problemas derivados del alineamiento de secuencias
    \end{topics}
    \begin{unitgoals}
        \item Reconocimiento de las técnicas básicas usadas en el alineamiento de secuencias
        \item Implementación de los diversos algoritmos de comparación de secuencias
        \item Introducción a la programación dinámica
        \item Introducción y comparativa entre métodos heurísticos y exactos
        \item Métodos probabilísticos: PAM
    \end{unitgoals}
\end{unit}

\begin{unit}{Clustering}{dav01,pav04,pete00}{0}{4}
\begin{topics}
        \item El problema del Clustering
        \item Clustering Jerárquico
        \item Algoritmo de Neighbour Joining
        \item Algoritmo del Average linkage
        \item Clustering no jerárquico o K-means
        \item EST clustering
    \end{topics}
    \begin{unitgoals}
        \item Identificar métodos de distancia aplicables a grafos del tipo árboles
        \item Conocer la transformación de Matrices en estructuras de grafos
        \item Reconocer a los métodos de Clustering como útiles para la identificación de funciones en genes no conocidos a partir de genes similares
        \item Identificar la importancia del Clustering en el reconocimiento de patrones de enfermedades
    \end{unitgoals}
\end{unit}

\begin{unit}{Árboles Filogenéticos}{dav01,pav04,pete00}{0}{3}
\begin{topics}
        \item Introducción a la Filogenia
        \item Algoritmos comunes
        \item Aplicaciones biológicas
        \item Algoritmos Exactos
        \item Algoritmos Probabilísticos
    \end{topics}
    \begin{unitgoals}
        \item Reconocer algoritmos de mediciones de distancias
        \item Analizar la complejidad computacional de cada uno de los algoritmos estudiados
        \item Reconocer la importancia de la filogenía en casos de evolución de epidemias como el HIV
        \item Utilización de herramientas de software de libre uso
        \item Implementación de los algoritmos estudiados
    \end{unitgoals}
\end{unit}

\begin{unit}{Mapeo de Secuencias}{dav01,pav04,pete00}{0}{3}
\begin{topics}
        \item Problema del \textit{Double Digest} y \textit{Partial Digest}
        \item Técnicas utilizadas en el mapeo de secuencias
        \item Mapeo con \textit{Non-Unique Probes}
        \item Mapeo con \textit{Unique Probes}
        \item Grafos de Intervalos
        \item Mapeo con Señales de Frecuencias de Restricción
    \end{topics}
    \begin{unitgoals}
        \item Identificación de problemas NP-Complejos
        \item Aplicación e implementación de técnicas diversas a fín de dar solución a éstos problemas biológicos
        \item Introducción a los métodos de tipo goloso
        \item Reconocimiento de tópicos avanzados en teoría de grafos
    \end{unitgoals}
\end{unit}

\begin{unit}{Introducción a la Estructura de las Proteínas}{dav01,pav04,pete00}{0}{2}
\begin{topics}
        \item Fundamentos biológicos de las proteínas
        \item Motivación para la predicción de las estructuras de las proteínas
        \item Alineamiento rígido de Proteínas
        \item Técnica del alineamiento por Hashing Geométrico
        \item Predicción de la estructuras de las proteínas
    \end{topics}

    \begin{unitgoals}
        \item Examina algunos tópicos de reconocimiento visual en Computación Gráfica
        \item Implementación de algunos estructuras simples como el folding 2D
    \end{unitgoals}
\end{unit}



\begin{coursebibliography}
\bibfile{BasicSciences/CB309}
\end{coursebibliography}

\end{syllabus}
