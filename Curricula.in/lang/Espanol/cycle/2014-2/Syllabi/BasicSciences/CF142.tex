\begin{syllabus}

\curso{CF142. Física General II}{Obligatorio}{CF142}

\begin{justification}
Mostrar un alto grado de dominio de las leyes del movimiento ondulatorio, la naturaleza de los fluidos y la termodinámica. Utilizando adecuadamente los conceptos de movimiento ondulatorio, de fluidos y de termodinámica en la resolución de problemas de la vida cotidiana. Poseer capacidad y habilidad en la interpretación de los fenómenos ondulatorios, de fluidos y termodinámicos, que contribuyan en la elaboración de soluciones eficientes y útiles en diferentes áreas de la ciencia de la computación.
\end{justification}

\begin{goals}
\item  Que el alumno aprenda y domine los principios fluídos estáticos y en movimiento.
\item  Que el alumno aprenda y domine los principios del MAS, particularmente del movimiento ondulatorio.
\item  Que el alumno aprenda y domine los principios de Termodinámica.
\item  Que el alumno aprenda a aplicar principios de la Física de fluídos, ondas y termodinámica para desarrollar modelos computacionales.
\end{goals}

\begin{outcomes}
\ExpandOutcome{a}
\ExpandOutcome{i}
\ExpandOutcome{j}
\end{outcomes}

\begin{unit}{FI1. Elasticidad}{Sears98,Eisberg98}{4}
\begin{topics}
         \item  Esfuerzo y deformación unitaria
	 \item  Módulo de Young
         \item  Módulo y Coeficiente de Poisson
	 \item  Módulo de Rigidez
         \item  Módulo y Coeficiente de comprensibilidad
   \end{topics}

   \begin{unitgoals}
         \item  Entender y caracterizar los procesos de elasticidad
         \item  Resolver problemas
   \end{unitgoals}
\end{unit}

\begin{unit}{FI2. Fluidos}{Serway98, Tipler98}{8}
\begin{topics}
         \item  Densidad y peso específico
	 \item  Presión en los fluidos. Presión atmosférica y presión manométrica
         \item  Principio de Pascal. Medición de la presión: manómetro y barómetro
	 \item  Flotabilidad y Principio de Arquímedes
         \item  Fluidos en movimiento: flujo y ecuación de continuidad
	 \item  Ecuación de Bernoulli. Aplicaciones del principio de Bernoulli: teorema de Torricelli, el tubo ventura
         \item  Tensión superficial y capilaridad
   \end{topics}

   \begin{unitgoals}
         \item  Explicar, analizar y caracterizar la presión de fluidos
         \item  Entender, caracterizar y aplicar el principio de Arquímedes
         \item  Entender, caracterizar y aplicar el principio de Bernoulli
         \item  Explicar, analizar y caracterizar la tensión superficial y capilaridad
   \end{unitgoals}
\end{unit}

\begin{unit}{FI3. Movimiento Periódico}{Sears98, Serway98}{8}
\begin{topics}
         \item  Introducción. Constante elástica de un resorte
	 \item  Movimiento armónico simple. Energía en el oscilador armónico simple
         \item  Círculo de referencia: el período y la naturaleza senoidal del movimiento armónico simple
	 \item  Péndulo simple.
         \item  Movimiento armónico amortiguado.
         \item  Oscilaciones forzadas: resonancia.
   \end{topics}

   \begin{unitgoals}
         \item  Explicar, analizar y caracterizar el movimiento oscilatorio a partir del MAS.
         \item  Resolver problemas.
   \end{unitgoals}
\end{unit}

\begin{unit}{FI4. Ondas}{Eisberg98,Resnick98,Douglas84}{8}
\begin{topics}
         \item  Movimiento ondulatorio. Tipos de ondas. Ondas viajeras unidimensionales
	 \item  Superposición e interferencia de ondas
         \item  Velocidad de las ondas en una cuerda tensa. Reflexión y transmisión de ondas
	 \item  Ondas senoidales. Energía transmitida por ondas senoidales en cuerdas
         \item  Ondas estacionarias en una cuerda. Ondas sonoras. Velocidad de las ondas sonoras
	 \item  Ondas sonoras periódicas. Intensidad de ondas sonoras periódicas
	 \item  Fuentes de sonido: cuerdas vibratorias y columnas vibratorias de aire
	 \item  Efecto Doppler
   \end{topics}

   \begin{unitgoals}
         \item  Explicar, encontrar y caracterizar mediante problemas de la vida cotidiana el movimiento ondulatorio, así como, la reflexión y transmisión de ondas en el espacio
         \item  Resolver problemas
   \end{unitgoals}
\end{unit}

\begin{unit}{FI5. Temperatura y Teoría Cinética}{Eisberg98,Resnick98}{12}
\begin{topics}
         \item  Átomos. Temperatura. Termómetros y escalas de temperatura
	 \item  Dilatación térmica de sólidos y líquidos. Coeficientes de dilatación lineal, superficial y cúbico
         \item  Leyes de los gases y la temperatura absoluta. La ley del gas ideal en términos moleculares: número de Avogadro
	 \item  Teoría cinética e interpretación molecular de la temperatura. Distribución de velocidades moleculares
         \item  Procesos isotérmicos y adiabáticos para un gas ideal. La equipartición de la energía
	 \item  Termodinámica. Tipos de sistemas que estudia la Termodinámica
         \item  Ley cero de la Termodinámica
	 \item  El termómetro de gas a volumen constante y la escala Kelvin
         \item  Punto triple del agua
   \end{topics}

   \begin{unitgoals}
         \item  Explicar, analizar y caracterizar el concepto de Temperatura y la dilatación térmica de sólidos y líquidos
         \item  Entender la ley del gas ideal y los procesos isotérmicos y adiabáticos para un gas ideal
         \item  Entender la ley cero de la Termodinámica
         \item  Resolver problemas
   \end{unitgoals}
\end{unit}

\begin{unit}{FI6. Calor y primera Ley de la Termodinámica}{Eisberg98,Resnick98}{8}
\begin{topics}
         \item  Calor como transferencia de energía
	 \item  Capacidad calorífica y calor específico
         \item  Energía interna de un gas ideal
	 \item  Calor específico de un gas ideal
         \item  Cambios de fase. Calor latente de fusión y de vaporización
	 \item  Calorimetría. Trabajo y calor en procesos termodinámicos
         \item  La primera ley de la Termodinámica
	 \item  Algunas aplicaciones de la primera ley de la Termodinámica
         \item  Transmisión del calor por conducción, convección y radiación
   \end{topics}

   \begin{unitgoals}
         \item  Entender el concepto de calor y de energía interna de un gas ideal
         \item  Explicar, analizar y caracterizar la primera ley de la Termodinámica
         \item  Resolver problemas
   \end{unitgoals}
\end{unit}

\begin{unit}{FI7. Máquinas térmicas, entropía y la segunda ley de la Termodinámica}{Eisberg98,Resnick98}{8}
\begin{topics}
         \item  Máquinas térmicas y la segunda ley de la Termodinámica
	 \item  Procesos reversibles e irreversibles. La máquina de Carnot
         \item  Escala de temperatura absoluta. Refrigeradores
	 \item  Entropía. Cambios de entropía en procesos irreversibles
   \end{topics}

   \begin{unitgoals}
         \item  Explicar, analizar y caracterizar la primera ley de la Termodinámica
         \item  Explicar, analizar y caracterizar la máquina de Carnot
         \item  Resolver problemas
   \end{unitgoals}
\end{unit}

\begin{coursebibliography}
\bibfile{BasicSciences/CF141}
\end{coursebibliography}

\end{syllabus}

%\end{document}
