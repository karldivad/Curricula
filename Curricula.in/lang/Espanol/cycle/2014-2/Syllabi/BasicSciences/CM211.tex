\begin{syllabus}

\curso{CM211. Cálculo Diferencial e Integral Avanzado}{Obligatorio}{CM211}

\begin{justification}
Se trata funciones vectoriales de una variable real, funciones reales de varias variables incidiéndose en funciones diferenciables y sus aplicaciones. Se estudia las integrales múltiples con aplicación a obtener áreas y volúmenes.
Asimismo, se trata integrales de línea y superficie analizando los teoremas de Stokes y Gauss con aplicaciones a flujo.
\end{justification}

\begin{goals}
\item  Presentar el Cálculo Diferencial e integral para funciones de varias variables.
\item  Desarrollar técnicas numéricas y analíticas, para abordar algunos problemas que surgen en las aplicaciones. de las matemáticas.
\end{goals}

\begin{outcomes}
\ExpandOutcome{a}
\ExpandOutcome{i}
\ExpandOutcome{j}
\end{outcomes}

\begin{unit}{Funciones Vectoriales de una Variable Real}{Apostol70,Hasser97}{14}
   \begin{topics}
	\item  Definición de una función vectorial de variable real f: $A \rightarrow R^n$, $A \subset R$, gráfica del rango de f y gráfica de f. Operaciones con funciones.
	\item Límites. Definición. Propiedades. Continuidad  Propiedades Definición de curva C : $r : I \rightarrow R^n$
	\item  Derivada. Vector tangente. Teoremas sobre la derivada. La diferenciabilidad
	\item  Integración. Propiedades. Longitud de arco. Curva Rectificable. Fórmula integral de la longitud de arco
	\item  Curvas parametrizadas. Curvas regulares. Reparametrización. La longitud de arco como parametrización. Cambio admisible de parámetro.
	\item  Vector tangente, normal, binormal (y sus unitarios). Plano osculador. Plano normal. Triedo Móvil
	\item Curvatura. Radio de curvatura.  Centro de curvatura. Torsión. Fórmulas de Frenet
   \end{topics}

   \begin{unitgoals}
         \item  Entender los conceptos y características de las funciones vectoriales de una variable real
         \item  Resolver problemas
   \end{unitgoals}
\end{unit}

\begin{unit}{Funciones Reales de varias variables}{Apostol70,Hasser97}{12}
   \begin{topics}
	\item  Definición de función real de varias variables f: $A \rightarrow R$, $A \subset R^n$. Gráficas y rangos. Curvas de nivel, superficies de nivel. Operaciones sobre funciones.
	\item  Superficies cuadráticas, superficies cilíndricas y superficies regladas.
	\item  Limites: Propiedades. Continuidad: Propiedades. Teorema del valor intermedio. Teorema de Weierstrass f: $K \rightarrow R$, $K$ compacto. Máximo y mínimo.
	\item  Derivadas direccionales: Significado geométrico. Teorema de valor medio. Derivada parcial. Propiedades.
	\item  Diferenciabilidad. Propiedades. Teorema de valor medio. Vector gradiente. Condición suficiente de diferenciabilidad.
	\item  Regla de la cadena. Plano tangente y recta normal. Razón de cambio máximo.
	\item Funciones implícitas. Teorema de la función implícita
	\item Derivadas parciales de orden superior. Teorema de Taylor
	\item Máximos y mínimos relativos y absolutos. Criterio de las derivadas parciales (primera derivadas y segunda derivadas.)
   \end{topics}

   \begin{unitgoals}
         \item  Entender los conceptos y características de las funciones reales de varias variables
         \item  Resolver problemas
   \end{unitgoals}
\end{unit}

\begin{unit}{Funciones Vectoriales de Variables Vectoriales}{Apostol70, Hasser97}{8}
   \begin{topics}
         \item  Transformaciones de $R^n$ a $R^m$. Transformaciones afines de $R^n$.
	 \item  Limite. Definición: Propiedades. Continuidad: Propiedades.
         \item  La derivada y la diferencial. Propiedades. Funciones de clase $C^k$. Regla de la cadena
	 \item  Transformaciones en coordenadas polares $(r,\theta)$. Transformaciones cilíndricas  $(r,\theta,z)$ y transformaciones esféricas $(\rho,/theta,\varphi)$
         \item  Matriz Jacobiana. Transformaciones con Jacobianos no nulos. Teorema de la función inversa. Interpretación geométrica. Teorema de la función implícita
   \end{topics}

   \begin{unitgoals}
         \item  Entender los conceptos y características de las funciones vectoriales de variables variables
         \item  Resolver problemas
   \end{unitgoals}
\end{unit}

\begin{unit}{Integrales Múltiples}{Protter69,Venero94}{12}
   \begin{topics}
         \item  Integrales dobles de una función acotada sobre un rectángulo en $R^2$. Funciones integrales.
	 \item  Propiedades básicas de $\iint_{a}^{b} f$
         \item  Integral doble de una función acotada sobre un rectángulo en $R^2$. Propiedades
	 \item  Evaluación de una integral doble por integrales iteradas:
	\begin{subtopicos}
		\item sobre un rectángulo y
		\item sobre una región acotada de $R^2$
	\end{subtopicos}
         \item  Cambio de variables para integrales dobles. Integrales dobles en coordenadas polares.
	 \item  Cálculo de áreas y volúmenes bajo una superficies y volúmenes de revolución
         \item  Integral triple. Integral triple sobre un rectángulo y sobre un conjunto acotado en R3. Propiedades
	\item Evaluación de una integral por integrales iteradas. Cambio de variables para integrales triples. Volumen.
	\item Integrales triples en coordenadas cilíndricas. Integrales triples en coordenadas esféricas.
   \end{topics}

   \begin{unitgoals}
         \item  Entender y aplicar los conceptos de Integrales múltiples
         \item  Resolver problemas
   \end{unitgoals}
\end{unit}

\begin{unit}{Integrales de Líneas}{Protter69,Venero94}{8}
   \begin{topics}
	\item  Integral de línea primer tipo (Integral de línea con respecto a la longitud de arco). Aplicaciones. Centro de gravedad
	\item  Integral de línea de segundo tipo. Propiedades. Comportamiento de una integral de línea frente a un cambio de parámetro. El trabajo como integral de línea
	\item  Región simplemente conexo. Región múltiplemente conexo. Teorema de Green. Aplicaciones.
	\item  Independencia del camino. Teoremas fundamentales del cálculo para integrales de línea.
	\item Condiciones necesarias y suficientes para que un campo vectorial sea un campo gradiente. Construcción de funciones potenciales.
   \end{topics}

   \begin{unitgoals}
         \item  Entender y aplicar los conceptos de Integrales de líneas
         \item  Resolver problemas
   \end{unitgoals}
\end{unit}

\begin{unit}{Integrales de Superficies}{Protter69,Venero94}{8}
   \begin{topics}
         \item  Superficies paramétricas. Plano tangente. Plano normal.
	 \item  Producto vectorial fundamental. Área de una superficie paramétrica.
         \item  Integral de superficie de primer tipo (Integral de Campos escalares sobre superficies). Significado físico.
	 \item  Superficies orientables. Integral de superficies de segundo tipo (Integrales de campos vectoriales sobre superficies).
         \item  Teorema de Stokes. El rotacional y la divergencia de un campo vectorial. Propiedades.
         \item  Teorema de Gauss (Teorema de la Divergencia). Aplicaciones.
   \end{topics}

   \begin{unitgoals}
         \item  Entender y aplicar los conceptos de Integrales de Superficie
         \item  Resolver problemas
   \end{unitgoals}
\end{unit}

\begin{coursebibliography}
\bibfile{BasicSciences/CM211}
\end{coursebibliography}

\end{syllabus}
