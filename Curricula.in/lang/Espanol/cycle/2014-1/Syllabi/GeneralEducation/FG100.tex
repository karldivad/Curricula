\begin{syllabus}

\course{FG100. Informática I}{Obligatorio}{FG101}

\begin{justification}
Esta materia de carácter práctica, permitirá a los alumnos dominar los principales programas de uso informático, reconociendo que el uso de las computadoras en los tiempos modernos es una herramienta de suma importancia, no sólo para el desarrollo de nuestros pueblos, sino también, para el desarrollo de la ciencia y de nuevas tecnologías, debido a los crecientes avances. La materia permite visualizar la revolución que ha dado lugar a la sociedad de la información, también denominada sociedad digital, en la cual el insumo de mayor valor económico es la información. En el tratamiento de esta información convergen múltiples tecnologías, como son la informática y las telecomunicaciones. 
\end{justification}

\begin{goals}
Al terminar la asignatura el estudiante será capaz de:

\item Identificar los diferentes recursos informáticos y tecnológicos existentes en el mercado que coadyuven en el desarrollo de la carrera.
\item Desarrollar destrezas utilizando los diferentes paquetes informáticos.
\end{goals}

\begin{outcomes}
\ExpandOutcome{a}{3}
\ExpandOutcome{d}{3}
\ExpandOutcome{i}{3}
\ExpandOutcome{o}{3}
\end{outcomes}

\begin{unit}{Sistemas Operativos}{USGP}{9}{3}
\begin{topics}
      \item Definiciones Básicas
      \item Escritorio de Windows
      \item Administrador de tareas
      \item Elementos de una ventana
      \item Organizar iconos
      \item Manipulación de carpetas
      \item Gadgets de Windows 
      \item Métodos de Selección
      \item Configuración del Sistema
      \item Buscar archivos o carpetas
      \item Ejecutar archivos o programas
      \item Utilización de medios de almacenamiento extraíbles
      \item Introducción al software libre
      \item El perfil de usuario y entorno gráfico de Linux
      \item Conveniencia de usar Linux sobre Windows
\end{topics}
\begin{unitgoals}
   \item dentificar los diferentes sistemas operativos o plataformas de mayor demanda en el mundo, tales como Windows y Linux.
\end{unitgoals}
\end{unit}

\begin{unit}{Navegadores de Internet}{USGP}{9}{3}
\begin{topics}
      \item Generalidades
      \item Manejo y uso del navegador
      \item Web
      \item Búsqueda
      \item Correo electrónico
      \item Chat
      \item Descargas
      \item Seguridad de la información
\end{topics}
\begin{unitgoals}
   \item Manipular el Internet a través de navegadores tales Mozilla, Opera e Internet Explorer para buscar, intercambiar, navegar, chatear, transferir y descargar información en todo el mundo a través de un computador o red de computadores.
\end{unitgoals}
\end{unit}

\begin{unit}{Uso del procesador de palabras}{USGP}{9}{3}
\begin{topics}
      \item Acceso al programa
      \item Elementos de la ventana principal
      \item Pasos para digitar un documento
      \item Métodos de selección
      \item Copiar, cortar o borrar información
      \item Barra de herramienta de acceso rápido
      \item Cinta de Opciones Inicio
      \item Cinta de Opciones Insertar
      \item Cinta de Opciones Diseño de página
      \item Cinta de opciones Revisar
      \item Cinta de opciones Vista
      \item Cinta de opciones Correspondencia
\end{topics}
\begin{unitgoals}
   \item Desarrollar y escribir documentos utilizando el procesador de palabras.
\end{unitgoals}
\end{unit}

\begin{unit}{Uso de la hoja de cálculo}{USGP}{9}{3}
\begin{topics}
      \item Acceso al programa
      \item Ventana principal
      \item Ingreso de datos 
      \item Operaciones básicas 
      \item Métodos de selección 
      \item Encabezado y pie de página 
      \item Configuración de página 
      \item Opciones de formato 
      \item Guía de repetición
      \item Fórmulas y funciones 
      \item Gráficos estadísticos
\end{topics}
\begin{unitgoals}
   \item Realizar operaciones y cálculos matemáticos, cuadros y gráficos estadísticos, utilizando fórmulas y funciones de la hoja de cálculo.
\end{unitgoals}
\end{unit}

\begin{unit}{Uso del presentador de diapositivas}{USGP}{9}{3}
\begin{topics}
      \item Acceso al programa
      \item Ventana principal
      \item Operaciones básicas
      \item Diseño y configuración de diapositivas
      \item Vistas de diapositivas
      \item Configuración e impresión de diapositivas
      \item Plantillas de diseño 
      \item Inserción de imágenes, objetos, sonidos y videos
      \item Vínculos
\end{topics}
\begin{unitgoals}
   \item Diseñar presentaciones que contengan animaciones, sonidos, estilos de fondos, entre otros.
\end{unitgoals}
\end{unit}

\begin{coursebibliography}
\bibfile{GeneralEducation/FG101}

\end{coursebibliography}

\end{syllabus}
