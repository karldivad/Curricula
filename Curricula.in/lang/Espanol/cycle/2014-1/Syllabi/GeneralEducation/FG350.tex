\begin{syllabus}

\course{FG350. Liderazgo y Desempeño}{Obligatorio}{FG350}

\begin{justification}
El mundo de hoy y las organizaciones existentes exigen de líderes que permitan orientarlas hacia la construcción de una sociedad más justa y reconciliada.  Ese desafío pasa por la necesidad de formar personas con un recto conocimiento de sí mismos, con la capacidad de juzgar objetivamente la realidad y de proponer orientaciones que busquen modificar positivamente el entorno.

El curso de Liderazgo y Desempeño pretende desarrollar los criterios, habilidades y actitudes necesarios para cumplir con éste propósito.
\end{justification}

\begin{goals}
\item \OutcomeHU
\item Mostrar la influencia del liderazgo a través de la historia.
\item Dar a conocer la imortancia de un liderazgo equilibrado en nuestra sociedad.
\item Forjar en el alumno un desempeño honesto y preciso.
\end{goals}

\begin{outcomes}
\ExpandOutcome{d}{3}
\ExpandOutcome{f}{3}
\ExpandOutcome{HU}{3}
\end{outcomes}

\begin{unit}{Aproximación al liderazgo}{NotasLiderazgo2006,TheodorHaecker,Guardini1992,Hesselbein1999}{20}{2}
\begin{topics}
	\item Introducción al liderazgo
	\item Estilos actuales de liderazgo
	\item Visiones erradas del ser humano
	\item La vocación humana
	\item Ensayando una definición de liderazgo
	\item Liderazgo en la historia
	\item Importancia de las aproximaciones históricas
	\item Elementos para analizar un liderazgo histórico
\end{topics}
\begin{unitgoals}
	\item Conocer las características del liderazgo, su importancia y trascendencia a través de la historia.
\end{unitgoals}
\end{unit}

\begin{unit}{Liderazgo personal/Maestría personal}{NotasLiderazgo2006,TheodorHaecker,Guardini1992,Hesselbein1999}{45}{3}
\begin{topics}
	\item Introducción al liderazgo personal
	\item El primer campo de liderazgo soy yo
	\item Autoridad y liderazgo
	\item Introducción al autoconocimiento y liderazgo
	\item El ruido
	\item Hacer silencio
	\item Obstáculos para el autoconocimiento
	\item Empezando a conocerme
	\item Que no es conocerme
	\item Aproximación al autoconocimiento.
	\item El hombre unidad de mente cuerpo y espíritu.
	\item El cuerpo
	\item La mente
	\item El espíritu
	\item Características de la mismidad
	\item La libertad
	\item La dimisión de lo humano
	\item La Prudencia
	\item Toma de conciencia
	\item Mi liderazgo personal
	\item Análisis FODA personal
	\item Plan de vida
	\item Manejo de horario
\end{topics}
\begin{unitgoals}
	\item Entender que el primer campo de liderazgo es la misma persona
	\item Profundizar en el descubrimiento del misterio de la persona humana
	\item Desarrollar habilidades y actitudes de líder
\end{unitgoals}
\end{unit}

\begin{unit}{Liderazgo en grupos}{NotasLiderazgo2006,TheodorHaecker,Guardini1992,Hesselbein1999}{10}{3}
\begin{topics}
	\item La relación personal con el equipo
	\item Liderazgo integral
	\item Acompañamiento y discipulado
	\item Fundamentos de unidad
\end{topics}
\begin{unitgoals}
	\item Desarrollar habilidades para el trabajo en equipo
\end{unitgoals}
\end{unit}



\begin{coursebibliography}
\bibfile{GeneralEducation/FG350}
\end{coursebibliography}

\end{syllabus}
