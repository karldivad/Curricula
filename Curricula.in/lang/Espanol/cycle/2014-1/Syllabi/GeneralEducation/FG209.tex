\begin{syllabus}

\course{FG209. Psicología}{Electivo}{FG209}

\begin{justification}
Es necesario que los alumnos puedan llegar a entender los principios fundamentales en  los que se basa la Psicología general desde un aporte científico pero también critico tomando en cuenta a la persona como unidad biopsicoespiritual.
\end{justification}

\begin{goals}
\item Identificar las tendencias clasicas y actuales en psicología.
\end{goals}

\begin{outcomes}
\ExpandOutcome{HU}{2}
\end{outcomes}

\begin{unit}{Primera Unidad}{Morris,Felman}{0}{3}
\begin{topics}
	\item Conceptos Básicos: Ciencia de la Psicología.
	\item Concepto de Campo de aplicación.
	\item Introspección, extrospección, análisis y experimentación.
\end{topics}
\begin{unitgoals}
	\item Identificación del rol de la psicología en sus diversos contextos.
	\item Precisiones sobre la Psicología y su importancia.
	\item Destacar los diferentes modelos de investigación en psicología.
\end{unitgoals}
\end{unit}

\begin{unit}{Segunda Unidad}{Morris,Felman}{0}{2}
\begin{topics}
	\item Tipologías del temperamento.
	\item Estructuras de personalidad.
	\item Configuración de los aspectos cognitivos, emocionales y volitivos.
	\item Teorías de corte psicoanalítico.
	\item Teorías cognitivo conductuales.
	\item Teorías de Aprendizaje e imitación.
	\item Teorías humanistas.
	\item Desarrollo de la inteligencia cognitiva y la inteligencia emocional.
	\item Desarrollo sensorial: percepción, atención.
	\item Desarrollo emocional: emociones, afectos, motivaciones.
\end{topics}
\begin{unitgoals}
	\item Identificar la conformación de los aspectos ligados al desarrollo del temperamento,  
	\item       carácter y personalidad.
	\item Conocer las teorías sobre fundamentos de la personalidad.
	\item Ubicar los aspectos mas relevantes d as estructuras de personalidad.
\end{unitgoals}
\end{unit}

\begin{unit}{Tercera Unidad}{Morris,Felman}{0}{2}
\begin{topics}
	\item  Etapa Prenatal: el valor de la vida humana.
	\item Infancia y Niñez: forjando las bases.
	\item Adolescencia: Etapa del despliegue y potencialidades.
	\item Adultez senectud: Familia y trabajo.
	\item El proceso de salud-enfermedad.
	\item Principales trastornos psicológicos
\end{topics}
\begin{unitgoals}
	\item Comprensión del proceso evolutivo y el desarrollo humano en sus diversos aspectos.
	\item Analizar las variables de adaptación y desadaptación psicológica.
\end{unitgoals}
\end{unit}

\begin{unit}{Cuarta Unidad}{Morris,Felman}{0}{2}
\begin{topics}
	\item Definición del estrés, la calcificación, los estresares psicosociales.
	\item El proceso de enfermar por estrés. El síndrome general de adaptación.
	\item Relacionar los aspectos de las vivencias emocionales y su repercusión en el orden de lo anímico pero también de lo orgánico incluso, de lo inmunológico.
	\item Afrontamiento asertivo vs. Afrontamiento nocivo.
\end{topics}
\begin{unitgoals}
	\item Conceptualización del estrés y los procesos de enfermar en lo cotidiano.
	\item Conocer los aspectos de la Psiconeuroinmunología: emociones, inmunología, psicología.
	\item Estrés y capacidad de Afrontamiento.
	\item Estrés en la vida cotidiana y el síndrome del Bournaut o estrés laboral.
\end{unitgoals}
\end{unit}

\begin{unit}{Quinta Unidad}{Morris,Felman}{0}{2}
\begin{topics}
	\item Conceptualización, Familia, topología, adaptación, disfunción.
	\item Axiomas de la comunicación, comunicación analógica y digital, tropiezos en la comunicación, PNL.
	\item Análisis de los estilos de liderazgo en la comunicación intra e interpersonal.
\end{topics}
\begin{unitgoals}
	\item Diferenciación entre los diversos manejos de interacción en el contexto familiar.
	\item Conocimiento de los aspectos vinculados a Comunicación interpersonal.
	\item Liderazgo y comunicación.
\end{unitgoals}
\end{unit}

\begin{unit}{Sexta Unidad}{Morris,Felman}{0}{2}
\begin{topics}
	\item Personas, valores y familia.
	\item Inteligencia emocional.
	\item Sicopatología y desadaptación.
	\item Asertividad y comunicación.
	\item Autoestima y liderazgo.
\end{topics}
\begin{unitgoals}
	\item Trabajos de discusión, seminarios y talleres por grupos en diversos tópicos tratados en el curso.
\end{unitgoals}
\end{unit}



\begin{coursebibliography}
\bibfile{GeneralEducation/FG101}
\end{coursebibliography}

\end{syllabus}
