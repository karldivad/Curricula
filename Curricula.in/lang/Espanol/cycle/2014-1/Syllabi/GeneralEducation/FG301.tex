\begin{syllabus}

\course{FG301. Enseñanza Social de la Iglesia}{Obligatorio}{FG301}

\begin{justification}
El conocimiento y realización del Pensamiento Social de la Iglesia, es clave en el desarrollo personal 
y en la respuesta a la realidad peruana actual, buscando llegar a la construcción de una 
sociedad justa y reconciliada.
\end{justification}

\begin{goals}
\item \OutcomeFH
\end{goals}

\begin{outcomes}
\ExpandOutcome{FH}{2}
\end{outcomes}

\begin{unit}{Centralidad de la Persona Humana en la Cultura}{TOSO,PALUMBO}{8}{2}
\begin{topics}
	\item Naturaleza de la Doctrina Social de la Iglesia.  El Pensamiento Social de la Iglesia.
	\item La Iglesia y la sociedad.
	\item Presupuestos antropológicos y eclesiales
	\item Cultura, Centralidad de la persona humana en al cultura.
	\item Los ordenes social, económico y político, expresión de la cultura en función a la persona humana.  Instancias de pertenencia
\end{topics}
\begin{unitgoals}
	\item Comprender la naturaleza de la acción de la Iglesia en el mundo.
	\item Comprender la naturaleza del término cultura para la Iglesia.
	\item Comprender los órdenes social, económico y político insertos en al cultura.
\end{unitgoals}
\end{unit}

\begin{unit}{Perspectiva Histórica de la Doctrina Social de la Iglesia}{TOSO,PALUMBO}{7}{2}
\begin{topics}
	\item Fundamentación bíblica, Antiguo y Nuevo Testamento, la misión de Jesús, la misión de la Iglesia.
	\item Antropología y derechos.
\end{topics}
\begin{unitgoals}
	\item Comprender que los fundamentos de al Doctrina Social de la Iglesia se inspiran en la revelación.
	\item Descubrir en al historia el desarrollo de distintas acciones e instituciones como practica social de la iglesia.
\end{unitgoals}
\end{unit}

\begin{unit}{Principios y Valores de la Doctrina Social de  la Iglesia}{TOSO,PALUMBO}{7}{2}
\begin{topics}
	\item Dignidad humana.
	\item Destino universal de los bienes.
	\item Solidaridad.
	\item Subsidiaridad.
	\item Bien común.
	\item Pluralismo social. 
\end{topics}
\begin{unitgoals}
	\item Conocer y comprender los principios permanentes y valores fundamentales que están presentes en la Enseñanza Magisterial, los cuales deben ser la base para la formación de las diversas instancias sociales.
\end{unitgoals}
\end{unit}

\begin{unit}{Instancias de Pertenencia: La Familia}{TOSO,PALUMBO}{8}{2}
\begin{topics}
	\item Familia, comunión y comunidad de vida.
	\item Comunión conyugal y matrimonio. 
	\item Matrimonio e indisolubilidad.
	\item Familia y sociedad.
	\item Familia, dentro de la civilizaron del amo.
	\item Familia es sociedad natural
	\item Familia necesaria para la vida social.
	\item Características de la familia.
\end{topics}
\begin{unitgoals}
	\item Comprender que de la naturaleza social del hombre deriva, algunos órdenes sociales necesarios, como la familia.
	\item Conocer, comprender y valorar la naturaleza de la familia y el matrimonio y su rol en al sociedad.
\end{unitgoals}
\end{unit}

\begin{unit}{Comunidad Política}{TOSO,PALUMBO}{7}{2}
\begin{topics}
	\item Nación, Patria y Estado.
	\item Origen, valor, relación con a sociedad civil.
	\item Elementos constitutivos del ser de la comunidad política.
	\item Autoridad
	\item Bien común y derechos humanos.
	\item Democracia.
	\item Iglesia y estado
\end{topics}
\begin{unitgoals}
	\item Comprender que de la naturaleza social del hombre derivan, la nación y el Estado como órdenes sociales necesarios.
\end{unitgoals}
\end{unit}

\begin{unit}{Orden Económico y Trabajo}{TOSO,PALUMBO}{8}{2}
\begin{topics}
	\item Aspectos bíblicos sobre los bienes, la riqueza y la actividad económica.
	\item La globalización de la economía.
	\item Vida económica
	\item Mundialización de la economía.
	\item El padre trabaja siempre.
	\item Iglesia y nuevas características en el mundo del trabajo.
	\item Prioridad del trabajo sobre el capital.
	\item Deber y derecho del trabajo.
	\item La desocupación
	\item Derechos de los trabajadores.
	\item La huelga.
\end{topics}
\begin{unitgoals}
	\item Conocer y comprender los principios de la Doctrina Social de la Iglesia en el campo de la actividad económica.
	\item Formación de la conciencia cristiana para el posterior desenvolvimiento profesional.
	\item Comprender que los principios del Evangelio y de la ética natural pueden ser aplicados a las concreciones del orden económico de la actividad humana.
\end{unitgoals}
\end{unit}



\begin{coursebibliography}
\bibfile{GeneralEducation/FG101}
\end{coursebibliography}

\end{syllabus}
