\begin{syllabus}

\course{FG105. Apreciación de la Música}{Electivo}{FG105}

\begin{justification}
El egresado de la Universidad San Pablo, no sólo deberá ser un excelente profesional, conocedor de la más avanzada tecnología, sino también, un ser humano sensible y de amplia cultura. En esta perspectiva, el curso proporciona los instrumentos conceptuales básicos para una óptima comprensión de las obras musicales como producto cultural y artístico creado por el hombre.
\end{justification}

\begin{goals}
\item Analizar de manera crítica las diferentes manifestaciones artísticas a través de la historia identificando su naturaleza expresiva, compositiva y características estéticas así como las nuevas tendencias artísticas identificando su relación directa con los actuales indicadores socioculturales. Demostrar conducta sensible, crítica, creativa y asertiva, y conductas valorativas como indicadores de un elevado desarrollo personal.
\end{goals}

\begin{outcomes}
\ExpandOutcome{FH}{2}
\ExpandOutcome{HU}{3}
\end{outcomes}

\begin{unit}{Conceptos Básicos del Lenguaje Musical}{Aopland,Salvat,Hamel}{9}{2}
\begin{topics}
	\item La música en la vida del hombre. Concepto. El Sonido: cualidades.
	\item Los elementos de la música. Actividades y audiciones.
\end{topics}
\begin{unitgoals}
	\item Dotar al alumno de un lenguaje musical básico, que le permita apreciar y emitir un juicio con propiedad.
\end{unitgoals}
\end{unit}

\begin{unit}{Instrumentos Musicales}{Salvat,Hamel}{15}{2}
\begin{topics}
	\item La voz, el canto y sus intérpretes. Práctica de canto.
	\item Los instrumentos musicales. El conjunto instrumental.
	\item El estilo, género y las formas musicales. Actividades y audiciones.
\end{topics}
\begin{unitgoals}
	\item Que el alumno conozca, discrimine y aprecie los elementos que integran la obra de arte musical.
\end{unitgoals}
\end{unit}

\begin{unit}{La Música a Través de la Historia}{Donald,Hamel}{15}{4}
\begin{topics}
	\item El origen de la música - fuentes. La música en la antigÌedad.
	\item La música medieval: Música religiosa.  Canto Gregoriano. Música profana.
	\item El Renacimiento: Música instrumental y música vocal.
	\item El Barroco y sus representantes. Nuevos instrumentos, nuevas formas.
	\item El Clasicismo. Las formas clásicas y sus más destacados representantes.
	\item El Romanticismo y el Nacionalismo, características generales instrumentos y formas. Las escuelas nacionalistas europeas.
	\item La música contemporánea: Impresionismo, Postromanticismo, Expresionismo y las nuevas corrientes de vanguardia.
\end{topics}
\begin{unitgoals}
	\item Que el alumno conozca y distinga con precisión los diferentes momentos del desarrollo musical.
	\item Dotar al alumno de un repertorio mínimo que le permita poner en práctica lo aprendido antes de emitir una apreciación crítica de ellas.
\end{unitgoals}
\end{unit}

\begin{unit}{La Música a Través de la Historia Latinoamericana}{Donald,Bellenger,Carpio,Aretz}{6}{4}
\begin{topics}
	\item Principales corrientes musicales del Siglo XX.
	\item La música peruana: Autóctona, Mestiza, Manifestaciones musicales actuales.
	\item Música arequipeña, principales expresiones.
	\item Música latinoamericana y sus principales manifestaciones.
\end{topics}
\begin{unitgoals}
	\item Que el alumno conozca e identifique las diferentes manifestaciones populares actuales. 
	\item Que el alumno Se identifique con sus raíces musicales.
\end{unitgoals}
\end{unit}



\begin{coursebibliography}
\bibfile{GeneralEducation/FG105}
\end{coursebibliography}

\end{syllabus}
