\begin{syllabus}

\course{FG202. Apreciación Literaria}{Electivo}{FG202}

\begin{justification}
La Universidad Católica San Pablo dentro de su proyecto Educativo señala la importancia de la Formación Humana de sus alumnos, que mejor vehículo para contribuir a este objetivo que la Literatura que es un importante medio de expresión humana, a través de estas  conocemos el ama de los pueblos y el pensamiento, vivencias, sueños, sufrimientos y esperanzas del hombre a través de los tiempos.
\end{justification}

\begin{goals}
\item Este curso contribuye a entender la literatura como un medio de expresión del ser humano.
\end{goals}

\begin{outcomes}
\ExpandOutcome{FH}{2}
\ExpandOutcome{HU}{3}
\end{outcomes}

\begin{unit}{La literatura}{Jackson,Riquer}{0}{2}
\begin{topics}
	\item La comunicación Literaria
	\item Géneros Literarios
	\item El comentario y análisis de testos.
	\item El lenguaje, herramienta fundamental de la Literatura: Sus posibilidades e imposibilidades.
\end{topics}
\begin{unitgoals}
	\item Definir y caracterizar la Literatura, indicando sus géneros, recursos y lenguaje.
	\item Valorar la expresión Literaria en su esencia, adoptando una postura de apertura y sensibilidad hacia ella.
\end{unitgoals}
\end{unit}

\begin{unit}{El Clasicismo}{Jackson,Riquer}{3}{2}
\begin{topics}
	\item Homero ``La Iliada''
	\item Sófocles ``Edipo Rey''
	\item Virgilio ``La Eneida''
	\item Literatura Cristiana ``La Biblia''
\end{topics}
\begin{unitgoals}
	\item Conocer y Valorar el Clasicismo y la Literatura Clásica.
	\item Leer, comentar y apreciar fragmentos selectos de Literatura Clásica.
\end{unitgoals}
\end{unit}
\begin{unit}{Literatura Medieval}{Jackson,Riquer}{3}{2}
\begin{topics}
	\item Dante Alighieri ``La Divina Comedia'' ``Poema del Mio Cid''
	\item    San Agustín ``Confesiones''  
	\item    El Renacimiento
	\item    Shakespeare ``Hamlet''
	\item    Miguel de Cervantes ``Don Quijote''
\end{topics}
\begin{unitgoals}
	\item Señalar características básicas de la Edad Media y de la Literatura Medieval.
	\item Leer, analizar y valorar textos de literatura medieval.
	\item Caracterizar el Renacimiento
	\item Valorar textos renacentistas.
\end{unitgoals}
\end{unit}

\begin{unit}{El Romanticismo}{Jackson,Riquer}{3}{2}
\begin{topics}
	\item Goethe ``Werther''
	\item Allan Poe ``Narraciones Extraordinarias''
	\item A. Bécquer ``Rimas y Leyendas''
	\item Mariano Melgar ``Yaraví''
	\item Realismo Ruso Fedor Dostoievski 
	\item ``Crimen y Castigo''
	\item Manuel Gonzáles Prada ``Paginas Libres''
\end{topics}
\begin{unitgoals}
	\item Valorar la literatura del Romanticismo a través de la lectura de textos románticos.
	\item Caracterizar el Realismo y leer y analizar fragmentos de literatura realista.
\end{unitgoals}
\end{unit}

\begin{unit}{El Modernismo}{Jackson,Riquer}{3}{2}
\begin{topics}
	\item Rubén Darío
	\item Antonio Machado
	\item El Postmodernismo-.Gabriela Mistral, César Vallejo.
\end{topics}
\begin{unitgoals}
	\item Caracterizar e Movimiento Literario Modernista y Post-Modernista.
	\item Leer y valorar selectos textos Modernistas y Post Modernistas.
\end{unitgoals}
\end{unit}

\begin{unit}{Narrativa del Siglo XX}{Jackson,Riquer}{3}{2}
\begin{topics}
	\item Franz Kafka ``Metamorfosis''
	\item Albert Camus ``La Peste''
	\item Dramatica del siglo XX
	\item Bertolt Brecht, Garcia Lorca
	\item Poesia del siglo XX
	\item Pablo Neruda
	\item Octavio Paz
	\item Javier Heraud
\end{topics}
\begin{unitgoals}
	\item Leer, analizar y valorar selectos textos de literatura del siglo XX, ubicándolos en su respectivo contexto histórico.
\end{unitgoals}
\end{unit}

\begin{unit}{Lectura}{Jackson,Riquer}{3}{3}
\begin{topics}
	\item Gabriel García Márquez ``Cien años de Soledad''
	\item Arturo Pérez Reverte
	\item Camili José Cela
	\item Mario Vargas Llosa, Alfredo Bryce E.
	\item Otros Autores de hoy
	\item Isabel Allende, José Saramago, Paulo Coelho.
\end{topics}
\end{unit}



\begin{coursebibliography}
\bibfile{GeneralEducation/FG101}
\end{coursebibliography}
\end{syllabus}
