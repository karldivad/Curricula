\begin{syllabus}

\course{FG206. Sociología}{Electivo}{FG206}

\begin{justification}
La complejidad de la vida social y la rapidez con que se suceden los cambios sociales requieren 
de explicaciones que vayan mas allá del sentido común, en este sentido la sociología 
aporta con nuevas ideas y perspectivas de explicación a los problemas que la modernidad 
ha ido generando.
\end{justification}

\begin{goals}
\item Identificar las tendencias clasicas y actuales en sociología.
\end{goals}

\begin{outcomes}
\ExpandOutcome{HU}{3}
\end{outcomes}

\begin{unit}{La Sociedad}{Giddens,Gelles}{0}{2}
    \begin{topics}
      \item La sociedad
      \item Factores que identifican a las sociedades
      \item Evolución de las sociedades
      \item La sociología como ciencia
      \item La Relación Principal Realidad Social y conocimiento de la realidad social.
    \end{topics}
    \begin{unitgoals}
      \item Discutir el concepto de sociedad y los elementos que permiten clasificarla
      \item La sociología como Ciencia.
    \end{unitgoals}
\end{unit}

\begin{unit}{La Investigación}{Giddens,Gelles}{0}{2}
    \begin{topics}
      \item Proceso de Investigación
      \item Rol de la teoría en la Investigación
      \item La investigación cuantitativa
      \item Encuesta
      \item Censo
      \item Modelos
      \item La investigación cualitativa
      \item Grupo focal
      \item Observación
      \item Entrevista
      \item La historia de vida
    \end{topics}
    \begin{unitgoals}
      \item Analizar los métodos de la Investigación Sociológica.
    \end{unitgoals}
\end{unit}

\begin{unit}{Procesos de Socialización}{Giddens,Gelles}{0}{2}
    \begin{topics}
      \item La cultura una visión Global.
      \item La socialización a través del curso de la vida.
      \item Grupos y Organizaciones sociales
      \item Las estructuras sociales
    \end{topics}
    \begin{unitgoals}
      \item Analizar los procesos de socialización y factores que la condicionan.
    \end{unitgoals}
\end{unit}

\begin{unit}{Organización de la Sociedad}{Giddens,Gelles}{0}{2}
    \begin{topics}
      \item Estratificación social
      \item Estratificación Racial
      \item Estratificación de genero
    \end{topics}
    \begin{unitgoals}
      \item Analizar las formas en que se organiza y estratifica la sociedad.
    \end{unitgoals}
\end{unit}

\begin{unit}{Cambios en la Sociedad}{Giddens,Gelles}{0}{2}
    \begin{topics}
      \item Globalización
      \item Impacto en nuestras vidas
      \item Los efectos en las sociedades subdesarrolladas
      \item Las desigualdades sociales
    \end{topics}
    \begin{unitgoals}
      \item Análisis de un mundo en cambio
    \end{unitgoals}
\end{unit}

\begin{unit}{Comunicación y Tecnología}{Giddens,Gelles}{0}{2}
    \begin{topics}
      \item Los medios de comunicación
      \item El Impacto de la Televisión
      \item Las teorías de la comunicación social
      \item Nuevas tecnologías de comunicación
      \item Globalización y medios de comunicación
    \end{topics}
    \begin{unitgoals}
      \item Análisis de un mundo en cambio
    \end{unitgoals}
\end{unit}



\begin{coursebibliography}
\bibfile{GeneralEducation/FG101}
\end{coursebibliography}

\end{syllabus}
