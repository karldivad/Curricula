\begin{syllabus}

\course{FG211. Ética Profesional}{Obligatorio}{FG211}

\begin{justification}
La ética es una parte constitutiva inherente al ser humano, y  como tal debe plasmarse en el 
actuar cotidiano y profesional de la persona humana. Es indispensable que la persona humana 
asuma su centralidad y rol en la sociedad pues os sistemas económico, político y social no 
siempre están en función de ella entendida como lo que realmente es, una persona humana con 
dignidad y derechos.

Contribución a la formación humana: Comprender que la realización personal implica  un 
discernimiento constante para el buen ejercicio de la libertad en la consecución del bien 
personal y social.

Contribución a la formación profesional: Comprender la carrera profesional elegida como 
una actitud de servicio y como contribución en la edificación de la sociedad, actividad 
en la que podemos construir  y cualificar personalmente la sociedad que deseamos heredar 
a nuestros hijos, viviendo cotidianamente en el actuar profesional capaz de reconocer y 
afrontar de manera integral las exigencias de una moralidad madura.

\end{justification}

\begin{goals}
\item Aportar en la formación de los estudiantes capaces de afrontar el reto de participar 
en el desarrollo económico social de la ciudad, región, país y comunidad global así 
como ampliar los criterios de discernimiento en la toma de decisiones profesional de 
manera que no respondan solamente a criterios técnicos sino que incorporen en toda 
decisión cuestionamientos de orden moral, para el reconocimiento de la persona 
humana como centro del trabajo profesional.
\end{goals}

\begin{outcomes}
\ExpandOutcome{e}{2}
\ExpandOutcome{g}{4}
\ExpandOutcome{TASDSH}{3}
\end{outcomes}

\begin{unit}{Objetividad moral}{Schmidi}{8}{2}
\begin{topics}
      \item Ser Profesional y ser moral.
      \item La objetividad moral y la formulación de principios morales.
      \item El profesional y sus valores.
      \item La conciencia moral de la persona.
\end{topics}

\begin{unitgoals}
	\item Presentar al estudiante la importancia de tener y aplicar principios y valores en la sociedad actual.
	\item Presentar algunos de los principios que podrían contribuir en la sociedad de ser aplicados y vividos día a día.
\end{unitgoals}
\end{unit}

\begin{unit}{Ética y Nuevas Tecnologías}{Hernandez2006, Bilbao2006}{8}{2}
\begin{topics}
      \item Ética profesional frente a la ética general.
      \item Principios de la ética profesional.
      \item Trabajo y profesión en los tiempos actuales.
      \item Ética, ciencia y tecnología.
      \item Valores éticos en la era de la Sociedad de la Información.
      \item La Utilización de la Información.
\end{topics}
\end{unit}

\begin{unit}{Aplicaciones Prácticas}{Bilbao2006, Foley2002, IEEE2010CodeOfEthics}{6}{3}
\begin{topics}
      \item La ética en Informática.
      \item Ética y Software.
      \item Software como producto intangible.
      \item Calidad del producto.
      \item Responsabilidad ante empleadores y clientes.
      \item El software y plataformas libres.
      \item Derechos de Autor y patentes.
      \item Copia y Escritura.
      \item Copia y Escritura.
      \item Auditoria Informática.
      \item Regulación y Ética de Telecomunicaciones.
      \item Ética en Internet.
      \item Ética en los procesos de innovación tecnológica.
      \item Ética en la gestión tecnológica y en empresas de base tecnológica.
      \item Principales desafíos y posibilidades futuras: poder, libertad y control en lo telecomunicativo.
\end{topics}
\end{unit}



\begin{coursebibliography}
\bibfile{GeneralEducation/FG211}
\end{coursebibliography}

\end{syllabus}
