\begin{syllabus}

\course{FG221. Historia de la Ciencia y Tecnología}{Obligatorio}{FG221}

\begin{justification}
Contemplada en su esencia, la tecnología (técnica) es un proceso histórico universal, en el cual el hombre descompone la realidad en sus elementos y funciones elementales, formando a partir de éstos nuevas estructuras más aptas para sus fines específicos.
El fin positivo de este hecho es el dominio del hombre,  supuesto este dominio, podrá vivir experiencialmente su propia libertad. Este fin no llega a realizarse, en gran parte a causa de la falta de respeto mutuo entre los hombres y a causa de la falta de respeto a la naturaleza, a causa en fin, de la opresión, de la explotación y de la destrucción mutua.
Por esta razón, se impone la tarea de hacerse aptos para la configuración responsable del poder técnico. Y este aprendizaje se logrará por medio de una estructura social solidaria y en régimen de compañerismo. Pero, sin la correspondiente aceptación de la experiencia dolorosa de la técnica, difícilmente se tendrá éxito.
\end{justification}

\begin{goals}
\item Desarrollar capacidades y habilidades para que el alumno tenga un pensamiento crítico acerca de  la ciencia y tecnología, las cuales deben estar al servicio del hombre.  [\Familiarity]
\end{goals}

\begin{outcomes}
    \item \ShowOutcome{g}{3}
    \item \ShowOutcome{ñ}{3}
\end{outcomes}

\begin{competences}
    \item \ShowCompetence{C10}{g,ñ}
    \item \ShowCompetence{C20}{ñ}
\end{competences}

\begin{unit}{}{Primera Unidad: Amanecer de la Ciencia}{Asimov1997,Asimov1992,Artigas1992,Morande2009,Comellas2007,Childe1996}{6}{C10,C20}
\begin{topics}
  \item Introducción.
    \begin{subtopics}
      \item ?`Qué es la ciencia?
      \item ?`Qué es la tecnología?
    \end{subtopics}
  \item Amanecer de la Ciencia.
    \begin{subtopics}
      \item Prehistoria.
      \item El fuego.
      \item Los metales
      \item La agricultura.
      \item La rueda.
      \item Medios de transporte.
      \item Efectos de la tecnología primitiva.
    \end{subtopics}
\end{topics}
\begin{learningoutcomes}
	\item Comprender y diferenciar lo que es Ciencia y Tecnología. [\Familiarity]
	\item Analizar el papel de la técnica en la organización de la civilización antigua.[\Familiarity]
\end{learningoutcomes}
\end{unit}

% \begin{unit}{}{Segunda Unidad: Desarrollo de la Ciudades}{Reale1995,Asimov1997,Comellas2007}{9}{C10,C20}
% \begin{topics}
%     \item Mesopotamia: genios de la antigüedad.
%     \item La tecnología en Egipto.
%     \item La Ciencia en Alejandría.
%     \item La Ciencia de Grecia y Tecnología romana.
% \end{topics}
% \begin{learningoutcomes}
%     \item Comprender el papel de la técnica en la organización de la civilización antigua y diferenciar los aportes de cada cultura a la ciencia y tecnología.[\Familiarity]
% \end{learningoutcomes}
% \end{unit}
% 
% \begin{unit}{}{Tercera Unidad: Tecnología en la edad Media y Renacimiento}{Solar2003,Hubenák,Asimov1997,Comellas2007,}{6}{C10,C20}
% \begin{topics}
%     \item Tecnología en la edad Media.
%     \begin{subtopics}
% 	\item La pólvora y las armas de fuego en la Edad Media .
% 	\item Las Universidades .
% 	\item La tecnología en China.
%     \end{subtopics}
%     \item Tecnología en el Renacimiento.
%     \begin{subtopics}
% 	\item Imprenta.
% 	\item La revolución Científica.
%     \end{subtopics}
% \end{topics}
% \begin{learningoutcomes}
%       \item Reconocer los avances de la ciencia y tecnología en la edad media. [\Familiarity]
%       \item Distinguir los procesos y cambios en el mundo rural y urbano así como la difusión de la ciencia en el renacimiento. [\Usage]
% \end{learningoutcomes}
% \end{unit}
% 
% \begin{unit}{}{Cuarta Unidad: Tecnología en la edad Moderna}{Cruz,Solar2003,Comellas2007}{6}{C10,C20}
% \begin{topics}
%   \item La tecnología en la edad Moderna.
%   \begin{subtopics}
%     \item El siglo de las luces.
%     \item Orígenes del maquinismo.
%     \item La I Revolución Industrial.
%     \item La II Revolución Industrial: características y consecuencias.
%   \end{subtopics}
% \end{topics}
% \begin{learningoutcomes}
%     \item Determinar los principales aportes de la ciencia a la sociedad en la época moderna.[\Familiarity]
%     \item Describir los aportes de los principales científicos y el cambio de la sociedad industrial. [\Usage]
% \end{learningoutcomes}
% \end{unit}
% 
% \begin{unit}{}{Quinta Unidad: Tecnología en la edad Contemporánea}{Hubenák,Solar2003,Comellas2007}{9}{C10,C20}
% \begin{topics}
%   \item La era atómica.
%     \begin{subtopics}
%       \item La ecuación de Einstein.
%       \item Descubrimiento de la penicilina.
%       \item El Universo en expansión.
%       \item Primera bomba atómica.
%       \item Richard Feiynman.
%       \item Un pie en la luna.
%       \item Teoría del caos.
%     \end{subtopics}
%   \item La era de la Información
%     \begin{subtopics}
%       \item Inteligencia artificial y robots.
%       \item Partículas subatómicas.
%       \item Ingeniería genética.
%       \item Fecundación in vitro.
%       \item Clonación.
%       \item Nanotecnología.
%       \item El genoma humano.
%       \item Energías renovables.
%       \item Frente al cambio climático.
%     \end{subtopics}
% \end{topics}
% \begin{learningoutcomes}
%     \item Identifica el  papel de la tecnología en el neoliberalismo, y determinar su influencia en el mundo globalizado.[\Usage]
%     \item Analizar críticamente el papel de la ciencia y tecnología en el mundo actual. [\Usage]
% \end{learningoutcomes}
% \end{unit}
% 
% \begin{unit}{}{Sexta Unidad: Análisis Crítico de la Tecnología}{SumoPontifice,Huxley2003}{12}{C10,C20}
% \begin{topics}
%     \item Tecnología y Mundo actual.
%     \item Explorando lo que es la tecnología.
%     \item El Medio Tecnologizado.
%     \item El espacio, el tiempo y el cambio tecnológico.
%     \item La dimensión antropológica y cultural de la tecnología.
%     \item Lo artificial.
%     \item La mentalidad tecnologísta.
%     \item Entre la esclavitud y el relativismo ético.
%     \item La nueva idolatría: La Tecno-Idolatría.
% \end{topics}
% \begin{learningoutcomes}
%     \item Analizar  y debatir acerca del papel de la Tecnología en el desarrollo de la sociedad del futuro y establecer  sus implicancias. [\Assessment]
% \end{learningoutcomes}
% \end{unit}



\begin{coursebibliography}
\bibfile{GeneralEducation/FG221}
\end{coursebibliography}

\end{syllabus}
