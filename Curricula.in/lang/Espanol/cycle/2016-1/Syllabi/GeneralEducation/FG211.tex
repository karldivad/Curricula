\begin{syllabus}

\course{FG211. Ética Profesional}{Obligatorio}{FG211}

\begin{justification}
La ética es una parte constitutiva inherente al ser humano, y como tal debe plasmarse en el actuar cotidiano y profesional de la persona humana. Es indispensable que la persona asuma su rol activo en la sociedad pues los sistemas económico-industrial, político y social no siempre están en función de valores y principios, siendo éstos en realidad los pilares sobre los que debería basarse todo el actuar de los profesionales.
\end{justification}

\begin{goals}
\item Que el alumno amplíe sus propios criterios personales de discernimiento moral en el quehacer profesional, de forma que no sólo tome en cuenta los criterios técnicos pertinentes sino que incorpore a sí mismo cuestionamientos de orden moral y se adhiera a una ética profesional correcta, de forma que sea capaz de aportar positivamente en el desarrollo económico y social de la ciudad, región, país y comunidad global.[\Usage]
\end{goals}

\begin{outcomes}
    \item \ShowOutcome{n}{2}
    \item \ShowOutcome{ñ}{2}
	\item \ShowOutcome{o}{2}
\end{outcomes}

\begin{competences}
    \item \ShowCompetence{C10}{n,ñ,o}
    \item \ShowCompetence{C20}{n,ñ,o}
    \item \ShowCompetence{C21}{n,ñ,o}
    \item \ShowCompetence{C22}{n,ñ,o}
\end{competences}

\begin{unit}{}{Objetividad moral}{ACM1992,Schmidt1995,Loza2000,Argandon2006}{12}{C10,C21}
\begin{topics}
	\item Ser profesional y ser moral.
	\item La objetividad moral y la formulación de principios morales.
	\item El profesional y sus valores.
	\item La conciencia moral de la persona.
	\item El aporte de la DSI en el quehacer profesional.
	\item El bien común y el principio de subsidiaridad.
	\item Principios morales y propiedad privada.
	\item Justicia: Algunos conceptos básicos.
\end{topics}
\begin{learningoutcomes}
	\item Presentar al alumno la importancia de tener principios y valores en la sociedad actual.[\Usage]
	\item Presentar algunos de los principios de podrían contribuir en la sociedad de ser aplicados y vividos día a día. [\Usage]
	\item Presentar a los alumnos el aporte de la Doctrina Social de la Iglesia en el quehacer profesional. [\Usage]
\end{learningoutcomes}
\end{unit}

\begin{unit}{}{Liderazgo, Responsabilidad Individual}{ACM1992,Manzone2007,Schmidt1995,Perez1998,Nieburh2003}{12}{C20,C22}
\begin{topics}
	\item La responsabilidad individual del trabajador en la empresa.
	\item Liderazgo y ética profesional en el entorno laboral.
	\item Principios generales sobre la colaboración en hechos inmorales.
	\item El profesional frente al soborno: ?`víctima o colaboración?

\end{topics}
\begin{learningoutcomes}
	\item Presentar al alumno el rol de la responsabilidad social individual y del liderazgo en la empresa. [\Familiarity]
	\item Conocer el juicio de la ética frente a la corrupción y sobornos como forma de relación laboral. [\Familiarity]
	\item Presentar la profesión como una forma de realización personal, y como consecuencia. []
\end{learningoutcomes}
\end{unit}

\begin{unit}{}{Ética y Nuevas Tecnologías}{ACM1992,IEEE2004,Hernandez2006}{12}{C10,C20,C21}
\begin{topics}
	\item La ética profesional frente a la ética general.
	\item Trabajo y profesión en los tiempos actuales.
	\item Ética, ciencia y tecnología.
	\item Valores éticos en organizaciones relacionadas con el uso de la información.
	\item Valores éticos en la era de la Sociedad de la Información.
\end{topics}
\begin{learningoutcomes}
	\item Presentar al alumno las interrelaciones entre ética y las disciplinas de la última era tecnológica.[\Familiarity]
\end{learningoutcomes}
\end{unit}

\begin{unit}{}{Aplicaciones prácticas}{Comunicaciones2002,Hernandez2006,ACM1992}{12}{C21,C22}
\begin{topics}
    \item Ética informática.
	\begin{subtopics}
	    \item Ética y software.
	    \item El software libre.
	\end{subtopics}
    \item Regulación y ética de telecomunicaciones.
	\begin{subtopics}
	    \item Ética en Internet.
	\end{subtopics}
    \item Derechos de autor y patentes.
    \item Ética en los servicios de consultoría.
    \item Ética en los procesos de innovación tecnológica.
    \item Ética en la gestión tecnológica y en empresas de base tecnológica.
\end{topics}
\begin{learningoutcomes}
	\item Presentar al alumno algunos aspectos que confrontan la ética con el quehacer de las disciplinas emergentes en la sociedad de la información.[\Familiarity]
\end{learningoutcomes}
\end{unit}



\begin{coursebibliography}
\bibfile{GeneralEducation/FG211}
\end{coursebibliography}

\end{syllabus}
