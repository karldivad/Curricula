\begin{syllabus}

\course{MA306. Análisis Numérico}{Obligatorio}{MA306}

\begin{justification}
En este curso se estudia y analiza algoritmos numéricos que
contribuyen en la elaboración de soluciones eficientes y útiles en
diferentes áreas de las ciencias de la computación
\end{justification}

\begin{goals}
\item Se presentarán procedimientos numéricos más importantes para la resolución de 
ecuaciones no lineales, sistemas lineales y no lineales, junto con los métodos 
para la determinación de valores y vectores propios.

\item Se tratarán los temas de interpolación y aproximación de funciones y la 
derivación e integración numérica.

\item Se hará el análisis y desarrollo de métodos numéricos necesarios para la 
resolución de problemas en computación.
\end{goals}

\begin{outcomes}
\item \ShowOutcome{a}{3}
\item \ShowOutcome{i}{3}
\item \ShowOutcome{j}{4}
\end{outcomes}

\begin{unit}{}{CN1.A Introducción}{Burden02,Kinkaid94,Chapra88}{12}{3}
  \begin{topics}
      \item Aritmética de punto flotante
      \item Error, estabilidad, convergencia.
      \item Series de Taylor
   \end{topics}

   \begin{learningoutcomes}
      \item Comparar y contrastar las técnicas de análisi numérico presentadas en esta unidad. [\Usage]
      \item Definir error, estabilidad y conceptos de precision de máquinas, asi como la inexactitud de las operaciones computacionales.[\Usage]
      \item Identificar las fuentes de inexactitud en aproximaciones computacionales.[\Usage]
   \end{learningoutcomes}
\end{unit}

\begin{unit}{}{CN1.B Soluciones de ecuaciones de una variable}{Burden02,Kinkaid94}{24}{4}
\begin{topics}
      \item Soluciones iterativas para encontrar raíces (Método de Newton).
   \end{topics}
   \begin{learningoutcomes}
      \item Comparar y contrastar las técnicas de análisi numérico presentadas en esta unidad. [\Usage]
      \item Definir error, estabilidad y conceptos de precision de máquinas, asi como la inexactitud de las operaciones computacionales.[\Usage]
      \item Identificar las fuentes de inexactitud en aproximaciones computacionales.[\Usage]
   \end{learningoutcomes}
\end{unit}

\begin{unit}{}{CN1.C Interpolación y aproximación polinomial }{Burden02,Kinkaid94}{12}{4}
\begin{topics}
      \item Ajuste de curva, función de aproximación
   \end{topics}

   \begin{learningoutcomes}
      \item Comparar y contrastar las técnicas de análisi numérico presentadas en esta unidad. [\Usage]
      \item Definir error, estabilidad y conceptos de precision de máquinas, asi como la inexactitud de las operaciones computacionales.[\Usage]
      \item Identificar las fuentes de inexactitud en aproximaciones computacionales.[\Usage]
   \end{learningoutcomes}
\end{unit}

\begin{unit}{}{CN1. Diferenciación numérica e integración numérica}{Burden02,Kinkaid94,Zill02}{12}{4}
\begin{topics}
      \item Diferenciación numérica e integración (regla de Simpson)
      \item Métodos implícitos y explícitos
   \end{topics}
   \begin{learningoutcomes}
       \item Comparar y contrastar las técnicas de análisi numérico presentadas en esta unidad. [\Usage]
      \item Definir error, estabilidad y conceptos de precision de máquinas, asi como la inexactitud de las operaciones computacionales.[\Usage]
      \item Identificar las fuentes de inexactitud en aproximaciones computacionales.[\Usage]
   \end{learningoutcomes}
\end{unit}

\begin{unit}{}{CN1.E Problemas de valor inicial para ecuaciones diferenciales ordinarias}{Burden02,Kinkaid94}{24}{3}
\begin{topics}
      \item Ecuaciones diferenciales.
   \end{topics}
   \begin{learningoutcomes}
       \item Comparar y contrastar las técnicas de análisi numérico presentadas en esta unidad. [\Usage]
      \item Definir error, estabilidad y conceptos de precision de máquinas, asi como la inexactitud de las operaciones computacionales.[\Usage]
      \item Identificar las fuentes de inexactitud en aproximaciones computacionales.[\Usage]
   \end{learningoutcomes}
\end{unit}

\begin{unit}{}{CN1.F Métodos iterativos en el álgebra matricial}{Kinkaid94}{12}{3}
\begin{topics}
      \item Algebra lineal.
      \item Diferencia finita
   \end{topics}
   \begin{learningoutcomes}
      \item Comparar y contrastar las técnicas de análisi numérico presentadas en esta unidad. [\Usage]
      \item Definir error, estabilidad y conceptos de precision de máquinas, asi como la inexactitud de las operaciones computacionales.[\Usage]
      \item Identificar las fuentes de inexactitud en aproximaciones computacionales.[\Usage]
   \end{learningoutcomes}
\end{unit}




\begin{coursebibliography}
\bibfile{BasicSciences/MA306}
\end{coursebibliography}

\end{syllabus}

