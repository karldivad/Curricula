\begin{syllabus}

\course{CS411. Pasantía II}{Obligatorio}{CS411}

\begin{justification}
Se constituye en un espacio de comprobación y validación de conocimientos teóricos-prácticos desde su formación profesional, permite que el pasante se relacione con la utilización y avances de líneas técnicas y tecnológicas en las que se apoya la empresa y/o institución para su operación, hace mérito de su capacidad formativa para identificar carencias y capacidades en la empresa a través de la investigación. El pasante será participante o interviniente en los programas de demanda de las Ciencias de la Computación que aplica o desea aplicar la empresa, como planteamiento deliberante, capaz de dar a conocer el nivel de formación recibida y las proyecciones que visualiza en el plano tecnológico.  
\end{justification}

\begin{goals}
\item Potenciar los conocimientos teóricos y prácticos de formación profesional y 
canalizar los procesos técnicos de investigación y propuestas en sistemas de la 
Ciencias de la Computación. 
\end{goals}

\begin{outcomes}
\ExpandOutcome{f}{3}
\ExpandOutcome{h}{5}
\ExpandOutcome{n}{3}
\end{outcomes}

\begin{unit}{Pasantia I}{Pasantia}{75}{5}
   \begin{topics}
      \item Observación, apoyo y verificación de los procesos prácticos-técnicos que desarrolla la empresa en computación.
   \end{topics}

   \begin{unitgoals}
      \item Que el alumno tenga una experiencia en el campo laboral que le permita consolidar los conocimientos adquiridos en su carrera.
   \end{unitgoals}
\end{unit}
   
\begin{coursebibliography}
\bibfile{Computing/CS/CS410}
\end{coursebibliography}
\end{syllabus}
