\begin{syllabus}

\course{CS392. T�picos Avanzados en Ingenier�a de Software}{Electivos}{CS392}

\begin{justification}
Este curso es �til en la formaci�n profesional para que el alumno tenga contacto con t�picos especializados  en el �rea de Ingenier�a de Software.
\end{justification}

\begin{goals}
\item Los alumnos deben conocer los modelos de confiabilidad de software y la aplicaci�n de los m�todos de an�lisis probabil�sticas para un sistema de software.
\item Los alumnos deben identificar y aplicar la redundancia y tolerancia a fallas para un sistema de software.
\item El alumno aplicar� las diferentes t�cnicas de verificaci�n formal a software con baja complejidad.
\item Los alumnos utilizar�n un lenguaje de especificaci�n formal para especificar un sistema de software y demostrar� los beneficios de una perspectiva de calidad.
\item Familiarizar a los alumnos con los principios reconocidos para la construcci�n de componentes de software de alta calidad.
\item Aplicar m�todos orientados a componentes para el dise�o de un software.
\end{goals}

\begin{outcomes}
\ExpandOutcome{b}{4}
\ExpandOutcome{c}{4}
\ExpandOutcome{d}{4}
\ExpandOutcome{e}{4}
\ExpandOutcome{g}{3}
\ExpandOutcome{h}{4}
\ExpandOutcome{i}{3}
\ExpandOutcome{k}{5}
\ExpandOutcome{l}{3}
\end{outcomes}

\begin{unit}{\SESoftwareReliabilityDef}{Dean96,Baugh92,Lau03,Sitaraman00,Neufleder93,peled2001}{11}{4}
   \SESoftwareReliabilityAllTopics
   \SESoftwareReliabilityAllObjectives
\end{unit}

\begin{unit}{\SEFormalMethodsDef}{Dean96,Baugh92,Lau03,Sitaraman00,Neufleder93,peled2001}{12}{2}
   \SEFormalMethodsAllTopics
   \SEFormalMethodsAllObjectives
\end{unit}

\begin{unit}{\SEComponentBasedComputingDef}{Dean96,Baugh92,Lau03,Sitaraman00,Neufleder93,peled2001}{19}{4}
   \SEComponentBasedComputingAllTopics
   \SEComponentBasedComputingAllObjectives
\end{unit}



\begin{coursebibliography}
\bibfile{Computing/CS/CS392}
\end{coursebibliography}

\end{syllabus}
