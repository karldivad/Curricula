\begin{syllabus}

\course{FG221. Historia de la Ciencia y Tecnología}{Obligatorio}{FG221}

\begin{justification}

Contemplada en su esencia, la tecnología (técnica) es un proceso histórico universal, en el cual el hombre descompone la realidad en sus elementos y funciones elementales, formando a partir de éstos nuevas estructuras más aptas para sus fines específicos.
El fin positivo de este hecho es el dominio del hombre,  supuesto este dominio, podrá vivir experiencialmente su propia libertad. Este fin no llega a realizarse, en gran parte a causa de la falta de respeto mutuo entre los hombres y a causa de la falta de respeto a la naturaleza, a causa en fin, de la opresión, de la explotación y de la destrucción mutua.
Por esta razón, se impone la tarea de hacerse aptos para la configuración responsable del poder técnico. Y este aprendizaje se logrará por medio de una estructura social solidaria y en régimen de compañerismo. Pero, sin la correspondiente aceptación de la experiencia dolorosa de la técnica, difícilmente se tendrá éxito.
\end{justification}

\begin{goals}
\item Desarrollar capacidades y habilidades para que el alumno tenga un pensamiento crítico acerca de  la ciencia y tecnología, las cuales deben estar al servicio del hombre.
\end{goals}

\begin{outcomes}
\ShowOutcome{e}{2}
\ShowOutcome{f}{2}
\ShowOutcome{g}{1}
\ShowOutcome{h}{4}
\ShowOutcome{j}{2}
\ShowOutcome{FH}{2}
\ShowOutcome{HU}{2}
\end{outcomes}

\begin{unit}{Historia de la Tecnología Primitiva}{Morande,Lakatos,Comellas,Ratzinger,JuanPabloB}{9}{2}
\begin{topics}
      \item Introducción.
      \item Tecnología Primitiva.
\end{topics}

\begin{learningoutcomes}
	\item Comprender y diferenciar lo que es Ciencia y Tecnología. 
 	\item Comprender el papel de la técnica en la organización de la civilización antigua.
\end{learningoutcomes}
\end{unit}

\begin{unit}{Historia de la Tecnología Antigua}{Comellas}{6}{2}
\begin{topics}
    \item Desarrollo de las ciudades.
\end{topics}
\begin{learningoutcomes}
    \item Comprender el papel de la técnica en la organización de la civilización antigua y comparar cada una de estas culturas.
\end{learningoutcomes}
\end{unit}

\begin{unit}{Historia de la Tecnología Antigua}{Artigas2007,Comellas}{6}{2}
\begin{topics}
    \item Tecnología en la edad media.
    \item Tecnología en el renacimiento.
    \item Tecnología en el barroco.
\end{topics}
\begin{learningoutcomes}
    \item Comprender el  papel de la tecnología en la conquista de la tierra y la difusión que se dio en el período del renacimiento.
\end{learningoutcomes}

\end{unit}

\begin{unit}{Historia de la Ciencia y Tecnología en la Edad Moderna}{Comellas}{9}{2}
\begin{topics}
      \item {La tecnología en la edad moderna.}
\end{topics}
\begin{learningoutcomes}
	\item Comprender el  papel de la tecnología en el desarrollo del imperialismo e identificar sus consecuencias.
\end{learningoutcomes}
\end{unit}

\begin{unit}{Historia de la Ciencia y La Tecnología en la Edad Contemporánea}{Comellas}{6}{2}
\begin{topics}
      \item {Tecnología en la edad contemporánea.}
\end{topics}
\begin{learningoutcomes}
	\item Comprender el  papel de la tecnología en el neoliberalismo, y determinar la influencia de la ingeniería de telecomunicaciones y la informática en un mundo globalizado.
\end{learningoutcomes}
\end{unit}

\begin{unit}{Filosofía de la tecnología}{Huxley,JuanPabloC,Comellas}{12}{4}
\begin{topics}
      \item {Análisis crítico de la tecnología.}
\end{topics}
\begin{learningoutcomes}
	\item Analizar  y debatir acerca del papel de la Tecnología en el desarrollo de la sociedad del futuro y establecer  sus implicancias. 
\end{learningoutcomes}
\end{unit}



\begin{coursebibliography}
\bibfile{GeneralEducation/FG101}
\end{coursebibliography}

\end{syllabus}

%\end{document}
