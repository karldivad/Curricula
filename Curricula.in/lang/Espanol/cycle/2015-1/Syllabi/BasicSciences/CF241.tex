\begin{syllabus}

\curso{CF241. Física General III}{Obligatorio}{CF241}

\begin{justification}
Mostrar un alto grado de dominio de las leyes de la Electrostática y Magnetostática. Entender y aplicar los conceptos de energía eléctrica y magnética. Mostrar un alto grado de dominio de las leyes de Inducción electromagnética, leyes de Maxwell y ondas electromagnéticas. Poseer capacidad y habilidad en la interpretación de los fenómenos electromagnéticos, que contribuyan en la elaboración de soluciones eficientes y útiles en diferentes áreas de la ciencia de la computación. 
\end{justification}

\begin{goals}
\item  Que el alumno aprenda y domine los principios fluidos estaticos y en movimientode la Electrostática y Magnetostática.
\item  Que el alumno aprenda y domine los conceptos de energía eléctrica y magnética.
\item  Que el alumno aprenda y domine las leyes de Inducción electromagnética.
\item  Que el alumno aprenda a aplicar principios de las leyes de Maxwell para desarrollar modelos computacionales.
\end{goals}

\begin{outcomes}
\ExpandOutcome{a}
\ExpandOutcome{i}
\ExpandOutcome{j}
\end{outcomes}

\begin{unit}{FI1. Carga e Interacción Eléctrica}{Sears99, Fishbane99}{4}
   \begin{topics}
         \item  Carga eléctrica.
	 \item  Experimentos de Thomson y Millikan.
         \item  Conductores y aislantes.
	 \item  Ley de Coulomb.
         \item  Fuerza eléctrica y el principio de superposición.
         \item  Distribuciones continuas de carga.
   \end{topics}

   \begin{unitgoals}
         \item  Entender y caracterizar los procesos de fuerza eléctrica.
         \item  Resolver problemas.
   \end{unitgoals}
\end{unit}

\begin{unit}{FI2. Campo Eléctrico y Ley de Gauss}{Sears99, Fishbane99}{8}
   \begin{topics}
         \item  Noción de campo eléctrico.  Intensidad de campo eléctrico.
	 \item  Definición operacional de campo eléctrico y líneas de fuerza.
         \item  Campo eléctrico de una distribución continua de carga.
	 \item  Campo eléctrico de un dipolo y de dieléctricos cargados.
         \item  Flujo eléctrico.
	 \item  Ley de Gauss.
         \item  Forma diferencial de la ley de Gauss.
	 \item  Campo eléctrico de dieléctricos y conductores cargados usando la ley de Gauss.
   \end{topics}

   \begin{unitgoals}
         \item  Entender, caracterizar y aplicar el concepto de campo eléctrico y la ley de Gauss
         \item  Resolver problemas
   \end{unitgoals}
\end{unit}

\begin{unit}{FI3. Potencial eléctrico y energía electrostática}{Sears99, Fishbane99, Resnick98}{14}
   \begin{topics}
         \item  Función potencial eléctrico. Diferencia de potencial.
	 \item  Potencial eléctrico de dieléctricos y conductores cargados.
         \item  Relación entre el potencial eléctrico y el campo eléctrico.
	 \item  Rotacional de E.
         \item  Energía potencial electrostática de un sistema de cargas puntuales.
	 \item  Energía potencial de una distribución continua de carga.
         \item  Energía potencial de un sistema de conductores.
	 \item  El campo eléctrico como reservorio de energía.
         \item  Dipolo en un campo eléctrico.
         \item  Fuerza entre conductores.  Presión electrostática.
         \item  Capacidad de un conductor. Condensadores.
         \item  Condensadores en serie y en paralelo.  Circuitos equivalentes.
         \item  Energía electrostática de un condensador.
         \item  Dieléctrico en condensadores.  Polarización eléctrica.  Vector desplazamiento D.
         \item  Ley de Gauss con dieléctricos.
         \item  Ley de Coulomb en dieléctricos.
   \end{topics}

   \begin{unitgoals}
         \item  Entender, caracterizar y aplicar el concepto de potencial eléctrico y la ley de Gauss.
         \item  Entender, caracterizar y aplicar el concepto de energía potencial electrostática.
         \item  Resolver problemas.
   \end{unitgoals}
\end{unit}

\begin{unit}{FI4 Corriente, Resitencia, Fuerza electromotriz. Circuitos}{Serway98, Fishbane99}{6}
   \begin{topics}
         \item  Corriente y densidad de corriente.
	 \item  Flujo de carga a través de una superficie.
         \item  Conservación de la carga en movimiento. Ley de continuidad.
	 \item  Conductividad eléctrica y la ley de Ohm.
         \item  Resistividad y resistencia.  Efecto Joule.
	 \item  Fuerza electromotriz (f.e.m.).
         \item  Leyes de Kirchhoff.
	 \item  Circuitos equivalentes.
   \end{topics}

   \begin{unitgoals}
         \item  Entender y aplicar los conceptos de ley de Ohm y leyes de Kirchoff.
         \item  Resolver problemas.
   \end{unitgoals}
\end{unit}

\begin{unit}{FI5 Campo magnético, Ley de Biot-Savart, Ley de Ampere}{Tipler98, Serway98}{12}
   \begin{topics}
         \item  Magnetismo natural. Experimento de Oersted.
	 \item  Campo magnético y su representación.
         \item  Fuerza magnética sobre una carga. Definición operacional de campo magnético.
	 \item  Fuerza magnética sobre una corriente  Efecto Hall.
         \item  Torque magnético sobre una corriente: Motor DC.
	 \item  Momento dipolar magnético.  Idea de spin.
         \item  Campo magnético debido a una distribución arbitraria de corriente:  ley de Biot y Savart.
	 \item  Fuerza entre dos alambres portadores de corrientes.
         \item  Campo magnético de una carga en movimiento.
	 \item  Campo magnético debido a varias distribuciones de corrientes.
         \item  Ley de Ampere.  Forma diferencial de la ley de Ampere.
	 \item  Magnetización de la materia. Corrientes de magnetización. Campo magnetizante H.
   \end{topics}

   \begin{unitgoals}
         \item  Entender, caracterizar y aplicar el concepto de campo magnético.
         \item  Entender, caracterizar y aplicar la ley de Ampere.
         \item  Resolver problemas.
   \end{unitgoals}
\end{unit}

\begin{unit}{FI6 Inducción electromagnética}{Tipler98, Serway98}{8}
   \begin{topics}
         \item  Resumen de las leyes de la magnetostática.
	 \item  Experimentos de Faraday.  Ley de inducción de Faraday.
         \item  Ley de Lenz.
	 \item  Inducción electromagnética y conservación de la energía: generador AC.
         \item  Representación de Maxwell-Faraday.
         \item  Inductancia: Inducción mutua y autoinducción.
         \item  Circuitos RL.  Energía magnética.
         \item  Circuitos LC y RLC. Oscilaciones.
         \item  Corriente alterna: Representación compleja. Representación mediante fasores.
	 \item  Circuitos: R, C, L. Reactancias.
         \item  Circuito RLC Serie.  Impedancia.
   \end{topics}

   \begin{unitgoals}
         \item  Entender, caracterizar y aplicar el concepto de inducción electromagnética.
         \item  Resolver problemas de circuitos R, L, C.
   \end{unitgoals}
\end{unit}

\begin{unit}{FI7 Ecuaciones de Maxwell}{Tipler98, Serway98, Resnick98}{4}
   \begin{topics}
         \item  Ley de Ampere-Maxwell.
	 \item  Forma diferencial. Corriente de desplazamiento.
         \item  Ondas electromagnéticas.
	 \item  Ecuaciones de Maxwell.
   \end{topics}

   \begin{unitgoals}
         \item  Entender, caracterizar y aplicar el concepto de las leyes de Maxwell.
         \item  Entender el concepto de ondas electromagnéticas.
         \item  Resolver problemas.
   \end{unitgoals}
\end{unit}

\begin{coursebibliography}
\bibfile{BasicSciences/CF141}
\end{coursebibliography}

\end{syllabus}
