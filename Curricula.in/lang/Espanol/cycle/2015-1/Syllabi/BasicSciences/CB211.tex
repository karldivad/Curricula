\begin{syllabus}

\course{CB211. Investigación Operativa II}{Obligatorio}{CB211}

\begin{justification}
Este curso es importante en la medida que proporciona modelos de optimización útiles para la toma de decisiones en negocios.
\end{justification}

\begin{goals}
\item Reconocer, modelar, resolver, implementar e interpretar modelos de optimización no lineales y estocásticos en problemas reales.
\end{goals}

\begin{outcomes}
\ExpandOutcome{a}{1}
\ExpandOutcome{b}{1}
\ExpandOutcome{c}{1}
\ExpandOutcome{g}{1}
\ExpandOutcome{j}{1}
\end{outcomes}

\begin{unit}{Programación no lineal}{Taha2004,Hillier2006}{12}{1}
   \begin{topics}
      \item Funciones convexas y cóncavas.
      \item Soluciones de PNL con una variable.
      \item Maximización y minimización no restringida con varias variables.
      \item Métodos de ascenso no escalonado.
      \item Condiciones de Kuhn-Tucker.
   \end{topics}

   \begin{unitgoals}
      \item resolver problemas de optimización no lineal.
   \end{unitgoals}
\end{unit}

\begin{unit}{Toma de decisiones bajo incertidumbre}{Wayne2005,Hillier2002IO}{12}{1}
   \begin{topics}
      \item Teoría de la utilidad.
      \item Arboles de decisión.
      \item Toma de decisión con objetivos múltiples.
   \end{topics}

   \begin{unitgoals}
      \item Presentar los fundamentos de la teoría de decisiones en condiciones de incertidumbre.
   \end{unitgoals}
\end{unit}

\begin{unit}{Teoría de juegos}{Taha2004,Hillier2006}{12}{1}
   \begin{topics}
      \item Juegos de suma cero.
      \item Juegos de suma no constante para dos personas.
      \item Introducción a la teoría de juegos para n personas.
   \end{topics}

   \begin{unitgoals}
      \item Resolver problemas de toma de decisiones donde dos o más personas que deciden se enfretan a un conflicto de intereses.
   \end{unitgoals}
\end{unit}

\begin{unit}{Cadenas de Markov}{Wayne2005,Hillier2002IO}{12}{1}
   \begin{topics}
      \item Procesos estocáticos.
      \item Cadenas de Markov.
      \item Clasificación de los estados de una cadena de Markov.
	  \item Propiedades a largo de las cadenas de Markov.
   \end{topics}

   \begin{unitgoals}
      \item Dar elementos de toma de decisiones para variables aleatorias que varian el tiempo.
   \end{unitgoals}
\end{unit}

\begin{unit}{Teoría de colas}{Taha2004,Hillier2006}{12}{1}
   \begin{topics}
      \item Estructura básicas de los modelos de colas.
      \item Ejemplos de sistemas de colas reales.
      \item Proceso de nacimiento y muerte.
      \item Modelos de colas basados en el proceso de nacimiento y muerte.
      \item Modelos de colas con distribuciones no exponenciales.
      \item Modelos de colas con disciplina de prioridades.
   \end{topics}

   \begin{unitgoals}
      \item Responder preguntas relacionadas a modelos de espera.
   \end{unitgoals}
\end{unit}

\begin{unit}{Simulación}{Taha2004,Hillier2006}{10}{1}
   \begin{topics}
      \item Tipos comunes de aplicaciones.
      \item Generación de números aleatorios.
      \item Estudio de casos.
      \item Simulación con hojas de cálculos.
      \item Técnicas de reducción de varianza.
      \item Método regenerativo de análisis estadístico.
   \end{topics}

   \begin{unitgoals}
      \item Poder hacer un análisis de riesgo ante situaciones de alta complejidad.
   \end{unitgoals}
\end{unit}



\begin{coursebibliography}
\bibfile{BasicSciences/CB210}
\end{coursebibliography}

\end{syllabus}
