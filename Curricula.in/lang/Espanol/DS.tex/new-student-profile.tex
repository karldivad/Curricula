\section{Perfil del ingresante}

El aspirante a ingresar a la \SchoolFullName de la \University debe tener:

\noindent Conocimientos de:
\begin{enumerate}
\item La operación básica de una computadora.
\item Conceptos básicos de matemática y computación.
\item Su entorno actual en la sociedad.
\end{enumerate}

\noindent Habilidades para:
\begin{enumerate}
\item Entender las relaciones entre los hechos y las causas que los produjeron. De esa manera, disminuir la consecuencias y así­ poder resolver problemas de una manera coherente.
\item Observar la realidad, modelarla mentalmente a través de conceptos preconcebidos, los cuales son llamados patrones.
\item Percibir las relaciones lógicas (de funcionamiento o de comportamiento) existentes entre las observaciones realizadas.
\item Expresarse de manera oral o escrita las posibles soluciones para un problema dado.
\item Concentrarse y apertura al esfuerzo.
\item Comprender, analizar y sintetizar.
\item Formar hábitos y métodos adecuados para el estudio.
\end{enumerate}

\noindent Actitudes de:
\begin{enumerate}
\item Interés natural en experimentar con nuevas herramientas, métodos y datos para resolver problemas retadores.
\item Interés y gusto por el estudio de la computación, estadistica y matemáticas.
\item Disposición para el trabajo académico, en forma cooperativa y participativa, dentro y fuera del aula de clases.
\end{enumerate}
