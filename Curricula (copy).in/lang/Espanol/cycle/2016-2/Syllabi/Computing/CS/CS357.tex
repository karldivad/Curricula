\begin{syllabus}

\course{CS357. T�picos en Computaci�n Gr�fica II}{Electivo}{CS357}

\begin{justification}
En este curso se puede profundizar en alguno de los t�picos
mencionados en el �rea de Computaci�n Gr�fica (\textit{Graphics and Visual
Computing} - GV).

�ste curso est� destinado a realizar algun curso avanzado sugerido
por la curricula de la ACM/IEEE.\nocite{Foley90,HearnAndBaker94}
\end{justification}

\begin{goals}
\item Que el alumno utilice t�cnicas de computaci�n gr�fica m�s sofisticadas que involucren estructuras de datos y algoritmos complejos.
\item Que el alumno aplique los conceptos aprendidos para crear una aplicaci�n sobre un problema real.
\item Que el alumno investigue la posibilidad de crear un nuevo algoritmo y/o t�cnica nueva para resolver un problema real.
\end{goals}

\begin{outcomes}
\ExpandOutcome{a}{1}
\ExpandOutcome{b}{1}
\ExpandOutcome{i}{1}
\ExpandOutcome{j}{1}
\end{outcomes}

\begin{itemize}
\item CS355. Computaci�n Gr�fica avanzada
\item CS356. Animaci�n por computadora
\item CS313. Algoritmos Geom�tricos
\item CS357. Visualizaci�n
\item CS358. Realidad Virtual
\item CS359. Algoritmos Gen�ticos
\end{itemize}



\begin{coursebibliography}
\bibfile{Computing/CS/CS255}

\end{coursebibliography}

\end{syllabus}
