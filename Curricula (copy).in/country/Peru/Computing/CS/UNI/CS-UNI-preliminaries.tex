\section{Antecedentes}\label{sec:cs-antecedentes}

La Computaci\'on surge a medidados del siglo XX como un \'area multidisciplinaria del conocimiento a partir del impulso en el desarrollo de las computadoras dado por la Segunda Guerra Mundial. Dicho desarrollo toma nuevos br\'ios a principios del siglo XXI producto de los avances tecnol\'ogicos y te\'oricos, as\'i como de requerimientos de procesamiento de informaci\'on de las nuevas aplicaciones y de la tecnolog\'ia, influenciando todos los \'ambitos de la sociedad.

\begin{quote}
En M\'exico, por ejemplo, el desarrollo de las ciencias de la computaci\'on se da antes que el de la ingenier\'ia, primero cuando el Centro de C\'alculo Electr\'onico de la UNAM contrata a investigadores con posgrado en computaci�n (alrededor de 1967), y posteriormente cuando se crea el Centro de Investigaci�n en Matem\'aticas Aplicadas y Sistemas, el cual se nutre mayoritariamente de los egresados de la Facultad de Ciencias, y tiene como actividad principal la investigaci\'on en computaci\'on y la atenci\'on a la docencia de esta disciplina en el marco de la Facultad de Ciencias. Posteriormente, en el a\~no de 1967, dada la industrializaci\'on de la computaci\'on en M\'exico, se crea la licenciatura de Ingeniero en Sistemas de Computaci\'on.
\end{quote} 


La Facultad de Ciencias no ha sido ajena a los avances en el desarrollo de la Computaci\'on. La primera computadora donada por la IBM a la UNI y al Per\'u (??) se entrega en 19?? al profesor H. Valqui. A partir de entonces surgen los primeros cursos de programaci\'on, donde participan acad\'emicos de diversas facultades, ya asistiendo a cursos o imparti\'endolos.

En la reforma de las carreras de la Facultad de Ciencias de la UNI de 19?? aparecen por primera vez materias de Computaci\'on de una manera formal en las carreras de Ciencias. Ya antes se impart\'ian materias de Computaci\'on en otras carreras de la Facultad de Ingenier\'ia Industrial y de Sistemas, as\'i como en las Facultades de Ingenier\'ia Civil y de Electr\'onica.
